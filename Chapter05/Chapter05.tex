\chapter {Magnetohydrodynamic Fluids}
\section{Introduction}
As we have seen in Sect.~\ref{s3}, the MHD equations take the form
\begin{eqnarray}\label{e5.1a}
\frac{d\rho}{dt} + \rho\,\nabla\cdot{\bf V} &=& 0,\\[0.5ex]\label{e5.1b}
\rho\,\frac{d{\bf V}}{dt} + \nabla p - {\bf j}\times {\bf B} &=& {\bf 0},\\[0.5ex]
{\bf E} + {\bf V} \times {\bf B} &=& {\bf 0},\\[0.5ex]
\frac{d}{dt}\!\left(\frac{p}{\rho^{\Gamma}}\right) &=& 0,\label{e5.1d}
\end{eqnarray}
where $\rho\simeq m_i\,n$ is the plasma mass density, and ${\Gamma}=5/3$ is the
ratio of specific heats. 

It is often observed that the above set of equations
are identical to the equations governing the motion of an inviscid, adiabatic,
perfectly conducting, electrically neutral
 liquid. Indeed, this observation is sometimes used as the
sole justification for the MHD equations. After all, a hot, tenuous, quasi-neutral
plasma is  highly conducting, and if the motion is sufficiently fast
then both viscosity and heat conduction can be plausibly neglected. However,
we can appreciate, from Sect.~\ref{s3}, that this is a highly 
oversimplified and misleading argument. The problem is, of course, that
a weakly coupled plasma is a far more complicated dynamical system than
a conducting liquid. 

According to the discussion in
Sect.~\ref{s3}, the MHD equations are only valid when
\begin{equation}
\delta^{-1}\,v_t \gg V \gg \delta\,v_t.
\end{equation}
Here, $V$ is the typical velocity associated with the plasma
dynamics  under investigation,
$v_t$ is the typical thermal velocity, and $\delta$ is the typical magnetization
parameter ({\em i.e.}, the typical ratio of a particle gyro-radius to
the scale-length of the motion). Clearly, the above inequality is
most likely to be satisfied in a {\em highly magnetized}\/ ({\em i.e.},
$\delta\rightarrow 0$) plasma. 

If the plasma dynamics  becomes too fast ({\em i.e.}, $V\sim \delta^{-1}\,v_t$)
then resonances occur with the motions of individual particles 
({\em e.g.}, the cyclotron resonances) which
invalidate the MHD equations. Furthermore, effects, such as electron
inertia and the Hall effect, which are not taken into account in the
MHD equations, become important. 

MHD is essentially a {\em single-fluid}\/ plasma theory. A single-fluid
approach is justified because the perpendicular motion is dominated by
${\bf E}\times{\bf B}$ drifts, which are the same for both
plasma species. Furthermore, 
the relative streaming velocity, $U_\parallel$, of
both species parallel to the magnetic field is strongly constrained by the
fundamental MHD ordering (see Sect.~\ref{s3.9})
\begin{equation}
U \sim \delta \,V.
\end{equation}
Note, however, that if the plasma dynamics  becomes too slow 
({\em i.e.}, $V\sim \delta\,v_t$) then the motions of the electron
and ion fluids become sufficiently different that a single-fluid
approach is no longer tenable. This occurs whenever the diamagnetic
velocities, which are quite different for different plasma species, become
comparable to the ${\bf E}\times{\bf B}$ velocity (see Sect.~\ref{s3.12}).
Furthermore, effects such as plasma resistivity, viscosity, and thermal
conductivity, which are not taken into account in the MHD equations, become
important in this limit. 

Broadly speaking, the MHD equations describe relatively {\em violent}, 
{\em large-scale}\/
motions of highly {\em magnetized}\/ plasmas. 

Strictly speaking, the MHD equations are only valid in  {\em collisional}\/
plasmas ({\em i.e.},  plasmas in which the mean-free-path is much smaller
than the typical variation scale-length). However, as discussed in Sect.~\ref{s3.13},
the MHD equations also fairly well describe the {\em perpendicular}\/ (but not
the {\em parallel}\,!)  motions of  collisionless
plasmas.

Assuming that the MHD equations are valid, let us now investigate their properties.

\section{Magnetic Pressure}
The MHD equations can be combined with Maxwell's equations,
\begin{eqnarray}\label{e5.4a}
\nabla\times{\bf B} &=& \mu_0\,{\bf j},\\[0.5ex]
\nabla\times{\bf E} &=& - \frac{\partial{\bf B}}{\partial t},\label{e5.4b}
\end{eqnarray}
to form a closed set. The displacement current is neglected in
Eq.~(\ref{e5.4a}) on the reasonable assumption that MHD motions are slow
compared to the velocity of light. Note that Eq.~(\ref{e5.4b}) guarantees
that $\nabla\cdot{\bf B}=0$, provided that this relation is
presumed to hold initially. Similarly, the assumption of
quasi-neutrality renders the Poisson-Maxwell equation,
$\nabla\!\cdot\!{\bf E} = \rho_c/\epsilon_0$, irrelevant.

Equations~(\ref{e5.1b}) and (\ref{e5.4a}) can be combined to give the MHD equation
of motion:
\begin{equation}
\rho\,\frac{d{\bf V}}{dt} = -\nabla p + \nabla\!\cdot\!{\bf T},
\end{equation}
where
\begin{equation}
T_{ij} = \frac{B_i\,B_j- \delta_{ij} \,B^2/2}{\mu_0}.
\end{equation}

Suppose that the magnetic field is approximately uniform, and
directed along the $z$-axis. In this case, the above equation of
motion reduces to
\begin{equation}
\rho\,\frac{d{\bf V}}{dt} = -\nabla\!\cdot\!{\bf P},
\end{equation}
where
\begin{equation}
{\bf P} = \left(\begin{array}{ccc}
p + B^2/2\mu_0 & 0\\[0.5ex]
0 &p + B^2/2\mu_0& 0\\[0.5ex]
0&0& p - B^2/2\mu_0\end{array}
\right).
\end{equation}
Note that the magnetic field {\em increases}\/ the plasma pressure, by an
amount $B^2/2\mu_0$, in directions {\em perpendicular}\/ to the magnetic field,
and {\em decreases}\/ the plasma pressure, by the same amount, in the
{\em parallel}\/ direction. Thus, the magnetic field gives rise to
a {\em magnetic pressure}, $B^2/2\,\mu_0$, acting {\em perpendicular}\/ to field-lines,
and a {\em magnetic tension}, $B^2/2\,\mu_0$, acting 
{\em along}\/ field-lines. 
Since, as we shall see presently, the plasma is tied to magnetic field-lines, 
it follows that magnetic field-lines embedded in an
MHD plasma  act rather like
{\em mutually repulsive elastic bands}. 

\section{Flux Freezing}\label{s5.3}
The MHD Ohm's law,
\begin{equation}\label{e5.9}
{\bf E} + {\bf V}\times {\bf B} = {\bf 0},
\end{equation}
is sometimes referred to as the {\em perfect conductivity}\/ equation, for
obvious reasons, and sometimes as the {\em flux freezing}\/ equation. 
The latter nomenclature comes about because Eq.~(\ref{e5.9}) implies that the magnetic
flux  through any closed contour in the plasma, each element of
which moves with the local plasma velocity, is a {\em conserved quantity}. 

In order to verify the above assertion, let us consider the
magnetic flux, ${\Psi}$, through a contour, $C$, which is co-moving
with the plasma:
\begin{equation}
{\Psi} = \int_{\bf S} {\bf B}\!\cdot\!d{\bf S}.
\end{equation}
Here, ${\bf S}$ is some surface which spans $C$. The time rate of
change of ${\Psi}$ is made up of two parts. Firstly, there
is the part due to the time variation of ${\bf B}$  over the
surface ${\bf S}$. This can be written
\begin{equation}
\left(\frac{\partial{ \Psi}}{\partial t}\right)_1 = 
\int_{\bf S} \frac{\partial{\bf B}}{\partial t}\!\cdot\!d{\bf S}.
\end{equation}
Using the Faraday-Maxwell equation, this reduces to
\begin{equation}
\left(\frac{\partial{ \Psi}}{\partial t}\right)_1 = -
\int_{\bf S} \nabla \times{\bf E} \!\cdot\!d{\bf S}.
\end{equation}
Secondly, there is the part due to the motion of $C$. If $d{\bf l}$
is an element of $C$ then ${\bf V}\times d{\bf l}$ is the area swept out
by $d{\bf l}$ per unit time. Hence, the flux crossing this area is
${\bf B} \!\cdot\!{\bf V}\times d{\bf l}$.
It follows that 
\begin{equation}
\left(\frac{\partial{ \Psi}}{\partial t}\right)_2 = \int_C {\bf B} \!\cdot\!{\bf V}\times d{\bf l} = \int_C {\bf B}\times{\bf V} \!\cdot\!d{\bf l}.
\end{equation}
Using Stokes's theorem, we obtain
\begin{equation}
\left(\frac{\partial{ \Psi}}{\partial t}\right)_2= \int_{\bf S} \nabla\times
({\bf B}\times
{\bf V})\!\cdot\!d{\bf S}.
\end{equation}
Hence, the total time rate of change of ${\Psi}$ is given by
\begin{equation}
\frac{d{\Psi}}{dt} = - \int_{\bf S}\nabla\times\left({\bf E} + {\bf V}\times
{\bf B}\right)\!\cdot\!d{\bf S}.
\end{equation} The condition 
\begin{equation}
{\bf E} + {\bf V}\times{\bf B} ={\bf 0}
\end{equation}
clearly implies that ${\Psi}$ remains constant in time
for any arbitrary contour.
This, in turn, implies that magnetic field-lines must move with the
plasma. In other words, the field-lines are {\em frozen}\/ into the plasma. 

A {\em flux-tube}\/ is defined as a topologically   cylindrical volume whose
sides are defined by magnetic field-lines. Suppose that, at some initial
time, a flux-tube is embedded in the plasma. According to the flux-freezing
constraint,
\begin{equation}
\frac{d{\Psi}}{dt} = 0,
\end{equation}
the subsequent motion of the plasma and the magnetic field is always
such as to maintain the integrity of the flux-tube. Since magnetic
field-lines can be regarded as infinitely thin flux-tubes, we conclude that
MHD  plasma motion also maintains the integrity of field-lines. In other words,
magnetic field-lines embedded in an MHD plasma can never break and reconnect:
{\em i.e.}, MHD forbids any change in {\em topology}\/ of the field-lines. It turns
out that this is an extremely restrictive constraint. Later on, we shall discuss
situations in which this constraint is  relaxed. 

\section{MHD Waves}\label{s5.4}
Let us investigate the small amplitude waves which  propagate through
a spatially uniform MHD plasma. We start by combining Eqs.~(\ref{e5.1a})--(\ref{e5.1d}) and (\ref{e5.4a})--(\ref{e5.4b}) to form a
closed set of equations:
\begin{eqnarray}
\frac{d\rho}{dt} + \rho\,\nabla\cdot{\bf V} &=& 0,\\[0.5ex]
\rho\,\frac{d{\bf V}}{dt} + \nabla p - \frac{(\nabla\times{\bf B})\times {\bf B}}{\mu_0} &=& {\bf 0},\\[0.5ex]
-\frac{\partial{\bf B}}{\partial t}+ \nabla\times(
{\bf V} \times {\bf B}) &=& {\bf 0},\\[0.5ex]
\frac{d}{dt}\!\left(\frac{p}{\rho^{\Gamma}}\right) &=& 0.
\end{eqnarray}
Next, we linearize these equations (assuming, for the sake of
simplicity, that the equilibrium
flow velocity and equilibrium plasma current are both zero) to give
\begin{eqnarray}\label{e5.19a}
\frac{\partial\rho}{\partial t} + \rho_0\,\nabla\cdot{\bf V} &=& 0,\\[0.5ex]
\rho_0\,\frac{\partial {\bf V}}{\partial t} + \nabla p - 
\frac{(\nabla\times{\bf B})\times {\bf B}_0}{\mu_0} &=& {\bf 0},\\[0.5ex]
-\frac{\partial{\bf B}}{\partial t}+ \nabla\times(
{\bf V} \times {\bf B}_0) &=& {\bf 0},\\[0.5ex]
\frac{\partial}{\partial t}\!\left(\frac{p}{p_0} - \frac{{\Gamma}\,\rho}{\rho_0}\right) &=& 0.\label{e5.19d}
\end{eqnarray}
Here, the subscript 0 denotes an equilibrium quantity. Perturbed quantities
are written without subscripts. Of course, $\rho_0$, $p_0$,  and ${\bf B}_0$ are 
constants in a spatially uniform plasma. 

Let us search for wave-like solutions of Eqs.~(\ref{e5.19a})--(\ref{e5.19d}) in which perturbed
quantities vary like $\exp[\,{\rm i}\,({\bf k}\!\cdot\!{\bf r} - \omega t)]$. 
It follows that
\begin{eqnarray}
-\omega\,\rho + \rho_0\,{\bf k}\!\cdot\!{\bf V} &=& 0,\\[0.5ex]\label{e5.20b}
-\omega\,\rho_0\,{\bf V} + {\bf k}\,p - \frac{ ({\bf k}\times{\bf B})\times
{\bf B}_0}{\mu_0} &=& {\bf 0},\\[0.5ex]
\omega\,{\bf B} + {\bf k}\times({\bf V}\times{\bf B}_0) &= &{\bf 0},\\[0.5ex]
-\omega\left(\frac{p}{p_0} - \frac{{\Gamma}\,\rho}{\rho_0}\right) &=& 0.
\end{eqnarray}
Assuming that $\omega\neq 0$, the above equations yield
\begin{eqnarray}\label{e5.21a}
\rho &=& \rho_0\,\frac{{\bf k}\!\cdot\!{\bf V}}{\omega},\\[0.5ex]\label{e5.21b}
p &=& {\Gamma}\,p_0\,\frac{{\bf k}\!\cdot\!{\bf V}}{\omega},\\[0.5ex]
{\bf B} &=& \frac{ ({\bf k}\!\cdot\!{\bf V})\,{\bf B}_0
- ({\bf k}\!\cdot\!{\bf B}_0)\,{\bf V}}{\omega}.\label{e5.21c}
\end{eqnarray}
Substitution of these expressions into the linearized equation of
motion, Eq.~(\ref{e5.20b}), gives
\begin{eqnarray}\label{e5.22}
\left[ \omega^2 - \frac{({\bf k}\!\cdot\!{\bf B}_0)^2}{\mu_0\,\rho_0}
\right] {\bf V} &=& \left\{ \left[ \frac{{\Gamma}\,p_0}{\rho_0} 
+ \frac{B_0^{~2}}{\mu_0\,\rho_0}\right] {\bf k}
- \frac{({\bf k}\!\cdot\!{\bf B}_0)}{\mu_0\,\rho_0}\,{\bf B}_0
\right\} ({\bf k}\!\cdot\!{\bf V})\nonumber
\\[0.5ex]
&& - \frac{({\bf k}\!\cdot\!{\bf B}_0)\,
({\bf V}\!\cdot\!{\bf B}_0)}{\mu_0\,\rho_0} \,{\bf k}.
\end{eqnarray}

We can assume, without loss of generality, that the equilibrium magnetic
field  ${\bf B}_0$ 
is directed along the $z$-axis, and that the wave-vector ${\bf k}$ lies
in the $x$-$z$ plane. Let $\theta$ be the angle subtended between ${\bf B}_0$ and
${\bf k}$. Equation~(\ref{e5.22}) reduces to the eigenvalue
equation
\begin{equation}
\left(
\begin{array}{ccc}
{\scriptstyle \omega^2 - k^2\,V_A^{~2} -k^2\,V_S^{~2}\,\sin^2\theta}&
0& {\scriptstyle -k^2\,V_S^{~2}\,\sin\theta
\cos\theta}\\[0.5ex]
0&{\scriptstyle \omega^2-k^2\,V_A^{~2}\,\cos^2\theta} & 0\\[0.5ex]
{\scriptstyle -k^2\,V_S^{~2}\,\sin\theta\,\cos\theta} &0& 
{\scriptstyle \omega^2-k^2\,V_S^{~2}\,\cos^2\theta}
\end{array}\right)\!\left(\begin{array}{c}V_x\\[0.5ex]
V_y\\[0.5ex] V_z\end{array}\right) = {\bf 0}.\label{e5.23}
\end{equation}
Here,
\begin{equation}\label{e5.24}
V_A = \sqrt{\frac{B_0^{~2}}{\mu_0\,\rho_0}}
\end{equation}
is the {\em Alfv\'{e}n speed}, and 
\begin{equation}\label{e5.25}
V_S = \sqrt{\frac{{\Gamma}\,p_0}{\rho_0}}
\end{equation}
is the {\em sound speed}. The solubility condition for Eq.~(\ref{e5.23}) is that
the determinant of the square matrix is zero. This yields the dispersion
relation
\begin{equation}\label{e5.26}
(\omega^2 - k^2\,V_A^{~2}\,\cos^2\theta)\left[
\omega^4 - \omega^2\,k^2\,(V_A^{~2}+V_S^{~2}) + k^4\,V_A^{~2}\,V_S^{~2}\,\cos^2\theta
\right] = 0.
\end{equation}

There are {\em three}\/ independent roots of the above dispersion relation, 
corresponding to the three different types of wave that can propagate through an
MHD plasma. The first, and most obvious, root is
\begin{equation}\label{e5.27}
\omega = k\,V_A\,\cos\theta,
\end{equation}
which has the associated eigenvector $(0,V_y, 0)$. This
root is characterized by both ${\bf k}\cdot{\bf V} =0$ and ${\bf V}\cdot
{\bf B}_0=0$. It immediately follows from Eqs.~(\ref{e5.21a}) and (\ref{e5.21b}) that there is
{\em zero}\/   perturbation of the plasma density or pressure
associated with this root. In fact, this root can easily be
identified as the {\em shear-Alfv\'{e}n wave}, which was
introduced in Sect.~\ref{s4.8}. Note
that the properties of the shear-Alfv\'{e}n wave in a warm ({\em i.e.}, non-zero
pressure) plasma are unchanged from  those we found earlier in a cold plasma. 
Note, finally, that since the shear-Alfv\'{e}n wave only involves plasma
motion {\em perpendicular}\/ to the magnetic field, we can expect the
dispersion relation (\ref{e5.27}) to hold good in a {\em collisionless}, as well as a
collisional, plasma. 

The remaining two roots of the dispersion relation (\ref{e5.26}) are written
\begin{equation}\label{e5.28}
\omega = k\,V_+,
\end{equation}
and
\begin{equation}\label{e5.29}
\omega = k\,V_-,
\end{equation}
respectively.
Here,
\begin{equation}
V_\pm = \left\{\frac{1}{2}\left[V_A^{~2} + V_S^{~2} \pm \sqrt{
(V_A^{~2} + V_S^{~2})^2 - 4\, V_A^{~2}\,V_S^{~2}\,\cos^2\theta}
\right]\right\}^{1/2}.
\end{equation}
Note that $V_+\geq V_-$. The first root is generally termed the
{\em fast magnetosonic wave}, or fast wave, for short, whereas
the second root is usually called the {\em slow magnetosonic
wave}, or slow wave. The eigenvectors for these waves are
$(V_x,0,V_z)$. It follows that ${\bf k}\cdot{\bf V} \neq 0$ and ${\bf V}\cdot
{\bf B}_0\neq0$. Hence, these waves are associated with non-zero perturbations
in the plasma density and pressure, and also involve plasma motion parallel, as
well as perpendicular, to the magnetic field. The latter observation suggests
that the dispersion relations
(\ref{e5.28}) and (\ref{e5.29}) are likely to undergo significant modification in
collisionless plasmas.

In order to better understand the nature of the fast and slow waves, let us
consider the {\em cold-plasma}\/ limit, which is obtained by letting the sound
speed $V_S$ tend to zero. In this limit, the slow wave ceases to exist (in fact,
its phase velocity tends to zero) whereas the dispersion relation for
the fast wave reduces to
\begin{equation}
\omega = k\,V_A.
\end{equation}
This can be identified as the dispersion relation for the 
{\em compressional-Alfv\'{e}n wave}, which was introduced in Sect.~\ref{s4.8}.
Thus, we can identify the fast wave as the compressional-Alfv\'{e}n wave
modified by a non-zero plasma pressure. 

In the limit $V_A\gg V_S$, which is appropriate to low-$\beta$ plasmas (see
Sect.~\ref{s3.13}), the dispersion relation for the slow wave reduces to
\begin{equation}
\omega \simeq k\,V_S\,\cos\theta.
\end{equation}
This is actually the dispersion relation of a sound wave propagating
{\em along}\/ magnetic field-lines. Thus, in low-$\beta$ plasmas the slow
wave is a sound wave modified by the presence of the magnetic field.

The distinction between the fast and slow waves can be further understood
by comparing the signs of the wave induced fluctuations in the  plasma and magnetic
pressures: $p$ and ${\bf B}_0\!\cdot\! {\bf B}/\mu_0$, respectively. 
It follows from Eq.~(\ref{e5.21c}) that
\begin{equation}\label{e5.33}
\frac{{\bf B}_0\!\cdot\!{\bf B}}{\mu_0} = \frac{{\bf k}\!\cdot\!{\bf V}\,B_0^{~2}
- ({\bf k}\!\cdot\!{\bf B}_0)\,({\bf B}_0\!\cdot\!{\bf V})}
{\mu_0\,\omega}.
\end{equation}
Now, the $z$- component of Eq.~(\ref{e5.20b}) yields
\begin{equation}\label{e5.34}
\omega\,\rho_0\,V_z = k\,\cos\theta\,p.
\end{equation}
Combining Eqs.~(\ref{e5.21b}), (\ref{e5.24}), (\ref{e5.25}), (\ref{e5.33}), and (\ref{e5.34}), we obtain
\begin{equation}
\frac{{\bf B}_0\!\cdot\!{\bf B}}{\mu_0}
 =\frac{V_A^{~2}}{V_S^{~2}} \left(1-\frac{k^2\,V_S^{~2}\,\cos^2\theta}
{\omega^2}\right)p.
\end{equation}
Hence, $p$ and ${\bf B}_0\!\cdot\! {\bf B}/\mu_0$ have the same sign
if $V>V_S \,\cos\theta$, and the opposite sign if
$V<V_S\,\cos\theta$. Here, $V=\omega/k$ is the phase velocity. It is
straightforward to show that $V_+> V_S\,\cos\theta$, and $V_-<V_S\,\cos\theta$.
Thus, we conclude that in the fast magnetosonic wave the pressure and
magnetic energy fluctuations reinforce one another, whereas  the
fluctuations oppose one another in the slow magnetosonic wave.

\begin{figure}
\epsfysize=3in
\centerline{\epsffile{Chapter05/mhdwave.eps}}
\caption{\em Phase velocities of the three MHD waves in the $x$-$z$ plane.}\label{f18}
\end{figure}

Figure~\ref{f18} shows the phase velocities of the three MHD waves plotted in the
$x$-$z$ plane for a low-$\beta$ plasma in which $V_S<V_A$. It can be
seen that the slow wave always has a smaller phase velocity than the
shear-Alfv\'{e}n wave, which, in turn, always has a smaller phase
velocity than the fast wave.

\section{The Solar Wind}
The {\em solar wind}\/ is a high-speed particle stream continuously blown out from
the Sun into interplanetary space. It extends  far beyond the orbit
of the Earth, and terminates in a shock front, called the {\em heliopause},
where it interfaces with the weakly ionized interstellar medium. The heliopause
is predicted to lie between 110 and
 160 AU (1 Astronomical Unit is $1.5\times 10^{11}$\,m)
from the centre of the Sun. Voyager 1 is expected to pass through the
heliopause sometime in the next decade: hopefully, it will still be
functional at that time\,!

In the vicinity of the Earth, ({\em i.e.}, at about 1 AU from the
Sun) the solar wind velocity typically
ranges between 300 and 1400 ${\rm km}\,{\rm s}^{-1}$. The average value
is approximately $500\,{\rm km}\,{\rm s}^{-1}$, which corresponds to about a
4 day time of flight from the Sun. Note that the solar wind is
both {\em super-sonic}\/ and {\em super-Alfv\'{e}nic}. 

The solar wind is predominately composed of protons and electrons. 

Amazingly enough, the solar wind was predicted
theoretically by Eugine Parker\footnote{E.N.~Parker, Astrophys.\ J.\ {\bf 128},
664 (1958).}
 a number of years {\em before}\/ its
existence 
was confirmed using satellite data.\footnote{M.~Neugebauer, C.W.~Snyder, J.~Geophys.\
Res.\ {\bf 71}, 4469 (1966).} Parker's prediction of a super-sonic
outflow of gas from the Sun  is a
fascinating scientific detective story, as well as a wonderful application
of plasma physics. 

The solar wind originates from the {\em solar corona}. The solar corona
is a hot, tenuous plasma surrounding the Sun, with characteristic temperatures and
particle densities of about $10^6$\,K and $10^{14}\,{\rm m}^{-3}$,
respectively. Note that the corona is far hotter than the solar
atmosphere, or {\em photosphere}. In fact, the
temperature of the photosphere is  only about $6000$\,K. It is
thought that the corona is heated by Alfv\'{e}n waves emanating from the
photosphere. The solar corona is most easily observed during a total
solar eclipse, when it is visible as a white filamentary region
immediately surrounding the Sun. 

Let us start, following Chapman,\footnote{S.~Chapman, Smithsonian Contrib.\
Astrophys.\ {\bf 2}, 1 (1957).} by attempting
to construct a model for a {\em static}\/ solar corona. The equation
of hydrostatic equilibrium for the corona takes the form 
\begin{equation}\label{e5.36}
\frac{dp}{dr} = - \rho\,\frac{G\,M_\odot}{r^2},
\end{equation}
where $G= 6.67\times 10^{-11}\,{\rm m}^{3}\,{\rm s}^{-2}\,{\rm kg}^{-1}$
is the gravitational constant, and $M_\odot=2\times 10^{30}\,{\rm kg}$ is
the solar mass. 
The plasma density is written 
\begin{equation}\label{e5.37}
\rho\simeq n\,m_p, 
\end{equation}
where $n$ is the number
density of protons. If both protons and electrons are assumed
to possess a common temperature, $T(r)$, then the coronal pressure is
given by
\begin{equation}\label{e5.38}
p = 2\,n\,T.
\end{equation}

The thermal conductivity of the corona is dominated by the electron thermal
conductivity, and takes the form [see Eqs.~(\ref{e3.74}) and (\ref{e3.87a})]
\begin{equation}
\kappa = \kappa_0\,T^{5/2},
\end{equation}
where $\kappa_0$ is a relatively weak function of density and
temperature. For typical coronal conditions this conductivity is
extremely high: {\em i.e.}, it is about twenty times the thermal
conductivity of copper at room temperature. The coronal heat flux density
is written
\begin{equation}
{\bf q} = - \kappa\,\nabla T.
\end{equation}
For a static corona, in the absence of energy sources or sinks, we require
\begin{equation}
\nabla\!\cdot\!{\bf q} = 0.
\end{equation}
Assuming spherical symmetry, this expression reduces to
\begin{equation}
\frac{1}{r^2}\frac{d}{dr}\!\left(r^2\,\kappa_0\,T^{5/2}\,\frac{dT}{dr}\right) =0.
\end{equation}
Adopting the sensible boundary condition that the coronal temperature must
tend to zero at large distances from the Sun, we obtain
\begin{equation}\label{e5.43}
T(r) = T(a)\left(\frac{a}{r}\right)^{2/7}.
\end{equation}
The reference level $r=a$ is conveniently taken to be the
base of the corona, where $a\sim 7\times 10^5\,{\rm km}$, $n\sim
2\times 10^{14}\,{\rm m}^{-3}$, and $T\sim 2\times 10^{6}$\,K. 

Equations (\ref{e5.36}), (\ref{e5.37}), (\ref{e5.38}), and (\ref{e5.43}) can be combined and
integrated to give
\begin{equation}
p(r) = p(a) \exp\left\{\frac{7}{5}\,\frac{G\,M_\odot\,m_p}{2\,T(a)\,a}
\left[\left(\frac{a}{r}\right)^{5/7}-1\right]\right\}.
\end{equation}
Note that as $r\rightarrow\infty$ the coronal pressure tends towards a finite
constant value:
\begin{equation}
p(\infty) = p(a)\,\exp\left\{-\frac{7}{5}\,\frac{G\,M_\odot\,m_p}{2\,T(a)\,a}
\right\}.
\end{equation}
There is, of course, nothing at large distances from the Sun which could
contain such a pressure (the pressure of the interstellar medium is
negligibly small). Thus, we conclude, with Parker, that the
static coronal model is {\em unphysical}. 

Since we have just demonstrated that a static model of the solar corona is
unsatisfactory, let us now attempt to construct a {\em dynamic}\/ model
in which material flows outward from the Sun. 

\section{Parker Model of  Solar Wind}
By symmetry, we expect a purely radial coronal outflow.
The radial momentum conservation equation for the corona takes the form
\begin{equation}\label{e5.46}
\rho\,u\,\frac{du}{dr} = -\frac{dp}{dr} - \rho\,\frac{G\,M_\odot}{r^2},
\end{equation}
where $u$ is the radial expansion speed. 
The continuity equation reduces to
\begin{equation}\label{e5.47}
\frac{1}{r^2}\frac{d(r^2\,\rho\,u)}{dr} = 0.
\end{equation}
In order to obtain a closed set of equations, we now need to adopt an equation
of state for the corona, relating the pressure, $p$, and the density, $\rho$. For
the  sake of simplicity, we adopt the simplest conceivable equation
of state, which corresponds to an {\em isothermal}\/ corona. Thus,
we have
\begin{equation}\label{e5.48}
p = \frac{2\,\rho\,T}{m_p},
\end{equation}
where $T$ is a constant. Note that more realistic
equations of state complicate the analysis, but do not significantly modify
any of the physics results.

Equation (\ref{e5.47}) can be integrated to give
\begin{equation}\label{e5.49}
r^2\,\rho\,u = I,
\end{equation}
where $I$ is a constant. The above expression simply states that the mass flux per
unit solid angle, which takes the value $I$, is independent of the radius, $r$.
Equations~(\ref{e5.46}), (\ref{e5.48}), and (\ref{e5.49}) can be combined together to give
\begin{equation}\label{e5.50}
\frac{1}{u} \,\frac{du}{dr}\left(u^2 - \frac{2\,T}{m_p}\right)
= \frac{4\,T}{m_p\,r} - \frac{G\,M_\odot}{r^2}.
\end{equation}
Let us restrict our attention to coronal temperatures which satisfy
\begin{equation}\label{e5.51}
T < T_c \equiv \frac{G\,M_\odot\,m_p}{4\,a},
\end{equation}
where $a$ is the radius of the base of the corona. For typical
coronal parameters (see above), $T_c\simeq 5.8\times 10^6$\,K, which
is certainly greater than the temperature of the corona at $r=a$. For
$T<T_c$, the right-hand side of Eq.~(\ref{e5.50}) is negative for
$a<r<r_c$, where
\begin{equation}\label{e5.52}
\frac{r_c}{a} = \frac{T_c}{T},
\end{equation}
and positive for $r_c<r<\infty$. The right-hand side of (\ref{e5.50}) is zero at
$r=r_c$, implying that the left-hand side is also zero
at this radius, which is usually termed the ``critical radius.''
There are two ways in which the left-hand side of (\ref{e5.50}) can be zero at the critical
radius. Either
\begin{equation}\label{e5.53}
u^2(r_c) = u_c^{~2} \equiv \frac{2\,T}{m_p},
\end{equation}
or
\begin{equation}\label{e5.54}
\frac{du(r_c)}{dr} = 0.
\end{equation}
Note that $u_c$ is the coronal {\em sound speed}. 

As is easily demonstrated, if Eq.~(\ref{e5.53}) is satisfied then $du/dr$ has the
same sign for all $r$, and $u(r)$ is either a monotonically
increasing, or a monotonically decreasing, function of $r$. On the other
hand, if Eq.~(\ref{e5.54}) is satisfied then $u^2-u_c^{~2}$ has the same
sign for all $r$, and $u(r)$ has an extremum close to $r=r_c$. The flow
is either super-sonic for all $r$, or sub-sonic for all $r$. These
possibilities lead to the existence of {\em four}\/ classes of solutions
to Eq.~(\ref{e5.50}), with the 
following properties:
\begin{enumerate}
\item $u(r)$ is sub-sonic throughout the domain $a<r<\infty$. $u(r)$
increases with $r$, attains a maximum value around $r=r_c$, and then
decreases with $r$. 
\item a unique solution for which $u(r)$ increases monotonically 
with $r$, and $u(r_c) = u_c$. 
\item a unique solution for which $u(r)$ decreases monotonically 
with $r$, and $u(r_c) = u_c$. 
\item $u(r)$ is super-sonic throughout the domain $a<r<\infty$. 
 $u(r)$
decreases with $r$, attains a minimum value around $r=r_c$, and then
increases with $r$. 
\end{enumerate}
These four classes of solutions are illustrated in Fig.~\ref{f19}.

\begin{figure}
\epsfysize=4in
\centerline{\epsffile{Chapter05/class1.eps}}
\caption{\em The four classes of Parker outflow solutions for the solar wind.}\label{f19}
\end{figure}

Each of the classes of solutions described above fits a different
set of boundary conditions at $r=a$ and $r\rightarrow \infty$. The
{\em physical}\/ acceptability of these solutions depends on these
boundary conditions. For example, both Class~3 and Class~4 solutions can
be ruled out as plausible models for the solar corona since they predict
{\em super-sonic}\/ flow at the base of the corona, which is not observed, and is
also not consistent with a static solar photosphere. Class~1 and Class~2 solutions
remain acceptable models for the solar corona on the basis of their
properties around $r=a$, since they both predict sub-sonic flow in this region.
However, the Class~1 and Class~2 solutions behave quite differently
 as $r\rightarrow\infty$, and the physical acceptability of these two
classes hinges on this difference.

Equation~(\ref{e5.50}) can be rearranged to give
\begin{equation}
\frac{du^2}{dr}\left(1-\frac{u_c^{~2}}{u^2}\right) = \frac{4\,u_c^{~2}}{r}
\left(1-\frac{r_c}{r}\right),
\end{equation}
where use has been made of Eqs.~(\ref{e5.51}) and (\ref{e5.52}). 
The above expression can be integrated to give
\begin{equation}\label{e5.56}
\left(\frac{u}{u_c}\right)^2 -\ln\!\left(\frac{u}{u_c}\right)^2 = 4\,\ln r
+ 4\,\frac{r_c}{r} + C,
\end{equation}
where $C$ is a constant of integration. 

Let us consider the behaviour
of Class~1 solutions in the limit $r\rightarrow\infty$. It is
clear from Fig.~\ref{f19} that, for Class~1 solutions, $u/u_c$ is less than unity and monotonically
decreasing as $r\rightarrow\infty$. In the large-$r$ limit, Eq.~(\ref{e5.56})
reduces to
\begin{equation}
\ln\frac{u}{u_c} \simeq -2\,\ln r,
\end{equation}
so that
\begin{equation}
u\propto \frac{1}{r^2}.
\end{equation}
It follows from Eq.~(\ref{e5.49}) that the coronal density, $\rho$, approaches
a finite, constant value, $\rho_\infty$, as $r\rightarrow\infty$. Thus,
the Class~1 solutions yield a finite pressure,
\begin{equation}
p_\infty= \frac{2\,\rho_\infty\,T}{m_p},
\end{equation}
at large $r$, which cannot be matched to the much smaller pressure of the
interstellar medium. Clearly, Class~1 solutions are unphysical.

Let us consider the behaviour of the Class~2 solution
in the limit $r\rightarrow\infty$. It is
clear from Fig.~\ref{f19} that, for the Class~2 solution, $u/u_c$ is greater than unity and monotonically
increasing as $r\rightarrow\infty$. In the large-$r$ limit,
Eq.~(\ref{e5.56}) reduces to
\begin{equation}
\left(\frac{u}{u_c}\right)^2 \simeq 4\,\ln r,
\end{equation}
so that
\begin{equation}
u \simeq 2\,u_c\,(\ln r)^{1/2}.
\end{equation}
It follows from Eq.~(\ref{e5.49}) that $\rho\rightarrow 0$ and 
$r\rightarrow\infty$. Thus, the Class~2 solution yields
$p\rightarrow 0$ at large $r$, and  can, therefore,  be matched to the low
pressure interstellar medium. 

We conclude that the only solution to Eq.~(\ref{e5.50}) which is consistent
with physical boundary conditions at $r=a$ and $r\rightarrow\infty$ is
the Class~2 solution. This solution predicts that the
solar corona expands radially outward at
relatively modest, sub-sonic velocities close to the Sun,
 and gradually accelerates
to super-sonic velocities as it moves further away from the Sun.
Parker termed this continuous, super-sonic expansion of the corona
the {\em solar wind}. 

Equation~(\ref{e5.56}) can be rewritten
\begin{equation}\label{e5.62}
\left[\frac{u^2}{u_c^{~2}}-1\right] -\ln\frac{u^2}{u_c^{~2}}
= 4\,\ln\frac{r}{r_c} + 4\left[\frac{r_c}{r}-1\right],
\end{equation}
where the constant $C$ is determined  by demanding that
$u=u_c$ when $r=r_c$. Note that both $u_c$ and $r_c$ can be evaluated
in terms of the coronal temperature $T$ via Eqs.~(\ref{e5.52}) and (\ref{e5.53}).
Figure~\ref{f20} shows $u(r)$ calculated  from Eq.~(\ref{e5.62}) for various values
of the coronal temperature. It can be seen that plausible
values of $T$ ({\em i.e.}, $T\sim 1$--$2\times 10^6$\,K) yield
expansion speeds of several hundreds of kilometers per second
at 1~AU, which  accords well with satellite observations. The critical
surface  at which the solar wind makes the transition from sub-sonic to
super-sonic flow is predicted to lie a few solar radii away from the Sun
({\em i.e.}, $r_c\sim 5\,R_\odot$). Unfortunately, the Parker model's
prediction for the density of the solar wind at the Earth is significantly
too high compared to satellite observations. Consequently, there have
been many further developments of this model. In particular, the
unrealistic assumption that the solar wind plasma is isothermal has been relaxed, and
two-fluid effects have been incorporated into the analysis.\footnote{{\em Solar Magnetohydrodynamics}, E.R.~Priest, (D.~Reidel
Publishing Co., Dordrecht, Netherlands, 1987).}

\begin{figure}
\epsfysize=4in
\centerline{\epsffile{Chapter05/parker1.eps}}
\caption{\em Parker outflow solutions for the solar wind.}\label{f20}
\end{figure}

\section{Interplanetary  Magnetic Field}\label{s5.7}
Let us now investigate how the solar wind and  the interplanetary magnetic
field affect one another. 

The hot coronal plasma making up the
solar wind possesses an extremely high electrical conductivity. In such a
plasma, we expect the concept of ``frozen-in'' magnetic field-lines, discussed
in Sect.~\ref{s5.3}, to be applicable. The continuous flow of coronal material into
interplanetary space must, therefore, result in the transport of the solar
magnetic field into the interplanetary region. If the Sun did not rotate,
the resulting magnetic configuration would be very simple. The radial
coronal expansion considered above (with the neglect of any magnetic forces)
would produce magnetic field-lines extending radially outward from the Sun.

Of course, the Sun does rotate, with a (latitude dependent) period of
about 25 days.\footnote{To an observer orbiting with the Earth, the
rotation period appears to be about 27 days.} Since the solar photosphere is an
excellent electrical conductor, the magnetic field at the base of the
corona is frozen into the rotating frame of reference of the Sun.
A magnetic field-line starting from a given location on the surface of the Sun is
drawn out along the path followed by the element of the solar
wind emanating from that location. As before, let us suppose that the coronal expansion is
purely radial in a stationary frame of reference.
Consider a spherical
polar coordinate system $(r,\theta,\phi)$ which {\em co-rotates}\/ with the Sun.
Of course, the symmetry axis of the coordinate system is assumed to coincide
with the axis of the Sun's rotation. In the rotating coordinate system,
the velocity components of the solar wind are written
\begin{eqnarray}
u_r &=& u,\\[0.5ex]
u_\theta &=& 0,\\[0.5ex]
u_\phi &=& - {\Omega}\,r\,\sin\theta,
\end{eqnarray}
where ${\Omega}= 2.7\times 10^{-6}\,{\rm rad\,sec}^{-1}$ is the angular
velocity of solar rotation. The azimuthal velocity $u_\phi$ is entirely
due to the transformation to the rotating frame of reference. The stream-lines
of the flow satisfy the differential equation
\begin{equation}\label{e5.64}
\frac{1}{r\,\sin\theta}\frac{dr}{d\phi} \simeq \frac{u_r}{u_\phi} = -\frac{u}{
{\Omega}\,r\,\sin\theta}
\end{equation}
at constant $\theta$. The stream-lines are also magnetic field-lines,
so Eq.~(\ref{e5.64}) can also
be regarded as the differential equation of
a magnetic field-line. For radii $r$ greater than several times the
critical radius, $r_c$, the solar wind solution (\ref{e5.62}) predicts that $u(r)$
is almost constant (see Fig.~\ref{f20}). Thus, for $r\gg r_c$ it is reasonable
to write $u(r) = u_s$, where $u_s$ is a constant. Equation~(\ref{e5.64})
can then be integrated to give the equation of a magnetic field-line:
\begin{equation}
r-r_0 = -\frac{u_s}{{\Omega}}\,(\phi-\phi_0),
\end{equation}
where the field-line is assumed to pass through the point $(r_0,\theta,\phi_0)$.
Maxwell's equation $\nabla\!\cdot\!{\bf B} =0$, plus the assumption of a spherically
symmetric magnetic field, easily yields the following expressions for the
components of the interplanetary magnetic field:
\begin{eqnarray}\label{e5.66a}
B_r(r,\theta,\phi) &=& B(r_0,\theta,\phi_0)\left(\frac{r_0}{r}\right)^2,\\[0.5ex]
B_\theta(r,\theta,\phi) &=& 0,\\[0.5ex]
B_\phi(r,\theta,\phi)&=& - B(r_0,\theta,\phi_0)\,\frac{{\Omega}\,r_0}{u_s}
\,\frac{r_0}{r}\,\sin\theta.
\end{eqnarray}

Figure~\ref{f21} illustrates the interplanetary magnetic field close to the
ecliptic plane. The magnetic field-lines of the Sun are drawn into spirals
(Archemedian spirals, to be more exact) by the solar rotation. Transformation
to a stationary frame of reference give the same magnetic field configuration,
with the addition of an electric field
\begin{equation}
{\bf E} = - {\bf u} \times {\bf B} = -u_s\,B_\phi\,\hat{\btheta}.
\end{equation}
The latter field arises because the radial plasma flow is no longer parallel to
magnetic field-lines in the stationary frame. 

\begin{figure}
\epsfysize=4in
\centerline{\epsffile{Chapter05/spiral1.eps}}
\caption{\em The interplanetary magnetic field.}\label{f21}
\end{figure}

The interplanetary magnetic field at 1 AU is observed to lie in the
ecliptic plane, and is directed at an angle of approximately $45^\circ$ from
the radial direction to the Sun. This is in basic agreement with the spiral
configuration predicted above.

The analysis presented above is premised on the assumption that the
interplanetary magnetic field is too weak to affect the coronal outflow, and
is, therefore, passively convected  by the solar wind. In fact, this is only
likely to be the case if the interplanetary 
magnetic energy density, $B^2/2\,\mu_0$, is
much less that the kinetic energy density, $\rho \,u^2/2$, of the solar wind.
Rearrangement yields the condition
\begin{equation}
u > V_A,
\end{equation}
where $V_A$ is the Alfv\'{e}n speed. It turns out that $u\sim 10\,V_A$ at
1 AU. On the other hand, $u\ll V_A$ close to the base of the
corona. In fact, the solar wind becomes super-Alfv\'{e}nic at a radius,
denoted $r_A$, which is typically $50\,R_\odot$, or $1/4$ of an
astronomical unit. We conclude that the previous analysis is only
valid well outside the Alfv\'{e}n radius: {\em i.e.}, in the region $r\gg r_A$.

Well inside the Alfv\'{e}n radius ({\em i.e.}, in the region $r\ll r_A$),
the solar wind is too weak to modify the structure of the solar magnetic field.
In fact, in this region we expect the solar  magnetic field to
force  the solar wind to
{\em co-rotate}\/ with the Sun. Note that flux-freezing is a two-way-street:
if the energy density of the flow greatly exceeds that of the magnetic field
then the magnetic field is passively convected by the flow, but if
the energy density of the magnetic field greatly exceeds that of the flow
then the flow is forced to conform to the magnetic field.

The above discussion leads us to the following rather crude picture of the
interaction of the solar wind and the interplanetary magnetic field.
We expect the interplanetary 
magnetic field to be simply the undistorted continuation of
the Sun's magnetic field for $r<r_A$. On the other hand, we
expect the interplanetary field to be dragged out into a spiral
pattern for $r>r_A$. Furthermore, we expect the Sun's magnetic field
to impart   a non-zero  azimuthal velocity $u_\phi(r)$ to the solar
wind. In the ecliptic plane, we expect
\begin{equation}
u_\phi = {\Omega}\,r
\end{equation}
for $r<r_A$, and
\begin{equation}
u_\phi = {\Omega}\,r_A\left(\frac{r_A}{r}\right)
\end{equation}
for $r>r_A$.
This corresponds to co-rotation with the Sun inside the Alfv\'{e}n radius,
and outflow at constant angular velocity outside the Alfv\'{e}n radius.
We, therefore, expect the solar wind at 1 AU to possess a small azimuthal
velocity component. This is indeed the case. In fact, the direction
of the solar wind at 1~AU
deviates from purely radial outflow by about $1.5^\circ$. 

\section{Mass and Angular Momentum Loss}
Since the Sun is the best observed of any star, it is interesting to
ask what impact the solar wind has as far as solar, and stellar, evolution
are concerned. The most obvious question is whether the mass loss due
to the wind is significant, or not. Using typical measured values ({\em i.e.},
a typical solar wind velocity and particle density at 1 AU of 
$500\,{\rm km}\,{\rm s}^{-1}$ and $7\times 10^6\,\,{\rm m}^{-3}$, respectively),
the Sun is apparently losing mass at a rate of $3\times 10^{-14}\,M_\odot$
per year, implying a time-scale for significant mass loss of $3\times 10^{13}$
years, or some $6,000$ times longer than the estimated $5\times 10^9$ year
age of the Sun. Clearly, the mass carried off by the solar wind has a negligible
effect on the Sun's evolution. Note, however, that many other stars in the Galaxy
exhibit significant mass loss via stellar winds. This is particularly
the case for late-type stars.

Let us now consider the angular momentum carried off by the solar wind.
Angular momentum loss is a crucially important topic in astrophysics, since
only by losing angular momentum can large, diffuse objects, such as
interstellar gas clouds, collapse under the influence of gravity to produce
small, compact objects, such as stars and proto-stars. Magnetic fields
generally play a crucial role in angular momentum loss. This is certainly
the case for the solar wind, where the solar magnetic field enforces
co-rotation with the Sun out to the Alfv\'{e}n radius, $r_A$. Thus, the
angular momentum carried away by a particle of mass $m$ is ${\Omega}\,r_A^{~2}\,
m$, rather than ${\Omega}\,R_\odot^{~2}\,m$. The angular momentum
loss time-scale is, therefore, shorter than the mass loss time-scale by a factor
$(R_\odot/r_A)^2\simeq 1/2500$, making the angular momentum loss time-scale
comparable to the solar lifetime. It is clear that magnetized stellar
winds represent a very important vehicle for angular momentum loss in the
Universe. Let us investigate angular momentum loss via
stellar winds in more detail.

Under the assumption of spherical symmetry and steady flow, the azimuthal
momentum evolution equation for the solar wind, taking into account the
influence of the interplanetary magnetic field, is written
\begin{equation}
\rho\,\frac{u_r}{r}\frac{d(r\,u_\phi)}{dr} = ({\bf j}\times{\bf B})_\phi
= \frac{B_r}{\mu_0\,r}\frac{d(rB_\phi)}{dr}.
\end{equation}
The constancy of the mass flux [see Eq.~(\ref{e5.49})] and the $1/r^2$ dependence
of $B_r$ [see Eq.~(\ref{e5.66a})] permit the immediate integration of the
above equation to give
\begin{equation}\label{e5.72}
r\,u_\phi -\frac{r\,B_r\,B_\phi}{\mu_0\,\rho\,u_r} = L,
\end{equation}
where $L$ is the angular momentum per unit mass carried off by the solar wind.
In the presence of an azimuthal wind velocity,  the magnetic field and
velocity components are related by an expression similar to Eq.~(\ref{e5.64}):
\begin{equation}\label{e5.73}
\frac{B_r}{B_\phi} = \frac{u_r}{u_\phi - {\Omega}\,r\,\sin\theta}.
\end{equation}
The fundamental physics assumption underlying the above expression is
the absence of an electric field in the frame of reference co-rotating
with the Sun. Using Eq.~(\ref{e5.73}) to eliminate $B_\phi$ from Eq.~(\ref{e5.72}), we obtain
(in the ecliptic plane, where $\sin\theta=1$)
\begin{equation}\label{e5.74}
r\,u_\phi = \frac{L\,M_A^{~2} - {\Omega}\,r^2}{M_A^{~2} - 1},
\end{equation}
where
\begin{equation}
M_A = \sqrt{\frac{u_r^{~2}}{B_r^{~2}/\mu_0\,\rho}}
\end{equation}
is the {\em radial Alfv\'{e}n Mach number}. The radial Alfv\'{e}n Mach number
is small near the base of the corona, and about 10 at 1~AU: it passes through
unity at the Alfv\'{e}n radius, $r_A$, which is about $0.25$\,AU from the Sun.
The zero denominator on the right-hand side
of Eq.~(\ref{e5.74}) at $r=r_A$ implies that $u_\phi$ is
finite and continuous only if the numerator is also zero at the Alfv\'{e}n radius.
This condition then determines the angular momentum content of the outflow
via
\begin{equation}\label{e5.76}
L = {\Omega}\,r_A^{~2}.
\end{equation}
Note that the angular momentum carried off by the solar wind is indeed
equivalent to that which would be carried off were coronal plasma to
co-rotate with the Sun out to the Alfv\'{e}n radius, and subsequently outflow
at constant angular velocity. Of course, the solar wind does not actually
rotate rigidly with the Sun in the region $r<r_A$: much of the angular
momentum in this region is carried in the form of electromagnetic stresses. 

It is easily demonstrated that the quantity $M_A^{~2}/u_r\,r^2$ is a constant,
and can, therefore, be evaluated at $r=r_A$ to give
\begin{equation}\label{e5.77}
M_A^{~2} = \frac{u_r\,r^2}{u_{rA}\,r_A^{~2}},
\end{equation}
where $u_{rA} \equiv u_r(r_A)$. Equations~(\ref{e5.74}), (\ref{e5.76}), and (\ref{e5.77}) can
be combined to give 
\begin{equation}
u_\phi = \frac{{\Omega}\,r}{u_{rA}} \frac{u_{rA} - u_r}{1-M_A^{~2}}.
\end{equation}
In the limit $r\rightarrow\infty$, we have $M_A\gg 1$, so the above
expression yields
\begin{equation}\label{e5.79}
u_\phi \rightarrow {\Omega}\,r_A\left(\frac{r_A}{r}\right)\left(1-\frac{u_{rA}}
{u_r}\right)
\end{equation}
at large distances from the Sun. Recall, from Sect.~\ref{s5.7}, that if the coronal
plasma were to simply co-rotate with the Sun out to $r=r_A$, and experience
no torque beyond this radius, then we would expect
\begin{equation}
u_\phi \rightarrow {\Omega}\,r_A\left(\frac{r_A}{r}\right)
\end{equation}
at large distances from the Sun.
The difference between the above two expressions is the factor $1-u_{rA}/u_r$,
which is a correction for the angular momentum retained by the magnetic
field at large $r$. 

The analysis presented above was first incorporated into a quantitative
coronal expansion model by Weber and Davis.\footnote{E.J.~Weber, and L.~Davis
Jr., Astrophys.\ J.\ {\bf 148}, 217 (1967).} The model of Weber and Davis is
very complicated. For instance, the solar wind is required to flow smoothly
through no less than {\em three}\/ critical points. These are associated
with the sound speed (as in Parker's original model), the radial Alfv\'{e}n
speed, $B_r/\sqrt{\mu_0\,\rho}$, (as described above), and the total
Alfv\'{e}n speed, $B/\sqrt{\mu_0\,\rho}$. 
Nevertheless, the simplified analysis
outlined above captures most of the essential features of the outflow. 
For instance, Fig.~\ref{f22} shows a comparison between the large-$r$ asymptotic
form for the azimuthal flow velocity predicted above [see Eq.~(\ref{e5.79})] and
that calculated by Weber and Davis, showing the close agreement between 
the two.

\begin{figure}
\epsfysize=3in
\centerline{\epsffile{Chapter05/asym1.eps}}
\caption{\em Comparison of asymptotic form for azimuthal flow velocity
of solar wind with Weber-Davis solution.}\label{f22}
\end{figure}

\section{MHD Dynamo Theory}
Many stars, planets, and galaxies possess magnetic fields whose origins
are not easily explained. Even the ``solid'' planets could not possibly
be sufficiently ferromagnetic to account for their magnetism, since the bulk
of their interiors are above the Curie temperature at which permanent magnetism 
disappears. It goes without saying that stars and galaxies cannot
be ferromagnetic at all. Magnetic fields cannot be dismissed as transient
phenomena which just happen to be present today. For instance, 
{\em paleomagnetism}, the study of magnetic fields ``fossilized'' in rocks
at the time of their formation in the remote geological past, shows
that the Earth's magnetic field has existed at much its present
strength for at least the past $3\times 10^9$ years. The problem is that,
in the absence of an internal  source of electric currents, magnetic fields contained in a
conducting body {\em decay}\/ ohmically on a time-scale
\begin{equation}
\tau_{\rm ohm} = \mu_0\,\sigma\,L^2,
\end{equation}
where $\sigma$ is the typical electrical conductivity, and $L$ is the
typical length-scale of the body, and  this decay time-scale is generally
very {\em small}\/  compared to the inferred  lifetimes of astronomical magnetic fields. For instance,
the Earth contains a highly conducting region, namely, its molten core, of
radius $L\sim 3.5\times 10^6$\,m, and conductivity $\sigma\sim 4\times 10^5\,
{\rm S}\,{\rm m}^{-1}$. This yields an ohmic decay time for the terrestrial
magnetic field of only $\tau_{\rm ohm}\sim 2\times 10^5$ years, which is
obviously far shorter than the inferred lifetime of this field.
Clearly, some process inside the Earth must be actively maintaining the
terrestrial magnetic field. Such a process is conventionally termed a
{\em dynamo}. Similar considerations lead us to postulate the existence
of 
dynamos acting inside stars and galaxies, in order to account for the persistence
of stellar and galactic magnetic fields over cosmological time-scales. 

The basic premise of dynamo theory is that all astrophysical bodies which
contain anomalously long-lived magnetic fields also contain  highly conducting
fluids ({\em e.g.}, the Earth's molten core, the ionized gas which makes
up the Sun), and it is the electric currents
associated with the motions of these fluids which maintain the
observed magnetic fields. At first sight, this proposal, first
made by Larmor in 1919,\footnote{J.~Larmor, Brit.\ Assoc.\ Reports, {\bf 159}
(1919).} sounds suspiciously like
pulling yourself up by your own shoelaces. However, there is really
no conflict with the demands of energy conservation. The 
magnetic  energy irreversibly
lost via ohmic heating is replenished at the rate
(per unit volume) ${\bf V}\cdot
({\bf j}\times{\bf B})$: {\em i.e.}, by the rate of work done against the
Lorentz force. The flow  field, ${\bf V}$, is assumed to be driven via thermal
convention. If the flow is sufficiently vigorous then it is, at least,
plausible that the energy input to the magnetic field can overcome the losses
due to ohmic heating, thus permitting the field to persist over time-scales
far longer than the characteristic ohmic decay time.

Dynamo theory  involves two vector fields, 
${\bf V}$ and ${\bf B}$, coupled by a rather complicated force: {\em i.e.},
the Lorentz force. 
It is not surprising, therefore, that dynamo theory tends to be extremely
complicated, and is, at present,  far from completely understood. 
Fig.~\ref{f23} shows paleomagnetic data illustrating  the variation of the polarity of the Earth's
magnetic field over the last few  million years, as
deduced from marine sediment cores. It can be seen that the Earth's magnetic
field is quite variable, and actually reversed polarity about $700,000$ years
ago. In fact, more extensive data shows that the Earth's magnetic field
reverses polarity about once every ohmic decay time-scale ({\em i.e.}, a few
times every million years). The Sun's magnetic field exhibits similar
behaviour, reversing polarity about once every 11 years. It is clear from
examining this type of data that dynamo magnetic fields (and velocity fields) are 
essentially {\em chaotic}\/
in nature, 
exhibiting strong random variability superimposed on more regular quasi-periodic
oscillations.

\begin{figure}
\epsfysize=5in
\centerline{\epsffile{Chapter05/timescale.eps}}
\caption{\em Polarity of the Earth's magnetic field as a function of time, as deduced from
marine sediment cores.}\label{f23}
\end{figure}

Obviously, we are not going to attempt to tackle full-blown dynamo
theory in this course: that would be far too difficult. Instead, we shall examine a
far simpler theory, known as {\em kinematic dynamo theory}, in which
the velocity field, ${\bf V}$, is {\em prescribed}.
In order for this approach to be self-consistent, the magnetic field must be
assumed to be sufficiently small that it does not affect the velocity field.
Let us start from the MHD Ohm's law, modified by resistivity:
\begin{equation}
{\bf E} + {\bf V} \times{\bf B} = \eta\,{\bf j}.
\end{equation}
Here, the resistivity $\eta$ is assumed to be a constant, for the sake
of simplicity. Taking the curl of the above equation, and making use
of Maxwell's equations, we obtain
\begin{equation}
\frac{\partial {\bf B}}{\partial t} - \nabla\times({\bf V}\times{\bf B})
=  \frac{\eta}{\mu_0}\nabla^2{\bf B}.
\end{equation}
If the velocity field, ${\bf V}$, is prescribed, and unaffected by the
presence of the magnetic field, then the above equation is
essentially a {\em linear eigenvalue}\/ equation for the magnetic field, ${\bf B}$.
The question we wish to address is as follows: for what sort
of velocity fields, if any, does the above equation possess solutions
where the magnetic field grows exponentially? In trying to answer this question,
we hope to learn what type of motion of an MHD fluid is capable of
self-generating 
a magnetic field. 

\section{Homopolar Generators}
Some of the peculiarities of dynamo theory are well illustrated
by the prototype example of self-excited dynamo action, which is the
{\em homopolar disk dynamo}. As illustrated in Fig.~\ref{f24}, this device
consists of a conducting disk which rotates at angular
frequency ${\Omega}$  about its axis under the
action of an applied torque. A wire, twisted about the axis in the
manner shown, makes sliding contact with the disc at $A$, and with
the axis at $B$, and carries a current $I(t)$. The magnetic field
${\bf B}$ associated with this current has a flux ${\Phi} = M\,I$
across the disc, where $M$ is the mutual inductance between the wire and the
rim of the disc. The rotation of the disc in the
presence of this flux generates a radial electromotive
force
\begin{equation}
\frac{{\Omega}}{2\pi}\,{\Phi} = \frac{{\Omega}}{2\pi}\,M\,I,
\end{equation}
since a radius of the disc cuts the magnetic flux ${\Phi}$ once
every $2\pi/{\Omega}$ seconds. According to this simplistic
description, the equation for $I$ is written
\begin{equation}\label{e5.85}
L\,\frac{dI}{dt} + R\,I = \frac{M}{2\pi}\,{\Omega}\,I,
\end{equation}
where $R$ is the total resistance of the circuit, and $L$ is its
self-inductance. 

\begin{figure}
\epsfysize=2.5in
\centerline{\epsffile{Chapter05/homopolar1.eps}}
\caption{\em The homopolar generator.}\label{f24}
\end{figure}

Suppose that the angular velocity ${\Omega}$ is maintained by suitable adjustment
of the driving torque. It follows that Eq.~(\ref{e5.85}) possesses an
exponential solution $I(t) = I(0)\,\exp(\gamma\,t)$, where
\begin{equation}\label{e5.86}
\gamma = L^{-1}\left[\frac{M}{2\pi}\,{\Omega} - R\right].
\end{equation}
Clearly, we have exponential growth of $I(t)$, and, hence, of the magnetic
field to which it gives rise ({\em i.e.}, we have dynamo action),
provided that
\begin{equation}
{\Omega} > \frac{2\pi\,R}{M}:
\end{equation}
{\em i.e.}, provided that the disk rotates rapidly enough. Note that
the homopolar generator depends for its success on its built-in
axial  {\em asymmetry}. If the disk rotates in the opposite
direction to that shown in Fig.~\ref{f24} then ${\Omega}<0$, and the
electromotive force generated by the rotation of the disk always acts
to reduce $I$. In this case, dynamo action is impossible ({\em i.e.}, $\gamma$
is always negative). This is a troubling observation,
since most astrophysical objects, such as stars and planets, possess very
good axial symmetry. We conclude that if such bodies are to act
as dynamos then the asymmetry of their internal motions must somehow
compensate for their lack of built-in asymmetry. It is far from obvious
how this is going to happen.

Incidentally, although the above analysis of a homopolar generator 
(which is the standard analysis found in most textbooks) is
very appealing in its simplicity, it cannot be entirely correct. 
Consider the limiting situation of a {\em perfectly
conducting}\/ disk and wire, in which $R=0$. On the one hand,
Eq.~(\ref{e5.86}) yields $\gamma= M\,{\Omega}/2\pi\,L$, so that
we still have dynamo action. But, on the other hand, the rim of the disk
is a closed circuit embedded in a perfectly conducting medium, so the
flux freezing constraint requires that the flux, ${\Phi}$,
through this circuit must remain a constant. There is
an obvious contradiction. 
The problem is that we have neglected the currents
that flow azimuthally in the disc: {\em i.e.}, the very currents
which control the diffusion of magnetic flux across the rim of
the disk. These currents become particularly important
in the limit $R\rightarrow\infty$. 

The above paradox can be resolved by supposing that the azimuthal current
$J(t)$ is constrained to flow around the rim of the disk ({\em e.g.},
by a suitable distribution of radial insulating strips). In this
case, the fluxes through the $I$ and $J$ circuits are
\begin{eqnarray}
{\Phi}_1 &=& L\,I + M\, J,\\[0.5ex]
{\Phi}_2 &=& M\,I + L'\,J,
\end{eqnarray}
and the equations governing the current flow are

\begin{eqnarray}
\frac{d{\Phi}_1}{dt} &=& \frac{{\Omega}}{2\pi}\,{\Phi}_2 - R\,I,\\[0.5ex]
\frac{d{\Phi}_2}{dt} &=& - R'\,J,
\end{eqnarray}
where $R'$, and $L'$ refer to the $J$ circuit. Let us search
for exponential solutions, $(I,J)\propto \exp(\gamma\,t)$, of the
above system of equations. It is easily
demonstrated that

\begin{equation}
\gamma = \frac{ -[L\,R'+L'\,R] \pm \sqrt{[L\,R'+L'\,R]^2 + 4\,R'\,[L\,L'-M^2]\,
[M{\Omega}/2\pi -R]}}{2\,[L\,L'-M^2]}.
\end{equation}
Recall the standard result in electromagnetic theory that $L\,L'>M^2$ for two
non-coincident circuits. It is clear, from the above expression, that the
condition for dynamo action ({\em i.e.}, $\gamma>0$) is
\begin{equation}
{\Omega} > \frac{ 2\pi\,R}{M},
\end{equation}
as before. Note, however, that $\gamma\rightarrow 0$ as $R'\rightarrow 0$.
In other words, if the rotating disk is a perfect conductor then dynamo
action is impossible. The above system of equations can transformed
into the well-known Lorenz system, which exhibits {\em chaotic}\/ behaviour
in certain parameter regimes.\footnote{E.~Knobloch, Phys.\ Lett.\ {\bf 82A},
439 (1981).} It is noteworthy that this simplest prototype
dynamo system already contains the seeds of chaos (provided that
the formulation is self-consistent). 

It is clear from the above discussion that, whilst dynamo action requires the
resistance of the circuit, $R$, to be low, we lose dynamo action 
altogether if we go to the
perfectly conducting limit, $R\rightarrow 0$, because magnetic fields are unable to diffuse
into the region in which magnetic induction is operating. Thus, an efficient
dynamo requires a conductivity that is large, but not too large. 

\section{Slow  and Fast Dynamos}
Let us search for solutions of the MHD kinematic dynamo equation,
\begin{equation}\label{e5.92}
\frac{\partial {\bf B}}{\partial t} = \nabla\times({\bf V}\times{\bf B})
+ \frac{\eta}{\mu_0}\nabla^2{\bf B},
\end{equation}
for a prescribed {\em steady-state} velocity field, ${\bf V}({\bf r})$, 
subject to certain practical constraints. Firstly, we require a
{\em self-contained} solution: {\em i.e.}, a solution in which the
magnetic field is maintained by the motion of the MHD fluid, rather than 
by currents at infinity. This suggests that $V, B\rightarrow 0$ as $r\rightarrow
\infty$. Secondly, we require an exponentially growing solution:
{\em i.e.}, a solution for which ${\bf B}\propto \exp(\gamma\,t)$, where
$\gamma>0$. 

In most  MHD fluids occurring in astrophysics, the resistivity, $\eta$, is extremely
small. Let us consider the perfectly conducting limit, $\eta\rightarrow 0$. 
In this limit, Vainshtein and Zel'dovich, in 1978, introduced an important 
distinction
between two fundamentally different classes of dynamo 
solutions.\footnote{S.~Vainshtein, and Ya.~B.~Zel'dovich, Sov.\ Phys.\
 Usp.\ {\bf 15}, 159 (1978).}
Suppose that we solve the eigenvalue equation (\ref{e5.92}) to obtain the
growth-rate, $\gamma$, of the magnetic field in the limit $\eta\rightarrow 0$.
We expect that
\begin{equation}
\lim_{\eta\rightarrow 0}\gamma \propto \eta^\alpha,
\end{equation}
where $0\leq \alpha\leq 1$. There are two possibilities. Either $\alpha>0$,
in which case the growth-rate depends on the resistivity, or $\alpha=0$,
in which case the growth-rate is independent of the resistivity. The
former case is termed a {\em slow dynamo}, whereas the latter case is termed
a {\em fast dynamo}. By definition, slow dynamos are unable to operate
in the perfectly conducting limit, since $\gamma\rightarrow 0$ as 
$\eta\rightarrow 0$.
On the other hand, fast dynamos can, in principle, operate when $\eta=0$. 

It is clear, from the above discussion, that a homopolar disk generator is
an example of a slow dynamo. In fact, it is easily seen that any
dynamo which depends on the motion of a {\em rigid} conductor for its
operation is bound to be a slow dynamo: in the perfectly conducting
limit, the magnetic flux linking the conductor could never change, so there
would be no magnetic induction. So, why do we believe that fast dynamo
action is even a possibility for an MHD fluid? The answer is, of course, that
an MHD fluid is a {\em non-rigid} body, and, thus, its motion possesses
degrees of freedom not accessible to rigid conductors. 

We know that in the perfectly conducting limit ($\eta\rightarrow 0$) magnetic
field-lines are frozen into an MHD fluid. If the motion is
incompressible ({\em i.e.}, $\nabla\!\cdot\!{\bf V} = 0$) then the stretching
of field-lines implies a proportionate intensification of the field-strength.
The simplest heuristic fast dynamo, first described by Vainshtein and Zel'dovich, 
 is based on this effect. As illustrated in Fig.~\ref{f25}, a magnetic
flux-tube can be {\em doubled} in intensity by taking it around a
stretch-twist-fold cycle. The doubling time for this
process clearly does not depend on the resistivity: in this sense, the
dynamo is a fast dynamo. However, under repeated application of this
cycle the magnetic field develops increasingly fine-scale structure.
In fact, in the limit $\eta\rightarrow 0$ both the ${\bf V}$ and
${\bf B}$ fields eventually become chaotic and non-differentiable.
A little resistivity is always required to smooth out the fields on
small length-scales: even in this case the fields remain {\em chaotic}.

\begin{figure}
\epsfysize=1.75in
\centerline{\epsffile{Chapter05/stretch1.eps}}
\caption{\em The stretch-twist-fold cycle of a fast dynamo.}\label{f25}
\end{figure}

At present, the physical existence of fast dynamos  has not been
conclusively  established, since most of the literature on this
subject is based on mathematical paradigms rather than actual solutions
of the dynamo equation. It should be noted, however, that the
need for fast dynamo solutions is fairly acute, especially in stellar
dynamo theory. For instance, consider the Sun. The ohmic decay time for the
Sun is about $10^{12}$ years, whereas the reversal time for the solar magnetic
field is only 11 years. It is obviously a little difficult to believe that resistivity
is playing any significant role in the solar dynamo.  

In the following, we shall restrict our analysis to slow dynamos, which
undoubtably exist in nature, and which are characterized by {\em non-chaotic}
${\bf V}$ and ${\bf B}$ fields.

\section{Cowling Anti-Dynamo Theorem}
One of the most important results in slow, kinematic dynamo theory
is credited to Cowling.\footnote{T.G.~Cowling, Mon.\ Not.\ Roy.\ Astr.\ Soc.\
{\bf 94}, 39 (1934); T.G.~Cowling, Quart.\ J.\ Mech.\ Appl.\ Math.\
{\bf 10}, 129 (1957).} The so-called {\em Cowling anti-dynamo theorem}
states that:
\begin{quote}
{\sf An axisymmetric magnetic field cannot be maintained via dynamo action.}
\end{quote}
Let us attempt to prove this proposition.

We adopt standard cylindrical polar coordinates: $(\varpi,\theta,z)$. The
system is assumed to possess axial symmetry, so that $\partial/\partial\theta
\equiv 0$. For the sake of simplicity, the plasma flow is assumed to be 
incompressible, which implies that $\nabla\!\cdot\!{\bf V}=0$. 

It is convenient to split the magnetic and velocity fields into {\em poloidal}
and {\em toroidal} components:
\begin{eqnarray}
{\bf B} &=& {\bf B}_p + {\bf B}_t,\\[0.5ex]
{\bf V} &=& {\bf V}_p + {\bf V}_t.
\end{eqnarray}
Note that a poloidal vector only possesses non-zero $\varpi$- and $z$-components,
whereas a toroidal vector only possesses a non-zero $\theta$-component.


The poloidal components of the magnetic and velocity fields are
written:
\begin{eqnarray}
{\bf B}_p = \nabla\times\left(\frac{\psi}{\varpi}\,\,\hat{\btheta}\right)
\equiv \frac{\nabla\psi\times\hat{\btheta}}{\varpi},\\[0.5ex]
{\bf V}_p = \nabla\times\left(\frac{\phi}{\varpi}\,\,\hat{\btheta}\right)
\equiv \frac{\nabla\phi\times\hat{\btheta}}{\varpi},
\end{eqnarray}
where $\psi=\psi(\varpi,z,t)$ and $\phi=\phi(\varpi,z,t)$. The
toroidal components are given by
\begin{eqnarray}
{\bf B}_t &=& B_t(\varpi,z,t)\,\hat{\btheta},\\[0.5ex]
{\bf V}_t &=& V_t(\varpi,z,t)\,\hat{\btheta}.
\end{eqnarray}
Note that by writing the ${\bf B}$ and ${\bf V}$ fields in the above form
we ensure that the constraints $\nabla\!\cdot\!{\bf B} =0$ and
$\nabla\!\cdot\!{\bf V} =0$ are {\em automatically} satisfied. Note, further,
that since ${\bf B}\!\cdot\!\nabla\psi=0$ and ${\bf V}\!\cdot\!\nabla\phi=0$,
we can regard $\psi$ and $\phi$ as {\em stream-functions} for the magnetic and
velocity fields, respectively.

The condition for the magnetic field to be maintained by dynamo currents,
rather than by currents at infinity, is
\begin{equation}\label{e5.97}
\psi\rightarrow \frac{1}{r}\mbox{\hspace{1cm}as $r\rightarrow\infty$},
\end{equation}
where $r=\sqrt{\varpi^2+z^2}$. We also require  the flow stream-function,
$\phi$, to remain bounded as $r\rightarrow\infty$.


Consider the MHD Ohm's law for a resistive plasma:
\begin{equation}\label{e5.98}
{\bf E} + {\bf V}\times{\bf B} =\eta\,{\bf j}.
\end{equation}
Taking the toroidal component of this equation, we obtain
\begin{equation}\label{e5.99}
E_t + ({\bf V}_p\times{\bf B}_p)\!\cdot \hat{\btheta}=\eta\, j_t.
\end{equation}
It is easily demonstrated that
\begin{equation}
E_t =-\frac{1}{\varpi}\frac{\partial\psi}{\partial t}.
\end{equation}
Furthermore,
\begin{equation}
({\bf V}_p\times{\bf B}_p)\!\cdot\hat{\btheta} =\frac{(\nabla\phi\times
\nabla\psi)\!\cdot \hat{\btheta}}{\varpi^2}=\frac{1}{\varpi^2}
\left(\frac{\partial\psi}{\partial\varpi}\frac{\partial\phi}{\partial z}
-\frac{\partial\phi}{\partial\varpi}\frac{\partial\psi}{\partial z}\right),
\end{equation}
and
\begin{equation}
\mu_0\,j_t = \nabla\times {\bf B}_p\!\cdot \hat{\btheta} =
-\left[\nabla^2\!\left(\frac{\psi}{\varpi}\right) -\frac{\psi}{\varpi^3}\right]
= -\frac{1}{\varpi}\left(\frac{\partial^2\psi}{\partial\varpi^2}-\frac{1}{\varpi}
\frac{\partial\psi}{\partial\varpi}+\frac{\partial^2\psi}{\partial z^2}\right).
\end{equation}
Thus,
Eq.~(\ref{e5.99}) reduces to
\begin{equation}
\frac{\partial\psi}{\partial t} - \frac{1}{\varpi}\left(\frac{\partial\psi}{\partial\varpi}\frac{\partial\phi}{\partial z}
-\frac{\partial\phi}{\partial\varpi}\frac{\partial\psi}{\partial z}\right)
= \frac{\eta}{\mu_0}\left(\frac{\partial^2\psi}{\partial\varpi^2}
-\frac{1}{\varpi}\frac{\partial\psi}{\partial\varpi}
+\frac{\partial^2\psi}{\partial z^2}\right).
\end{equation}

Multiplying the above equation by $\psi$ and integrating over all space,
we obtain
\begin{eqnarray}\label{e5.104}
\frac{1}{2}\frac{d}{dt}\!\int\psi^2\,dV& -&\int\!\!\int 2\pi\,\psi\left(\frac{\partial\psi}{\partial\varpi}\frac{\partial\phi}{\partial z}
-\frac{\partial\phi}{\partial\varpi}\frac{\partial\psi}{\partial z}\right) d\varpi
\,dz\\[0.5ex]
 &=&\frac{\eta}{\mu_0}\int\!\!\int2\pi\,\varpi\,\psi\left(\frac{\partial^2\psi}{\partial\varpi^2}
-\frac{1}{\varpi}\frac{\partial\psi}{\partial\varpi}
+\frac{\partial^2\psi}{\partial z^2}\right)d\varpi\,dz.\nonumber
\end{eqnarray}
The second term on the left-hand side of the above expression
can be integrated by parts to give
\begin{equation}
-\int\!\!\int 2\pi\left[-\phi\,\frac{\partial}{\partial z}\!
\left(\psi\,\frac{\partial\psi}{\partial\varpi}\right)
+\phi\,\frac{\partial}{\partial \varpi}\!\left(\psi\,\frac{\partial\psi}{\partial z}
\right)\right] d\varpi\,dz=0,
\end{equation}
where surface terms have been neglected, in  accordance with Eq.~(\ref{e5.97}).
Likewise, the term on the right-hand side of Eq.~(\ref{e5.104}) can be integrated by
parts to give
\begin{eqnarray}
\frac{\eta}{\mu_0}\int\!\!\int 2\pi\left[-\frac{\partial(\varpi\,\psi)}{\partial\varpi}
\,\frac{\partial\psi}{\partial\varpi} - \varpi\left(\frac{\partial\psi}
{\partial z}\right)^2\right]\,d\varpi\,d z=&&\nonumber\\[0.5ex]
\mbox{\hspace{2cm}}-\frac{\eta}{\mu_0}\int\!\!\int 2\pi\,
\varpi\left[\left(\frac{\partial\psi}{\partial\varpi}\right)^2
+\left(\frac{\partial\psi}{\partial z}\right)^2\right]\,d\varpi\,dz.&&
\end{eqnarray}
Thus, Eq.~(\ref{e5.104}) reduces to
\begin{equation}
\frac{d}{dt}\!\int\psi^2\,dV = -2\,\frac{\eta}{\mu_0}\int |\nabla\psi|^2\,dV.
\end{equation}
It is clear from the above expression that the poloidal stream-function, $\psi$,
and, hence, the poloidal magnetic field, ${\bf B}_p$, decays to zero
under the influence of resistivity. We conclude that the poloidal magnetic
field {\em cannot} be maintained via dynamo action.

Of course, we have not ruled out the possibility that the toroidal magnetic
field can be maintained via dynamo action. In the absence of a poloidal
field, the curl of the poloidal component of Eq.~(\ref{e5.98}) yields
\begin{equation}
-\frac{\partial {\bf B}_t}{\partial t} +\nabla\times({\bf V}_p\times
{\bf B}_t) = \eta\,\nabla\times {\bf j}_p,
\end{equation}
which reduces to
\begin{equation}\label{e5.109}
-\frac{\partial B_t}{\partial t} +
\nabla\times({\bf V}_p\times {\bf B}_t) 
\cdot\hat{\btheta}=-
\frac{\eta}{\mu_0}\,\nabla^2(B_t\,\hat{\btheta})\cdot\hat{\btheta}.
\end{equation}
Now
\begin{equation}
\nabla^2(B_t\,\hat{\btheta})\cdot\hat{\btheta}
=\frac{\partial^2 B_t}{\partial\varpi^2} +\frac{1}{\varpi}
\frac{\partial B_t}{\partial\varpi} + \frac{\partial^2 B_t}{\partial z^2}
-\frac{B_t}{\varpi^2},
\end{equation}
and
\begin{equation}
\nabla\times({\bf V}_p\times {\bf B}_t)
\cdot\hat{\btheta}=
\frac{\partial}{\partial\varpi}\!\left(\frac{B_t}{\varpi}\right)
\frac{\partial\phi}{\partial z} -\frac{\partial}{\partial z}\!
\left(\frac{B_t}{\varpi}\right) \frac{\partial\phi}{\partial\varpi}.
\end{equation}
Thus, Eq.~(\ref{e5.109}) yields
\begin{equation}\label{e5.112}
\frac{\partial\chi}{\partial t} -\frac{1}{\varpi}\left(\frac{\partial\chi}{\partial\varpi}
\frac{\partial\phi}{\partial z} -\frac{\partial
\phi}{\partial \varpi}\frac{\partial\chi}{\partial z}\right)=\frac{\eta}{\mu_0}\left(
\frac{\partial^2\chi}{\partial\varpi^2} +\frac{3}{\varpi}\frac{\partial\chi}
{\partial\varpi} +\frac{\partial^2\chi}{\partial z^2}\right),
\end{equation}
where 
\begin{equation}
B_t = \varpi\,\chi.
\end{equation}

Multiply  Eq.~(\ref{e5.112}) by $\chi$, integrating over all space, and
then integrating by parts, we obtain
\begin{equation}
\frac{d}{dt}\!\int\chi^2\,dV = -2\,\frac{\eta}{\mu_0}\int|\nabla\chi|^2\,dV.
\end{equation}
It is clear from this formula that $\chi$, and, hence, the toroidal magnetic
field, ${\bf B}_t$, decay to zero under the influence of resistivity.
We conclude that no axisymmetric magnetic field, either poloidal
or toroidal, can be maintained by
dynamo action, which proves Cowling's theorem.

Cowling's theorem is the earliest and most significant of a number of
{\em anti-dynamo theorems} which severely restrict the types of magnetic
fields which can be maintained via dynamo action. For instance, it is
possible to prove that a two-dimensional magnetic field cannot be maintained
by dynamo action. Here, ``two-dimensional'' implies that in some
Cartesian coordinate system, $(x,y,z)$, the magnetic field is independent of
$z$. The suite of anti-dynamo theorems can be summed up
by saying that successful dynamos possess a rather low degree of symmetry.

\section{Ponomarenko Dynamos}
The simplest known kinematic dynamo is that
of Ponomarenko.\footnote{Yu.~B.~Ponomarenko, J.~Appl.\ Mech.\
Tech.\ Phys.\ {\bf 14}, 775 (1973).} Consider a conducting fluid
of resistivity $\eta$ which fills all space. The motion of the fluid
is confined to a cylinder of radius $a$. Adopting cylindrical polar coordinates
$(r,\theta,z)$ aligned with this cylinder, the flow field is written
\begin{equation}
{\bf V} =\left\{
\begin{array}{lll}
(0,r\,{\Omega}, U)&\mbox{\hspace{1cm}}& \mbox{for $r\leq a$}\\[0.5ex]
{\bf 0} && \mbox{for $r>a$}
\end{array}
\right.,
\end{equation}
where ${\Omega}$ and $U$ are constants. Note that the flow is {\em 
incompressible}:
{\em i.e.}, $\nabla\cdot{\bf V} =0$. 

The dynamo equation can be written
\begin{equation}\label{e5.116}
\frac{\partial{\bf B}}{\partial t} = ({\bf B}\cdot\nabla){\bf V}
-({\bf V}\cdot\nabla){\bf B} + \frac{\eta}{\mu_0}\,\nabla^2{\bf B}.
\end{equation}
Let us search for solutions to this equation of the form
\begin{equation}
{\bf B}(r,\theta,z,t) = {\bf B}(r)\,\exp[\,{\rm i}\,(m\,\theta -k\, z)+\gamma\,t].
\end{equation}
The $r$- and $\theta$- components of Eq.~(\ref{e5.116}) are written
\begin{eqnarray}\label{e5.118}
\gamma\,B_r &=& -{\rm i}\,(m\,{\Omega}-k \,U)\,B_r \\[0.5ex]
&&+ \frac{\eta}{\mu_0}
\left[\frac{d^2 B_r}{dr^2} + \frac{1}{r}\frac{dB_r}{dr} - \frac{(m^2+k^2 r^2
+1)\,B_r}{r^2}- \frac{{\rm i}\,2\,m\,B_\theta}{r^2}\right],\nonumber
\end{eqnarray}
and
\begin{eqnarray}\label{e5.119}
\gamma\,B_\theta& = &r\,\frac{d{\Omega}}{dr}\,B_r
-{\rm i}\,(m\,{\Omega}-k\, U)\,B_\theta\\[0.5ex]
&&+ \frac{\eta}{\mu_0}
\left[\frac{d^2 B_\theta}{dr^2} + \frac{1}{r}\frac{dB_\theta}{dr} -
 \frac{(m^2+k^2 r^2
+1)\,B_\theta}{r^2}+ \frac{{\rm i}\,2\,m\,B_r}{r^2}\right],\nonumber
\end{eqnarray}
respectively. In general, the term involving $d{\Omega}/dr$
is zero. In fact, this term is only included in the analysis to enable 
us to evaluate the correct matching conditions at $r=a$. Note that we do not need to
write the $z$-component of Eq.~(\ref{e5.116}), since $B_z$ can be obtained 
more directly from $B_r$
and $B_\theta$ via the constraint $\nabla\!\cdot\!{\bf B} = 0$.  

Let
\begin{eqnarray}
B_\pm &=& B_r \pm {\rm i}\,B_\theta,\\[0.5ex]
y &=& \frac{r}{a},\\[0.5ex]
\tau_R &=& \frac{\mu_0\,a^2}{\eta},\\[0.5ex]
q^2 &=& k^2 a^2 + \gamma\tau_R + {\rm i}\,(m\,{\Omega} - k\,U)\,\tau_R,\\[0.5ex]
s^2 &=& k^2 a^2 + \gamma\tau_R.\label{e5.120e}
\end{eqnarray}
Here, $\tau_R$ is the typical time for magnetic flux to diffuse a distance $a$
under the action of resistivity. Equations~(\ref{e5.118})--(\ref{e5.120e}) can be
combined to give
\begin{equation}\label{e5.121}
y^2\,\frac{d^2 B_\pm}{dy^2} + y\,\frac{d B_\pm}{dy} -\left[(m\pm 1)^2 + q^2\,y^2\right]B_\pm = 0
\end{equation}
for $y\leq 1$, and
\begin{equation}\label{e5.122}
y^2\,\frac{d^2 B_\pm}{dy^2} + y\,\frac{d B_\pm}{dy} -\left[(m\pm 1)^2 + s^2\,y^2\right]B_\pm = 0
\end{equation}
for $y>1$. The above equations are immediately recognized as modified
Bessel's equations of order $m\pm 1$.\footnote{M.~Abramowitz, and I.A.~Stegun, {\em Handbook of Mathematical Functions} (Dover, New York NY, 1964), p.~374.} 
Thus, the physical solutions of Eqs.~(\ref{e5.121}) and (\ref{e5.122}), which are well behaved
as $y\rightarrow 0$ and $y\rightarrow\infty$, can be written
\begin{equation}\label{e5.123}
B_\pm = C_\pm\,\frac{I_{m\pm 1}(q\,y)}{I_{m\pm 1}(q)}
\end{equation}
for $y\leq 1$, and
\begin{equation}
B_\pm = D_\pm \, \frac{K_{m\pm 1}(s\,y)}{K_{m\pm 1}(s)}
\end{equation}
for $y>1$. Here, $C_\pm$ and $D_\pm$ are arbitrary constants.
Note that the arguments of $q$ and $s$ are both constrained to lie in the
range $-\pi/2$ to $+\pi/2$. 

The first set of matching conditions at $y=1$ are, obviously, that $B_\pm$ are
continuous, which yields
\begin{equation}
C_\pm = D_\pm.
\end{equation}
The second set of matching conditions are obtained by integrating Eq.~(\ref{e5.119})
from $r=a-\delta$ to $r=a-\delta$, where $\delta$ is an infinitesimal
 quantity, and making use of the fact that the angular velocity ${\Omega}$
jumps discontinuously to zero at $r=a$. It follows that
\begin{equation}
a\,{\Omega}\,B_r = \frac{\eta}{\mu_0}
\left[\frac{d B_\theta}{dr}\right]_{r=a_-}^{r=a_+}.
\end{equation}
Furthermore, integration of Eq.~(\ref{e5.118}) tells us that $dB_r/dr$ is continuous
at $r=a$. We can combine this information to give the matching
condition
\begin{equation}\label{e5.127}
\left[\frac{d B_\pm}{dy}\right]_{y=1_-}^{y=1_+} =\pm{\rm i}\,{\Omega}\,\tau_R\,
\frac{B_+ +B_-}{2}.
\end{equation}

Equations~(\ref{e5.123})--(\ref{e5.127}) can be combined to give the dispersion relation
\begin{equation}\label{e5.128}
G_+\,G_- = \frac{{\rm i}}{2}\,{\Omega}\,\tau_R\,(G_+-G_-),
\end{equation}
where
\begin{equation}
G_\pm = q\,\frac{I_{m\pm 1}'(q)}{I_{m\pm 1}(q)} - s\,\frac{K_{m\pm 1}'(s)}
{K_{m\pm 1}(s)}.
\end{equation}
Here, $'$ denotes a derivative.

Unfortunately, despite the fact that we are investigating the simplest known dynamo,
the dispersion relation (\ref{e5.128}) is sufficiently complicated that it can only
be solved numerically. We can simplify matters considerably
taking the limit $|q|, |s| \gg 1$, which corresponds either
to that of small wave-length ({\em i.e.}, $k\,a\gg 1$), or small
resistivity ({\em i.e.}, ${\Omega}\,\tau_R\gg 1$). 
The large argument asymptotic behaviour of the Bessel functions is
specified by\,\footnote{M.~Abramowitz, and I.A.~Stegun, {\em Handbook of Mathematical Functions} (Dover, New York NY, 1964), p.~377.} 
\begin{eqnarray}
\sqrt{\frac{2\,z}{\pi}}\,K_m(z) &=& {\rm e}^{-z}\left(1+\frac{4\,m^2-1}{8\,z} +\cdots\right),\\[0.5ex]
\sqrt{2\,z\,\pi}\,I_m(z)&=& {\rm e}^{+z}\left(1-\frac{4\,m^2-1}{8\,z} +\cdots\right),
\end{eqnarray}
where $|\arg(z)|<\pi/2$.
It follows that
\begin{equation}
G_\pm = q + s + (m^2/2\pm m + 3/8)(q^{-1} + s^{-1}) + O(q^{-2}+s^{-2}).
\end{equation}
Thus, the dispersion relation (\ref{e5.128}) reduces to
\begin{equation}\label{e5.133}
(q+s)\,q\,s = {\rm i}\,m\,{\Omega}\,\tau_R,
\end{equation}
where $|\arg(q)|$, $|\arg(s)|<\pi/2$. 

In the limit  $\mu\rightarrow 0$, where
\begin{equation}
\mu= (m\,{\Omega}-k\,U)\,\tau_R,
\end{equation}
which corresponds to $({\bf V}\cdot\nabla){\bf B} \rightarrow 0$, the simplified dispersion
relation (\ref{e5.133}) can be solved to give
\begin{equation}\label{e5.135}
\gamma\,\tau_R \simeq {\rm e}^{\,{\rm i}\,\pi/3}\left(\frac{m\,{\Omega}\,\tau_R}
{2}\right)^{2/3} - k^2\,a^2 - {\rm i}\,\frac{\mu}{2}.
\end{equation}
Dynamo behaviour [{\em i.e.}, ${\rm Re}(\gamma)>0$] takes place
when
\begin{equation}\label{e5.136}
{\Omega}\,\tau_R > \frac{2^{5/2}\,(ka)^3}{m}.
\end{equation}
Note that ${\rm Im}(\gamma)\neq 0$, implying that the
dynamo mode {\em oscillates}, or rotates, as well as growing exponentially in time.
The dynamo generated magnetic field is both non-axisymmetric [note that
dynamo activity is impossible, according to Eq.~(\ref{e5.135}), if $m=0$] and
three-dimensional, and is, thus, not subject to either of the anti-dynamo
theorems mentioned in the preceding section.

It is clear from Eq.~(\ref{e5.136}) that dynamo action occurs whenever the flow
is made sufficiently rapid. But, what is the minimum amount of flow
which gives rise to dynamo action? 
In order to answer this question we 
have to solve the full dispersion relation, (\ref{e5.128}), for various values
of $m$ and $k$ in order to find the dynamo mode which grows exponentially in time
for the smallest values of ${\Omega}$ and $U$. It is conventional
to parameterize the flow in terms of the {\em magnetic Reynolds number}
\begin{equation}
S = \frac{\tau_R}{\tau_H},
\end{equation}
where
\begin{equation}
\tau_H = \frac{L}{V}
\end{equation}
is the typical time-scale for convective motion across the system. Here,
$V$ is a typical flow velocity, and $L$ is the scale-length of the system.
Taking $V=|{\bf V}(a)|= \sqrt{{\Omega}^2\,a^2+U^2}$, and $L=a$, we 
have
\begin{equation}
S = \frac{\tau_R\,\sqrt{{\Omega}^2\,a^2+U^2}}{a}
\end{equation}
for the Ponomarenko dynamo. The critical value of the Reynolds number above
which dynamo action occurs is found to be 
\begin{equation}
S_c = 17.7.
\end{equation}
The most unstable dynamo mode is characterized by
 $m=1$, $U/{\Omega}\,a=1.3$, $k\,a=0.39$,
and ${\rm Im}(\gamma)\,\tau_R = 0.41$. As the magnetic Reynolds number, $S$, 
 is increased above
the critical value, $S_c$,  other  dynamo modes are eventually destabilized.

Interestingly enough, an attempt was made in the late 1980's to construct a
Ponomarenko dynamo by rapidly pumping liquid sodium through a cylindrical
pipe equipped with a set of twisted vanes at one end to induce helical
flow. Unfortunately, the experiment failed due to mechanical vibrations,
after achieving  a Reynolds number which was $80\%$ of the critical value required
for self-excitation of the magnetic field, and was not repaired due to budgetary
problems.\footnote{A.~Gailitis, {\em Topological Fluid Dynamics}, edited
by H.K.~Moffatt, and A.~Tsinober (Cambridge University Press,
Cambridge UK, 1990), p.~147.} More recently, there has been renewed
interest worldwide in the idea of
 constructing a liquid metal dynamo, and two such experiments (one in Riga, and one in Karlsruhe) have demonstrated
self-excited dynamo action in a controlled
laboratory setting. 

\section{Magnetic Reconnection}
Magnetic reconnection is a phenomenon which is of particular importance in solar
system plasmas. In the solar corona, it results in the rapid release to the
plasma of energy stored in the large-scale structure of the coronal magnetic
field, an effect which is thought to give rise to {\em solar flares}. Small-scale
reconnection may play a role in heating the corona, and, thereby, driving
the outflow of the solar wind. In the Earth's magnetosphere, magnetic
reconnection in the magnetotail is thought to be
the precursor for {\em auroral sub-storms}.

The evolution of the magnetic field in a resistive-MHD plasma is
governed by the following well-known equation:
\begin{equation}\label{e5.141}
\frac{\partial{\bf B}}{\partial t} = \nabla\times({\bf V}\times{\bf B})+
\frac{\eta}{\mu_0}\,\nabla^2{\bf B}.
\end{equation}
The first term on the right-hand side of this equation describes the
{\em convection}\/ of the magnetic field by the plasma flow. The second
term describes the resistive {\em diffusion}\/ of the field through the plasma.
If the first term dominates then magnetic flux is frozen into the plasma,
and the topology of the magnetic field cannot change. On the other hand,
if the second term dominates then there is little coupling between the
field and the plasma flow, and 
the topology of the magnetic field
is free to change. 

The relative magnitude of the two terms on the right-hand side of Eq.~(\ref{e5.141})
is conventionally measured in terms of {\em magnetic Reynolds number}, or {\em Lundquist
number}:
\begin{equation}
S = \frac{\mu_0\,V\,L}{\eta} \simeq \frac{|\nabla\times({\bf V}\times {\bf B})|}
{|(\eta/\mu_0)\,\nabla^2{\bf B}|},
\end{equation}
where $V$ is the characteristic flow speed, and $L$  the characteristic
length-scale of the plasma. If $S$ is much larger than unity then
convection dominates, and the {\em frozen flux} constraint prevails, whilst
if $S$ is much less than unity then diffusion dominates, and the coupling between
the plasma flow and the magnetic field is relatively weak.

It turns out that in the solar system  very large $S$-values are virtually guaranteed by the
the extremely large scale-lengths of solar system plasmas. For instance,
$S\sim 10^8$ for solar flares, whilst $S\sim 10^{11}$ is appropriate for
the solar wind and the Earth's magnetosphere. Of course, in calculating 
these values we have identified the scale-length $L$ with the overall
size of the plasma under investigation. 

On the basis of the above discussion, it seems reasonable to neglect diffusive
processes altogether in solar system plasmas. Of course, this leads to
very strong constraints on the behaviour of such plasmas, since all
cross-field mixing of plasma elements is suppressed in this limit. Particles
may freely mix along field-lines (within limitations imposed by magnetic
mirroring, {\em etc.}), but are completely ordered perpendicular to the field,
since they always remain tied to the same field-lines as they convect in the
plasma flow.

Let us consider what happens when two initially separate plasma regions
come into contact with one another, as occurs, for example, in the
interaction between the solar wind and the Earth's magnetic field.
Assuming that each plasma is frozen to its own magnetic field, and that cross-field
diffusion  is absent, we conclude that the two plasmas
will not mix, but, instead, that a thin {\em boundary layer} will form
between them, separating the two plasmas and their respective magnetic fields.
In equilibrium, the location of the boundary layer will be determined
by pressure balance. Since, in general, the frozen fields on either side of
the boundary will have differing strengths, and orientations tangential
to the boundary, the layer must also constitute a {\em current sheet}. 
Thus, flux freezing leads inevitably to the prediction that in plasma
systems space becomes divided into separate cells, wholly containing the 
plasma and magnetic field  from individual sources, and separated from
each other by thin current sheets. 

The ``separate cell'' picture constitutes an excellent zeroth-order approximation to
the interaction of solar system plasmas, as witnessed, for example, by
the well defined planetary magnetospheres. It must be noted, however, that
the large $S$-values upon which the applicability of the frozen flux constraint
was justified were derived using the {\em large} overall spatial scales of the
systems involved. However, strict application of this constraint to the
problem of the interaction of separate plasma systems leads to the 
inevitable conclusion
that structures will form having {\em small} spatial scales, at least in
one dimension: {\em i.e.}, the thin current sheets constituting the cell
boundaries. It is certainly not guaranteed, from the above discussion,  that
the effects of diffusion can be neglected in 
 these boundary layers.  In fact, we shall demonstrate that 
the localized breakdown of the flux freezing constraint in the boundary
regions, due to diffusion, not only has an impact on the
properties of the boundary regions themselves, but can also have a decisive
impact on the large length-scale plasma regions where the flux freezing 
constraint remains valid. This observation illustrates both the subtlety and the
significance of the magnetic reconnection process. 

\section{Linear Tearing Mode Theory}
Consider the interface between two plasmas  containing magnetic fields of
different orientations. The simplest imaginable field configuration is that
illustrated in Fig.~\ref{f26}. Here, the field varies only in the $x$-direction, and 
points only in the $y$-direction. The field is directed in the
$-y$-direction for $x<0$, and in the $+y$-direction for $x>0$. The interface
is situated at $x=0$. The sudden reversal of the field direction across the
interface gives rise to a $z$-directed current sheet at $x=0$. 

\begin{figure}
\epsfysize=3in
\centerline{\epsffile{Chapter05/tear.eps}}
\caption{\em A reconnecting magnetic field configuration.}\label{f26}
\end{figure}

With the neglect of plasma resistivity, the field configuration shown in Fig.~\ref{f26}
represents a {\em  stable} equilibrium state, assuming, of course,
 that we have normal pressure balance
across the interface. But, does the field configuration remain stable when we take
resistivity into account? If not, we expect an instability to develop which relaxes the
configuration to one possessing  lower magnetic energy. As we shall see, this
type of relaxation process inevitably entails the breaking and reconnection of magnetic
field lines, and is, therefore, termed {\em magnetic reconnection}. The
magnetic energy released during the reconnection process eventually appears as 
plasma thermal
energy. Thus, magnetic reconnection also involves plasma heating.

In the following, we shall outline the standard method for determining the
{\em linear} stability of the type of magnetic field configuration
shown in Fig.~26, taking into account the effect of plasma resistivity.
We are particularly
interested in plasma instabilities which are stable in the absence of resistivity,
and only grow when the resistivity is non-zero. Such instabilities are
conventionally termed {\em tearing modes}. 
Since magnetic reconnection is, in fact, a {\em nonlinear} process, we shall
then proceed to  investigate the nonlinear development of tearing modes.

The equilibrium magnetic field is written
\begin{equation}
{\bf B}_0 = B_{0\,y}(x)\,\hat{\bf y},
\end{equation}
where $B_{0\,y}(-x)=-B_{0\,y}(x)$. 
There is assumed to be no equilibrium plasma flow. The linearized
equations of resistive-MHD, assuming incompressible flow, take the form
\begin{eqnarray}\label{e5.144a}
\frac{\partial {\bf B}}{\partial t} &=& \nabla\times({\bf V}\times{\bf B}_0)
+ \frac{\eta}{\mu_0}\,\nabla^2{\bf B},\\[0.5ex]\label{e5.144b}
\rho_0\,\frac{\partial{\bf V}}{\partial t} &=& -\nabla p + 
\frac{(\nabla\times{\bf B})\times{\bf B}_0}{\mu_0} + 
\frac{(\nabla\times{\bf B}_0)\times{\bf B}}{\mu_0}\\[0.5ex]\label{e5.144c}
\nabla\cdot{\bf B} &=& 0,\\[0.5ex]
\nabla\cdot{\bf V} &=& 0.\label{e5.144d}
\end{eqnarray}
Here, $\rho_0$ is the equilibrium plasma density, ${\bf B}$  the
perturbed magnetic field, ${\bf V}$  the perturbed plasma velocity, and
$p$  the perturbed plasma pressure. The assumption of incompressible
plasma flow is valid provided that the plasma velocity associated with the
instability remains significantly
smaller than both the Alfv\'{e}n velocity and the sonic velocity. 

Suppose that all perturbed quantities vary like
\begin{equation}
A(x,y,z,t) = A(x)\,{\rm e}^{\,{\rm i}\,k\,y +\gamma\,t},
\end{equation}
where $\gamma$ is the instability growth-rate. The $x$-component of
Eq.~(\ref{e5.144a}) and the $z$-component of the curl of Eq.~(\ref{e5.144b})
reduce to
\begin{eqnarray}\label{e5.146a}
\gamma\,B_x &=& {\rm i}\,kB_{0\,y} \,V_x + \frac{\eta}{\mu_0}
\left(\frac{d^2}{dx^2} - k^2\right) B_x,\\[0.5ex]\label{e5.146b}
\gamma\,\rho_0\,\left(\frac{d^2}{dx^2}-k^2\right) V_x
&=& \frac{{\rm i}\,kB_{0\,y}}{\mu_0}\left(
\frac{d^2}{dx^2} - k^2 - \frac{B_{0\,y}''}{B_{0\,y}}\right) B_x,
\end{eqnarray}
respectively, where use has been made of Eqs.~(\ref{e5.144c}) and (\ref{e5.144d}).
Here, $'$ denotes $d/dx$.

It is convenient to normalize Eqs.~(\ref{e5.146a})--(\ref{e5.146b}) using a typical magnetic
field-strength, $B_0$, and a typical scale-length, $a$. Let us define the 
{\em Alfv\'{e}n time-scale}
\begin{equation}
\tau_A = \frac{a}{V_A},
\end{equation}
 where $V_A = B_0/\sqrt{\mu_0\,\rho_0}$
is the Alfv\'{e}n velocity, and the {\em resistive diffusion} time-scale
\begin{equation}\label{e5.148}
\tau_R = \frac{\mu_0\,a^2}{\eta}.
\end{equation}
The ratio of these two time-scales is the Lundquist number:
\begin{equation}
S = \frac{\tau_R}{\tau_A}.
\end{equation}
Let
$\psi= B_x/B_0$, $\phi = {\rm i}\,k\,V_y/\gamma$, $\bar{x}=x/a$, 
$F=B_{0\,y}/B_0$, $F'\equiv dF/d\bar{x}$, $\bar{\gamma} = \gamma\,\tau_A$,
and $\bar{k} = k\,a$. It follows that
\begin{eqnarray}\label{e5.150a}
\bar{\gamma}\,(\psi-F\,\phi) &=& S^{-1}\left(\frac{d^2}{d\bar{x}^2}-\bar{k}^2\right)
\psi,\\[0.5ex]\label{e5.150b}
\bar{\gamma}^2\left(\frac{d^2}{d\bar{x}^2} -\bar{k}^2\right)\phi
&=& -\bar{k}^2\,
F \left( \frac{d^2}{d\bar{x}^2} -\bar{k}^2 - \frac{F''}{F}\right) \psi.
\end{eqnarray}
The term on the right-hand side of Eq.~(\ref{e5.150a}) represents plasma
{\em resistivity}, whilst the term on the left-hand side of Eq.~(\ref{e5.150b})
represents plasma {\em inertia}.

It is assumed that the tearing instability  grows on a {\em hybrid} time-scale 
which is much less than $\tau_R$ but much greater than $\tau_A$. It follows
that 
\begin{equation}
\bar{\gamma} \ll 1 \ll S\,\bar{\gamma}.
\end{equation}
Thus, throughout most of the plasma we can neglect the right-hand side of
Eq.~(\ref{e5.150a}) and the left-hand side of Eq.~(\ref{e5.150b}), which is equivalent to
the neglect of plasma resistivity and inertia. In this case, Eqs.~(\ref{e5.150a})--(\ref{e5.150b})
reduce to
\begin{eqnarray}\label{e5.152a}
\phi &=& \frac{\psi}{F},\\[0.5ex]\label{e5.152b}
\frac{d^2\psi}{d\bar{x}^2} - \bar{k}^2\,\psi - \frac{F''}{F}\,\psi &=& 0.
\end{eqnarray}
Equation~(\ref{e5.152a}) is simply the flux freezing constraint, which requires the
plasma to move with the magnetic field. Equation~(\ref{e5.152b}) is the linearized,
static 
force balance criterion: $\nabla\times({\bf j}\times{\bf B}) = {\bf0}$.
Equations~(\ref{e5.152a})--(\ref{e5.152b}) are known collectively as the equations of {\em ideal-MHD}, and are valid
throughout virtually the whole plasma. However, it is clear that these
equations {\em break down} in the immediate vicinity of the interface, where $F=0$
({\em i.e.}, where the magnetic field reverses direction). Witness, for instance,
the fact that the normalized ``radial'' velocity, $\phi$, becomes infinite as
$F\rightarrow 0$, according to Eq.~(\ref{e5.152a}). 

The ideal-MHD equations break down close to the interface because the neglect
of plasma resistivity and inertia becomes untenable as $F\rightarrow 0$.
Thus, there is a thin layer, in the immediate vicinity
of the interface, $\bar{x}=0$, where the behaviour of the plasma is governed
by the full MHD equations, (\ref{e5.150a})--(\ref{e5.150b}). We can simplify these equations,
making use of the fact that $\bar{x}\ll 1$ and $d/d\bar{x} \gg 1$ in a
thin layer, to obtain the following layer equations:
\begin{eqnarray}\label{e5.153a}
\bar{\gamma}\,(\psi - \bar{x}\,\phi) &=& S^{-1}\frac{d^2\psi}{d\bar{x}^2},\\[0.5ex]
\bar{\gamma}^2\,\frac{d^2\phi}{d\bar{x}^2} &=& - \bar{x}\,\frac{d^2\psi}{d\bar{x}^2}.\label{e5.153b}
\end{eqnarray}
Note that we have redefined the variables $\phi$, $\bar{\gamma}$, and $S$, such
that $\phi\rightarrow F'(0)\,\phi$, $\bar{\gamma}\rightarrow \gamma\,\tau_H$,
and $S\rightarrow \tau_R/\tau_H$. Here,
\begin{equation}
\tau_H = \frac{\tau_A}{k\,a\,F'(0)}
\end{equation}
is the {\em hydromagnetic time-scale}.

The tearing mode stability problem reduces to solving the non-ideal-MHD layer equations,
(\ref{e5.153a})--(\ref{e5.153b}), in the immediate vicinity of the interface, $\bar{x}=0$, solving
the ideal-MHD equations, (\ref{e5.152a})--(\ref{e5.152b}), everywhere else in the plasma, matching
the two solutions at the edge of the layer, and applying physical
boundary conditions as $|\bar{x}|\rightarrow\infty$. This method
of solution was first described in a classic paper by Furth, Killeen, and
Rosenbluth.\footnote{H.P.~Furth, J.~Killeen, and M.N.~Rosenbluth,
Phys.\ Fluids {\bf 6}, 459 (1963).}

Let us consider the solution of the ideal-MHD equation (\ref{e5.152b}) throughout
the bulk of the plasma. We could imagine launching a solution $\psi(\bar{x})$
 at large positive
$\bar{x}$, which satisfies physical boundary conditions as $\bar{x}\rightarrow
\infty$, and integrating this solution to the right-hand boundary of
the non-ideal-MHD layer at $\bar{x}=0_+$. Likewise, we could also launch
a solution at large negative $\bar{x}$, which satisfies physical boundary
conditions as $\bar{x}\rightarrow-\infty$, and integrate this solution to
the left-hand boundary of the non-ideal-MHD layer at $\bar{x}=0_-$. 
Maxwell's equations demand that $\psi$ must be continuous on either side
of the layer.
Hence, we can multiply our two solutions by appropriate factors, so as to ensure that
$\psi$ matches to the left and right of the layer. This leaves
the function  $\psi(\bar{x})$ undetermined
to an overall arbitrary multiplicative constant, just  as we would expect in a
linear problem. In general, $d\psi/d\bar{x}$ is {\em not} continuous to the left and
right of the layer. Thus, the ideal solution can be characterized by the
real number
\begin{equation}
{\Delta}' =
\left[\frac{1}{\psi}\frac{d\psi}{d\bar{x}}\right]_{\bar{x}=0_-}^{\bar{x}=0_+}:
\end{equation}
{\em i.e.}, by the jump in the logarithmic derivative of $\psi$ to the left and
right of the layer.
This parameter is known as the {\em tearing stability index}, and is solely
a property
of the plasma equilibrium, the wave-number, $k$, 
 and the boundary conditions imposed at infinity.

The layer equations (\ref{e5.153a})--(\ref{e5.153b}) possess a trivial solution ($\phi=\phi_0$,
$\psi=\bar{x}\,\phi_0$, where $\phi_0$ is independent of $\bar{x}$), and
a nontrivial solution for which $\psi(-\bar{x})=\psi(\bar{x})$ and $\phi(-\bar{x})
=-\phi(\bar{x})$.
The asymptotic behaviour of the nontrivial solution at the
edge of the layer is
\begin{eqnarray}\label{e5.192x}
\psi(x) &\rightarrow & \left(\frac{\Delta}{2}\,|\bar{x}| + 1\right)\,
{\Psi},\\[0.5ex]
\phi(x) &\rightarrow & \frac{\psi}{\bar{x}},\label{e5.193x}
\end{eqnarray}
where the parameter ${\Delta}(\bar{\gamma}, S)$ is determined by solving the
layer equations, subject to the above boundary
conditions. Finally, the growth-rate, $\gamma$, of the tearing instability is determined
by the matching criterion
\begin{equation}\label{e5.157}
{\Delta}(\bar{\gamma}, S) = {\Delta}'.
\end{equation}

The layer equations (\ref{e5.153a})--(\ref{e5.153b}) can be solved in a fairly straightforward manner
in Fourier transform space. Let
\begin{eqnarray}\label{e5.158a}
\phi(\bar{x}) &=& \int_{-\infty}^{\infty}
\hat{\phi}(t) \,{\rm e}^{\,{\rm i}\,S^{1/3}\, \bar{x}\,t}\,
dt,\\[0.5ex]
\psi(\bar{x}) &=& \int_{-\infty}^{\infty} \hat{\psi}(t)\, 
{\rm e}^{\,{\rm i}\,S^{1/3}\,\bar{x}\,t}\,dt,\label{e5.158b}
\end{eqnarray}
where $\hat{\phi}(-t)=-\hat{\phi}(t)$.
Equations~(\ref{e5.153a})--(\ref{e5.153b}) can be Fourier transformed, and the results combined, to
give
\begin{equation}\label{e5.159}
\frac{d}{dt}\!\left(\frac{t^2}{Q+t^2}\frac{d\hat{\phi}}{dt}\right)
-Q\,t^2\,\hat{\phi} = 0,
\end{equation}
where
\begin{equation}
Q = \gamma\,\tau_H^{2/3}\,\tau_R^{1/3}.
\end{equation}

The most general small-$t$ asymptotic solution of Eq.~(\ref{e5.159}) is written
\begin{equation}\label{e5.161}
\hat{\phi}(t) \rightarrow \frac{a_{-1}}{t} + a_0 + O(t),
\end{equation}
where $a_{-1}$ and $a_0$ are independent of $t$, and it is assumed that $t>0$. 
When inverse Fourier transformed, the above expression leads to the
following expression for the asymptotic behaviour of
$\phi$ at the edge of
the non-ideal-MHD layer:
\begin{equation}
\phi(\bar{x})\rightarrow a_{-1}\,\frac{\pi}{2}\,S^{1/3}\,{\rm sgn}(x) + \frac{a_0}{\bar{x}}
+O(|\bar{x}|^{-2}).
\end{equation}
It follows from a comparison with Eqs.~(\ref{e5.192x})--(\ref{e5.193x}) that
\begin{equation}\label{e5.163}
{\Delta} = \pi\,\frac{a_{-1}}{a_0}\,S^{1/3}.
\end{equation}
Thus, the matching parameter ${\Delta}$ is determined from  the
small-$t$ asymptotic behaviour of the Fourier transformed layer solution.

Let us search for an unstable tearing mode, characterized by $Q>0$. It is
convenient to assume that
\begin{equation}\label{e5.164}
Q\ll 1.
\end{equation}
This ordering, which is known as the {\em constant-$\psi$ approximation} [since
it implies that $\psi(\bar{x})$ is approximately constant across the layer]
will be justified later on. 

In the limit $t\gg Q^{1/2}$, Eq.~(\ref{e5.159})
reduces to
\begin{equation}\label{e5.165}
\frac{d^2\hat{\phi}}{d t^2} - Q\,t^2\,\hat{\phi} = 0.
\end{equation}
The solution to this equation which is well behaved in the limit
$t\rightarrow \infty$ is written $U(0,\sqrt{2}\,Q^{1/4}\,t)$, where
$U(a,x)$ is a standard parabolic cylinder function.\footnote{M.~Abramowitz, and I.A.~Stegun, {\em Handbook of Mathematical Functions} (Dover, New York NY, 1964),
 p.~686.}  In the limit
\begin{equation}\label{e5.166}
Q^{1/2} \ll t \ll Q^{-1/4}
\end{equation}
we can make use of the standard small argument asymptotic expansion
of $U(a,x)$ to write the most general solution to Eq.~(\ref{e5.159}) in the
form
\begin{equation}\label{e5.167}
\hat{\phi}(t) = A\left[ 1-  2 \,\frac{{\Gamma}(3/4)}{{\Gamma}(1/4)}
\, Q^{1/4}\,t + O(t^2)\right].
\end{equation} 
Here, $A$ is an arbitrary constant.

In the limit
\begin{equation}
t \ll Q^{-1/4},
\end{equation}
Eq.~(\ref{e5.159}) reduces to
\begin{equation}
\frac{d}{dt}\!\left(\frac{t^2}{Q+t^2}\,\frac{d\hat{\phi}}{dt} \right) = 0.
\end{equation}
The most general solution to this equation is written
\begin{equation}\label{e5.170}
\hat{\phi}(t) = B\!\left(-\frac{Q}{t} + t\right) + C +O(t^2),
\end{equation}
where $B$ and $C$ are arbitrary constants. 
Matching coefficients between Eqs.~(\ref{e5.167}) and (\ref{e5.170}) in the range of $t$
satisfying the inequality (\ref{e5.166}) yields the following expression
for the most general solution to Eq.~(\ref{e5.159}) in the limit  $t\ll Q^{1/2}$:
\begin{equation}\label{e5.171}
\hat{\phi} = A\,\left[ 2\,\frac{{\Gamma}(3/4)}{{\Gamma}(1/4)}
\, \frac{Q^{5/4}}{t} + 1 + O(t)\right].
\end{equation}
Finally, a comparison of Eqs.~(\ref{e5.161}), (\ref{e5.163}), and (\ref{e5.171})
yields the result
\begin{equation}\label{e5.172}
{\Delta} = 2\pi\,\frac{{\Gamma}(3/4)}{{\Gamma}(1/4)}\,S^{1/3}\,
Q^{5/4}.
\end{equation}

The asymptotic matching condition (\ref{e5.157}) can be combined with the above
expression for ${\Delta}$ to give the tearing mode dispersion relation
\begin{equation}\label{e5.173}
\gamma = \left[\frac{{\Gamma}(1/4)}{2\pi\,{\Gamma}(3/4)}\right]^{4/5}\,
\frac{({\Delta}')^{4/5}}{\tau_H^{2/5}\,\tau_R^{3/5}}.
\end{equation}
Here, use has been made of the definitions of $S$ and $Q$. According
to the above dispersion relation, the tearing mode is unstable whenever ${\Delta}'>0$, and grows on the hybrid time-scale $\tau_H^{2/5}\,\tau_R^{3/5}$. 
It is easily demonstrated that the tearing mode is stable whenever ${\Delta}'<0$.
According to Eqs.~(\ref{e5.157}), (\ref{e5.164}), and (\ref{e5.172}), the constant-$\psi$
approximation holds provided that
\begin{equation}
{\Delta}' \ll S^{1/3}:
\end{equation}
{\em i.e.}, provided that the tearing mode does not become too
unstable.

From  Eq.~(\ref{e5.165}), the thickness of the non-ideal-MHD layer in $t$-space
is
\begin{equation}
\delta_t \sim \frac{1}{Q^{1/4}}.
\end{equation}
It follows from Eqs.~(\ref{e5.158a})--(\ref{e5.158b}) that the thickness of the layer in $\bar{x}$-space
is
\begin{equation}\label{e5.176}
\bar{\delta} \sim \frac{1}{S^{1/3}\,\delta_t} \sim \left(
\frac{\bar{\gamma}}{S}\right)^{1/4}.
\end{equation}
When ${\Delta}'\sim 0(1)$ then $\bar{\gamma}\sim S^{-3/5}$, according to
Eq.~(\ref{e5.173}), giving $\bar{\delta}\sim S^{-2/5}$. It is clear, therefore, that if
the Lundquist number, $S$, is very large then the non-ideal-MHD layer centred
on the interface, $\bar{x}=0$,  is {\em extremely narrow}.

The time-scale for magnetic flux to diffuse across a layer of thickness
$\bar{\delta}$ (in $\bar{x}$-space) is [{\em cf.}, Eq.~(\ref{e5.148})]
\begin{equation}\label{e5.177}
\tau \sim \tau_R\,\bar{\delta}^{~2}.
\end{equation}
If 
\begin{equation}
\gamma\,\tau\ll 1,
\end{equation}
then the tearing mode grows on a time-scale which is far longer than the
time-scale on which magnetic flux diffuses across the non-ideal layer. In this case,
we would expect the normalized ``radial'' magnetic field, $\psi$, to be approximately
{\em constant} across the layer, since any non-uniformities in $\psi$ would be
smoothed out via resistive diffusion. It follows from Eqs.~(\ref{e5.176}) and (\ref{e5.177})
that the constant-$\psi$ approximation holds provided that
\begin{equation}
\bar{\gamma} \ll S^{-1/3}
\end{equation}
({\em i.e.}, $Q\ll 1$), which is in agreement with Eq.~(\ref{e5.164}).

\section{Nonlinear Tearing Mode Theory}
We have seen that if ${\Delta}'>0$ then a magnetic field configuration
of the type shown in Fig.~\ref{f26} is unstable to a tearing mode.
Let us now investigate how a tearing instability affects the field
configuration as it develops.

It is convenient to write the magnetic field in terms of a flux-function:
\begin{equation}
{\bf B} = B_0\,a\,\nabla\psi\times\hat{\bf z}.
\end{equation}
Note that ${\bf B}\!\cdot\!\nabla\psi=0$. It follows that magnetic field-lines
run along contours of $\psi(x,y)$. 

We can write
\begin{equation}
\psi(\bar{x},\bar{y}) \simeq \psi_0(\bar{x}) + \psi_1(\bar{x},\bar{y}),
\end{equation}
where $\psi_0$ generates the equilibrium magnetic field, and $\psi_1$ generates
the perturbed magnetic field associated with the tearing mode. 
Here, $\bar{y}= y/a$. 
In the vicinity of the interface, we have
\begin{equation}
\psi \simeq - \frac{F'(0)}{2}\,\bar{x}^{~2} + {\Psi}\,\cos \bar{k}\, \bar{y},
\end{equation}
where ${\Psi}$ is a constant. Here, we have made use of the
fact that $\psi_1(\bar{x},\bar{y})\simeq \psi_1(\bar{y})$ if the
constant-$\psi$ approximation holds good (which is assumed to be the case). 

Let   $\chi = -\psi/{\Psi}$ and $\theta=\bar{k}\, \bar{y}$. 
It follows that the normalized perturbed magnetic flux function, $\chi$, 
in the vicinity of the interface takes the form
\begin{equation}
\chi = 8\,X^2 - \cos\theta,
\end{equation}
where $X = \bar{x}/\bar{W}$, and
\begin{equation}
\bar{W} = 4\sqrt{\frac{{\Psi}}{F'(0)}}.
\end{equation}
Figure~\ref{f27} shows the contours of $\chi$ plotted in $X$-$\theta$ space. It can
be seen that the tearing mode 	gives rise to
the formation of a {\em magnetic island} centred on the interface, $X=0$. 
Magnetic field-lines situated outside the separatrix are displaced by the
tearing mode, but still retain their original topology. By contrast, field-lines
inside the separatrix have been broken and reconnected, and now possess
quite different topology. The reconnection obviously takes place at the ``X-points,''
which are located at $X=0$ and $\theta = j\,2\pi$, where $j$ is an integer.
The maximum width of the reconnected region (in $\bar{x}$-space) is given by
the {\em island width}, $a\,\bar{W}$. Note that the island width is proportional
to the square root of the perturbed ``radial'' magnetic field at the interface
({\em i.e.}, $\bar{W}\propto \sqrt{\Psi}$).

\begin{figure}
\epsfysize=3in
\centerline{\epsffile{Chapter05/island.eps}}
\caption{\em Magnetic field-lines in the vicinity of a magnetic island.}\label{f27}
\end{figure}

According to a result first established in a very elegant paper by 
Rutherford,\footnote{P.H.~Rutherford, Phys.\ Fluids {\bf 16}, 1903 (1973).}
the nonlinear evolution of the island width is governed by
\begin{equation}
0.823\,\tau_R\,\frac{d\bar{W}}{dt} = {\Delta}'(\bar{W}),
\end{equation}
where
\begin{equation}
{\Delta}'(\bar{W}) =
 \left[\frac{1}{\psi}\frac{d\psi}{d\bar{x}}\right]_{-\bar{W}/2}^{+\bar{W}/2}
\end{equation}
is the jump in the logarithmic derivative of $\psi$ {\em taken across the island}.
It is clear that once the tearing mode enters the nonlinear regime ({\em i.e.},
once the normalized island width, $\bar{W}$, exceeds the normalized linear layer width, $S^{-2/5}$), 
the growth-rate of the instability slows down considerably, until the mode
eventually ends up growing on the extremely slow resistive time-scale, $\tau_R$. 
The tearing mode stops growing when it has attained a saturated island width
$\bar{W}_0$, satisfying
\begin{equation}
{\Delta}'(\bar{W}_0) = 0.
\end{equation}
The saturated width is a function of the original 
plasma equilibrium, but is independent
of the resistivity. Note that there is no particular reason why $\bar{W}_0$ should
be small: {\em i.e.}, in general, the saturated island width is comparable
with the scale-length of the  magnetic field configuration.
We conclude that, although ideal-MHD only breaks down in a narrow region of
width $S^{-2/5}$, centered on the interface, $\bar{x}=0$, the reconnection of
magnetic field-lines which takes place
in this region is capable of significantly
modifying the whole magnetic field configuration. 

\section{Fast Magnetic Reconnection}
Up to now, we have only considered {\em spontaneous magnetic reconnection}, which
develops from an instability of the plasma. As we have seen, such reconnection
takes place at a fairly leisurely pace. Let us now consider {\em forced
magnetic reconnection}\/ in which the reconnection takes place as a consequence
of an externally imposed flow or magnetic perturbation, rather than
developing spontaneously. The principle difference between forced
and spontaneous reconnection is the development of extremely large, positive
${\Delta}'$ values in the former case. Generally speaking, we expect ${\Delta}'$
to be $O(1)$ for spontaneous reconnection.  By analogy with the previous
analysis, we would expect forced reconnection to proceed {\em faster} than
spontaneous reconnection (since the reconnection rate increases with
increasing ${\Delta}'$). The question is, how much faster? To be more
exact, if we take the limit ${\Delta}'\rightarrow \infty$, which
corresponds to the limit of extreme forced reconnection, just  how fast can we
make the magnetic field reconnect? At present, this is a {\em very controversial}
question, which is far from being completely
resolved. In the following, we shall content
ourselves with a discussion of  the two ``classic''  fast reconnection
models. These models form the starting point of virtually all recent research on this
subject.

Let us first consider the {\em Sweet-Parker} model, which was first proposed
 by Sweet\footnote{P.A.~Sweet, {\em Electromagnetic Phenomena in Cosmical Physics},
(Cambridge University Press, Cambridge UK, 1958).} and
 Parker.\footnote{E.N.~Parker, J.~Geophys.\ Res.\ {\bf 62}, 509 (1957).} The
main features of the envisioned magnetic and plasma flow
fields  are illustrated in Fig.~\ref{f28}. The system is two
dimensional and steady-state ({\em i.e.}, $\partial/\partial z\equiv 0$ and
$\partial/\partial t\equiv 0$). The reconnecting magnetic fields are anti-parallel,
and of equal strength, $B_\ast$. We imagine that these fields are 
being forcibly
pushed together via the action of some external agency. 
We expect a strong current sheet to form at the boundary between the
two fields, where the direction of ${\bf B}$ suddenly changes.
This current sheet is assumed to be of thickness
$\delta$ and length $L$. 

\begin{figure}
\epsfysize=2.5in
\centerline{\epsffile{Chapter05/sp.eps}}
\caption{\em The Sweet-Parker magnetic reconnection scenario.}\label{f28}
\end{figure}

Plasma is assumed to diffuse into the current layer, along its whole length,
at some relatively small inflow velocity, $v_0$. The plasma is accelerated
along the layer, and eventually expelled from its two ends at some
relatively large exit velocity, $v_\ast$. The inflow velocity
is simply an ${\bf E}\times{\bf B}$ velocity, so
\begin{equation}
v_0 \sim \frac{E_z}{B_\ast}.
\end{equation}
The $z$-component of Ohm's law yields
\begin{equation}
E_z \sim \frac{\eta\,B_\ast}{\mu_0\,\delta}.
\end{equation}
Continuity of plasma flow inside the layer gives
\begin{equation}
L\,v_0 \sim \delta\,v_\ast,
\end{equation}
assuming incompressible flow.
Finally, pressure balance along the length of the layer yields
\begin{equation}
\frac{B_\ast^{~2}}{\mu_0} \sim \rho\,v_\ast^{~2}.
\end{equation}
Here, we have balanced the magnetic pressure at the centre of the layer
against the dynamic pressure of the outflowing plasma at the  ends of the
layer. Note that $\eta$ and $\rho$ are the plasma resistivity and density,
respectively.

We can measure the rate of reconnection via
the inflow velocity, $v_0$, since all of the magnetic field-lines which are
convected into the layer, with the plasma, are eventually reconnected.
The Alfv\'{e}n velocity is written
\begin{equation}
V_A = \frac{B_\ast}{\sqrt{\mu_0\,\rho}}.
\end{equation}
Likewise, we can write the Lundquist number of the plasma as
\begin{equation}
S = \frac{\mu_0\,L\,V_A}{\eta},
\end{equation}
where we have assumed that the length of the reconnecting layer, $L$,
is commensurate with the macroscopic length-scale of the system.
The reconnection rate is parameterized via the Alfv\'{e}nic Mach number of
the inflowing plasma, which is defined
\begin{equation}
M_0  = \frac{v_0}{V_A}.
\end{equation}

The above equations can be rearranged to give
\begin{equation}
v_\ast \sim V_A:
\end{equation}
{\em i.e.}, the plasma is squirted out of the ends of the
reconnecting layer at the Alfv\'{e}n velocity. Furthermore,
\begin{equation}
\delta \sim M_0\,L,
\end{equation}
and
\begin{equation}
M_0 \sim S^{-1/2}.
\end{equation}
We conclude that the reconnecting layer is extremely
narrow, assuming that the Lundquist number of the plasma is
very large. The magnetic reconnection
takes place on the hybrid time-scale $\tau_A^{1/2}\,\tau_R^{1/2}$,
where $\tau_A$ is the Alfv\'{e}n transit time-scale across the
plasma, and $\tau_R$ is the resistive diffusion time-scale across the
plasma. 

The Sweet-Parker reconnection ansatz  is undoubtedly correct.
It has been simulated numerically innumerable times, and was recently
confirmed experimentally in the Magnetic Reconnection Experiment (MRX)
operated by Princeton Plasma Physics Laboratory.\footnote{
H.~Ji, M.~Yamada, S.~Hsu, and R.~Kulsrud,  Phys.\ Rev.\ Lett.\ {\bf 80},
 3256 (1998).}  The problem is that
Sweet-Parker reconnection takes place {\em far too slowly} to account for
many reconnection processes which are thought to take place in the
solar system. For instance, in solar flares $S\sim 10^8$, $V_A\sim 100\,{\rm km}\,{\rm s}^{-1}$, and $L\sim 10^4\,{\rm km}$. According to the
Sweet-Parker model, magnetic energy is released to the plasma via
reconnection on a typical time-scale of a few tens of days. In reality,
the energy is released in a few minutes to an hour. Clearly, we can only hope to
account for solar flares using a reconnection mechanism which operates
{\em far faster} than the Sweet-Parker mechanism.

One, admittedly rather controversial, resolution of this problem was suggested by
Petschek.\footnote{H.E.~Petschek, {\em AAS-NASA Symposium on the Physics
of Solar Flares} (NASA Spec.\ Publ.\ Sp-50, 1964), p.~425.} He pointed
out that magnetic energy can be converted into plasma thermal energy as a
result of shock waves being set up in the plasma, in addition to the
conversion due to the action of resistive
diffusion. The configuration envisaged by Petschek is sketched in Fig.~\ref{f29}.
Two waves (slow mode shocks) stand in the flow on either side of the
interface, where the direction of ${\bf B}$ reverses, marking the
boundaries of the plasma outflow regions. A small diffusion region still
exists on the interface, but now constitutes a miniature (in length)
Sweet-Parker system. The width of the reconnecting layer is
given by 
\begin{equation}
\delta = \frac{L}{M_0\,S},
\end{equation}
just as in the Sweet-Parker model. However, we do not now assume that the
length, $L_\ast$, of the layer is comparable to the scale-size, $L$, 
of the system. Rather, the length may be considerably smaller than $L$, and
is determined self-consistently from the continuity condition
\begin{equation}
L_\ast = \frac{\delta}{M_0},
\end{equation}
where we have assumed incompressible flow, and an outflow speed of
order the Alfv\'{e}n speed, as before. Thus, if the inflow speed, $v_0$, is
much less than $V_A$ then the length of the reconnecting layer
is much larger than its width, as assumed by Sweet and Parker. On the
other hand, if we allow the inflow velocity to
approach the Alfv\'{e}n velocity then the layer shrinks in length, so that
$L_\ast$ becomes comparable with $\delta$. 

\begin{figure}
\epsfysize=2.5in
\centerline{\epsffile{Chapter05/pet.eps}}
\caption{\em The Petschek magnetic reconnection scenario.}\label{f29}
\end{figure}

It follows that for reasonably large reconnection rates ({\em i.e.}, $M_0\rightarrow
1$) the length of the diffusion region becomes much smaller than the scale-size
of the system, $L$, so that most of the plasma flowing into the
boundary region does so across the standing waves, rather than through the central
diffusion region. The angle $\theta$ that the shock waves make with
the interface is given approximately
by
\begin{equation}
\tan\theta \sim M_0.
\end{equation}
Thus, for small inflow speeds the outflow is confined to a narrow
wedge along the interface, but as the inflow speed increases the angle
of the outflow wedges increases to accommodate the increased flow. 

It turns out that there is a maximum inflow speed beyond which Petschek-type
solutions cease to exist. The corresponding maximum Alfv\'{e}nic Mach number,
\begin{equation}
(M_0)_{\rm max} = \frac{\pi}{8 \ln S},
\end{equation}
can be regarded as specifying the maximum allowable rate of magnetic
reconnection according to  the Petschek model. Clearly, since the maximum reconnection
rate depends inversely on the logarithm of the Lundquist number, rather
than its square root, it is much larger than that predicted by the
Sweet-Parker model. 

It must be pointed out that the Petschek model is {\em very} controversial. Many
physicists think that it is completely wrong, and that the maximum 
rate of magnetic reconnection allowed by  MHD is that predicted by the
Sweet-Parker model. In particular, Biskamp\footnote{D.~Biskamp, Phys.\ Fluids
{\bf 29}, 1520 (1986).} wrote an influential and widely quoted paper reporting the
results of a numerical experiment which appeared to disprove the Petschek
model. When the plasma inflow exceeded that allowed by
the  Sweet-Parker model, there was no
acceleration of the reconnection rate. Instead, magnetic flux ``piled up''
in front of the reconnecting layer, and the rate of reconnection never deviated
significantly from that predicted by the Sweet-Parker model. Priest and
Forbes\footnote{E.R.~Priest, and T.G.~Forbes, J.\ Geophys.\ Res.\ {\bf 97}, 
16757 (1992).} later argued that Biskamp imposed boundary conditions in
his numerical experiment which precluded Petschek reconnection. Probably
the most powerful argument against the validity of the Petschek model is
the fact that, more than 30 years after it was first proposed, nobody has
ever managed to simulate Petschek reconnection numerically (except by artificially
increasing the resistivity in the reconnecting region---which is not a
legitimate approach). 

\section{MHD Shocks}
Consider a subsonic disturbance moving through a conventional neutral fluid.
As is well-known, {\em sound waves}\/ propagating ahead of the disturbance
give advance warning of its arrival, and,
thereby, allow the  response of the fluid to be both smooth  and adiabatic. Now, consider a supersonic  distrurbance. In this case, sound waves are
unable to propagate ahead of the disturbance, and so there is no advance warning of its
arrival, and, consequently,  the fluid response is sharp and non-adiabatic. This type of response is generally known as a {\em shock}. 

 Let us investigate shocks in MHD fluids. Since information in such fluids is
carried via three different waves---namely,  {\em fast}\/
or compressional-Alfv\'{e}n waves, {\em intermediate}\/ or shear-Alfv\'{e}n waves, and  {\em slow}\/ or magnetosonic waves (see Sect.~\ref{s5.4})---we might expect MHD fluids to support
three different types of shock, corresponding to disturbances traveling
faster than each of the  aforementioned waves. This is indeed the case. 

In general, a shock propagating through an MHD fluid produces a significant  difference in plasma properties on either side of the shock front.
The thickness of the front is determined by a balance between convective
and dissipative effects. However, dissipative effects in high temperature
plasmas are only  comparable to convective effects when the spatial gradients
in plasma variables become extremely large. Hence, MHD shocks in such plasmas tend to be {\em extremely narrow}, and are well-approximated
as  {\em discontinuous}\/ changes in plasma parameters. The MHD equations,
and Maxwell's equations, can be integrated across a shock
to give a set of {\em jump conditions}\/ which   relate  plasma properties on
each side of the shock front. If the shock is sufficiently narrow then these relations become {\em independent}\/ of its detailed structure. Let us derive the
jump conditions for a narrow, planar, steady-state, MHD shock.

Maxwell's equations, and the MHD equations, (\ref{e5.1a})--(\ref{e5.1d}),
can be written in the following convenient form:
\begin{eqnarray}\label{e5.239}
\nabla\cdot{\bf B} &=& 0,\\[0.5ex]
\frac{\partial {\bf B}}{\partial t} - \nabla\times ({\bf V}\times {\bf B}) &=& {\bf 0},\\[0.5ex]
\frac{\partial\rho}{\partial t} + \nabla\cdot(\rho\,{\bf V}) &=& 0,\\[0.5ex]
\frac{\partial (\rho\,{\bf V})}{\partial t} + \nabla\cdot{\bf T} &=&{\bf 0},\\[0.5ex]
\frac{\partial U}{\partial t} + \nabla\cdot {\bf u}&=&0,\label{e5.243}
\end{eqnarray}
where 
\begin{equation}
{\bf T} = \rho\,{\bf V}\,{\bf V} + \left(p+ \frac{B^2}{2\mu_0}\right){\bf I}- \frac{{\bf B}\,{\bf B}}{\mu_0}
\end{equation}
is the total ({\em i.e.}, including electromagnetic, as well as plasma,
contributions)
{\em stress tensor}, ${\bf I}$ the identity tensor,
\begin{equation}
U = \frac{1}{2}\,\rho\,V^2 + \frac{p}{\Gamma-1} + \frac{B^2}{2\mu_0}
\end{equation}
the total {\em energy density}, and 
\begin{equation}
{\bf u} = \left(\frac{1}{2}\,\rho\,V^2+ \frac{\Gamma}{\Gamma-1}\,p\right){\bf V}
+ \frac{{\bf B}\times ({\bf V}\times {\bf B})}{\mu_0}
\end{equation}
the total {\em energy flux density}.

Let us move into the {\em rest frame}\/ of the shock. Suppose that the
shock front coincides with the $y$-$z$ plane.
 Furthermore, let the regions
of the plasma upstream and downstream of the shock, which are termed
regions 1 and 2, respectively, be {\em spatially uniform}\/ and {\em non-time-varying}. It follows
that $\partial/\partial t = \partial/\partial y =\partial/\partial z=0$. Moreover,
$\partial/\partial x=0$, except in the immediate vicinity of the shock. 
Finally,  let the velocity
and magnetic fields  upstream and downstream of the shock
all lie in the $x$-$y$ plane. The situation under discussion is illustrated in Fig.~\ref{f29a}. Here, $\rho_1$, $p_1$, ${\bf V}_1$, and ${\bf B}_1$ are
the downstream mass density, pressure, velocity, and magnetic field,
respectively, whereas $\rho_2$, $p_2$, ${\bf V}_2$, and ${\bf B}_2$
are the corresponding upstream quantities.

\begin{figure}
\epsfysize=2.5in
\centerline{\epsffile{Chapter05/shock.eps}}
\caption{\em A planar shock.}\label{f29a}
\end{figure}

In the immediate vicinity of the shock, Eqs.~(\ref{e5.239})--(\ref{e5.243})  reduce to
\begin{eqnarray}
\frac{dB_{x}}{dx} &=& 0,\\[0.5ex]
\frac{d}{dx}(V_x\,B_y-V_y\,B_x)&=&0,\\[0.5ex]
\frac{d (\rho\, V_x)}{dx} &=&0,\\[0.5ex]
\frac{d T_{xx}}{dx} &=&0,\\[0.5ex]
\frac{d T_{xy}}{dx} &=&0,\\[0.5ex]
\frac{d u_x}{dx} &=& 0.
\end{eqnarray}
Integration  across the shock yields the desired jump conditions:
\begin{eqnarray}\label{e5.253}
[B_x]^2_1 &=&0,\\[0.5ex]
[V_x\,B_y-V_y\,B_x]^2_1&=&0,\label{e5.254}\\[0.5ex]
[\rho\,V_x]_1^2&=&0,\\[0.5ex]
[\rho\,V_x^{\,2}+p + B_y^{\,2}/2\mu_0]_1^2&=&0,\\[0.5ex]
[\rho\,V_x\,V_y - B_x\,B_y/\mu_0]^2_1&=&0,\\[0.5ex]
\left[\frac{1}{2}\,\rho\,V^2\,V_x + \frac{\Gamma}{\Gamma-1}\,p\,V_x
+ \frac{B_y\,(V_x\,B_y-V_y\,B_x)}{\mu_0}\right]^2_1&=&0,\label{e5.258}
\end{eqnarray}
where $[A]_1^2\equiv A_2-A_1$. These relations are often called the {\em Rankine-Hugoniot}\/
relations for MHD.  
Assuming that all of the upstream plasma parameters
are known, there are six unknown  parameters in the problem---namely,
$B_{x\,2}$, $B_{y\,2}$, $V_{x\,2}$, $V_{y\,2}$, $\rho_2$, and $p_2$. These six
unknowns are fully determined by the six jump conditions. Unfortunately, the
general case is very complicated. So, before tackling it, let us examine
a couple of relatively simple special cases.

\section{Parallel Shocks}\label{shydro}
The first special case is the so-called {\em parallel shock}\/ in which both the
upstream and downstream plasma flows are  parallel to the magnetic field, as well as  perpendicular to the shock
front. In other
words,
\begin{eqnarray}
{\bf V}_1 = (V_1,\,0,\,0),&\mbox{\hspace{1cm}}&{\bf V}_2 = (V_2,\,0,\,0),\\[0.5ex]
{\bf B}_1 = (B_1,\,0,\,0),&\mbox{\hspace{1cm}}&{\bf B}_2 = (B_2,\,0,\,0).
\end{eqnarray}
Substitution  into the general jump conditions (\ref{e5.253})--(\ref{e5.258}) yields
\begin{eqnarray}\label{e5.261}
\frac{B_2}{B_1} &=& 1,\\[0.5ex]
\frac{\rho_2}{\rho_1} &=& r,\\[0.5ex]
\frac{V_2}{V_1}&=& r^{-1},\\[0.5ex]
\frac{p_2}{p_1} &=& R,\label{e5.264}
\end{eqnarray}
with
\begin{eqnarray}\label{e5.265}
r& =& \frac{(\Gamma+1)\,M_1^{\,2}}{2+(\Gamma-1)\,M_1^{\,2}},\\[0.5ex]
R &=& 1+ \Gamma\,M_1^{\,2}\,(1-r^{-1})= \frac{(\Gamma+1)\,r-(\Gamma-1)}{(\Gamma+1)-(\Gamma-1)\,r}.\label{e5.266}
\end{eqnarray}
Here, $M_1= V_1/V_{S\,1}$, where $V_{S\,1}=(\Gamma\,p_1/\rho_1)^{1/2}$
is the upstream sound speed. Thus, the upstream flow  is supersonic if $M_1>1$, and subsonic if $M_1<1$. Incidentally, as is clear from
the above expressions, a
parallel shock is {\em unaffected}\/ by the presence of a magnetic field. In fact, this
type of shock is identical to that which occurs in neutral fluids, and is, 
therefore, usually called a {\em hydrodynamic shock}. 

It is easily seen from  Eqs.~(\ref{e5.261})--(\ref{e5.264}) that there is no shock ({\em i.e.}, no jump in plasma parameters across the shock front) when the upstream flow is exactly
sonic: {\em i.e.}, when $M_1=1$. In other words, $r=R=1$ when $M_1=1$. 
However, if $M_1\neq 1$ then the upstream
and downstream plasma parameters become  different ({\em i.e.}, $r\neq 1$, $R\neq 1$) and a true shock develops. 
In fact, it is easily demonstrated that
\begin{eqnarray}\label{e5.267}
\frac{\Gamma-1}{\Gamma+1} \leq & r &\leq \frac{\Gamma+1}{\Gamma-1},\\[0.5ex]
0\leq &R &\leq \infty,\\[0.5ex]
\frac{\Gamma-1}{2\,\Gamma}\leq  &M_1^{\,2}& \leq \infty.\label{e5.269}
\end{eqnarray}
Note that the upper and lower limits in the above inequalities are all attained simultaneously.

The previous discussion seems to imply that a parallel shock can be either compressive ({\em i.e.}, $r>1$) or expansive ({\em i.e.}, $r<1$). However, there is one
additional physics principle which needs to be factored into
our analysis---namely, the {\em second law of thermodynamics}. This law states that  the {\em entropy}\/ of a closed system can spontaneously increase,
but can never spontaneously decrease. Now, in general, the entropy per particle is different on either side of a hydrodynamic shock front. Accordingly, the second law of thermodynamics
mandates that the downstream entropy  must {\em exceed}\/ the upstream
entropy, so as to ensure that the shock generates a net increase, rather
than a net decrease, in the overall entropy of the system, as the plasma flows through it.

The (suitably normalized) entropy per particle of an ideal plasma takes the form [see Eq.~(\ref{entropy})]
\begin{equation}
S = \ln(p/\rho^{\Gamma}).
\end{equation}
Hence, the difference between the upstream and downstream entropies is
\begin{equation}
[S]^2_1 =\ln R - \Gamma\,\ln r.
\end{equation}
Now, using (\ref{e5.265}), 
\begin{equation}
r\,\frac{d[S]_1^2}{dr} = \frac{r}{R}\,\frac{dR}{dr}-\Gamma
= \frac{\Gamma\,(\Gamma^2-1)\,(r-1)^2}{[(\Gamma+1)\,r-(\Gamma-1)]\,[(\Gamma+1)-(\Gamma-1)\,r]}.
\end{equation}
Furthermore, it is easily seen from  Eqs.~(\ref{e5.267})--(\ref{e5.269}) that $d[S]_1^2/dr\geq 0$ in all situations of physical interest. However, $[S]_1^2=0$
when $r=1$, since, in this case, there is no discontinuity in plasma parameters across the shock front. We conclude that $[S]_1^2<0$ for $r<1$, and
$[S]_1^2>0$ for $r>1$. It follows that  the second law of thermodynamics 
requires hydrodynamic shocks to be {\em compressive}: {\em i.e.}, $r>1$. In other words, the
plasma density must always {\em increase}\/ when a shock front
is crossed in the direction of the relative plasma flow. It turns out that this
is a general rule which applies to all three types of MHD shock.

The upstream Mach number, $M_1$, is a good measure of shock strength:
{\em i.e.}, if $M_1=1$ then there is no shock, if $M_1-1 \ll 1$ then the shock is
weak, and if $M_1\gg 1$ then  the shock is strong. We can define an analogous downstream Mach number, $M_2=V_2/(\Gamma\,p_2/\rho_2)^{1/2}$. 
It is easily demonstrated from  the jump conditions that if $M_1>1$ then $M_2 < 1$. In other
words, in the shock rest frame, the shock is associated with an irreversible (since the
entropy suddenly increases) transition from supersonic to subsonic flow. Note that $r\equiv \rho_2/\rho_1\rightarrow (\Gamma+1)/(\Gamma-1)$,
whereas $R\equiv p_2/p_1\rightarrow\infty$, in the limit $M_1\rightarrow \infty$. In other words, as the shock strength increases, the compression ratio, $r$,
asymptotes to a finite value,  whereas the
pressure ratio, $P$, increases without limit. For a conventional
plasma with $\Gamma=5/3$, the limiting value of the compression ratio is 4: {\em i.e.}, the downstream density can never be more than four times the upstream density. We conclude that, in the strong shock limit, 
$M_1 \gg 1$, the
large jump in  the plasma pressure across the shock front must be
predominately  a consequence of a large jump in  the plasma {\em temperature}, rather than  the plasma density. In fact,  Eqs.~(\ref{e5.265})--(\ref{e5.266}) imply that
\begin{equation}
\frac{T_2}{T_1} \equiv \frac{R}{r}\rightarrow \frac{2\,\Gamma\,(\Gamma-1)\,M_1^{\,2}}{(\Gamma+1)^2}\gg 1
\end{equation}
as $M_1\rightarrow\infty$. Thus, a strong parallel, or hydrodynamic, shock
is associated with intense plasma heating.

As we have seen, the condition for the existence of a hydrodynamic
shock is $M_1> 1$, or $V_1 > V_{S\,1}$. In other words, in the
shock frame, the upstream plasma velocity, $V_1$, must be supersonic. 
However, by Galilean invariance, $V_1$ can also be interpreted as the
 {\em propagation velocity}\/ of the shock through an initially {\em stationary}\/ plasma. It follows that, in a stationary plasma, 
a parallel, or hydrodynamic, shock propagates along the magnetic field with a {\em supersonic}\/ velocity. 

\section{Perpendicular Shocks}\label{sperp}
The second special case is the so-called {\em perpendicular shock}\/ in which both the
upstream and downstream plasma flows are  perpendicular to the magnetic field, as well as   the shock
front. In other
words,
\begin{eqnarray}\label{e5.274}
{\bf V}_1 = (V_1,\,0,\,0),&\mbox{\hspace{1cm}}&{\bf V}_2 = (V_2,\,0,\,0),\\[0.5ex]\label{e5.275}
{\bf B}_1 = (0,\,B_1,\,0),&\mbox{\hspace{1cm}}&{\bf B}_2 = (0,\,B_2,\,0).
\end{eqnarray}
Substitution  into the general jump conditions (\ref{e5.253})--(\ref{e5.258}) yields
\begin{eqnarray}
\frac{B_2}{B_1} &=& r,\label{e5.276}\\[0.5ex]
\frac{\rho_2}{\rho_1} &=& r,\\[0.5ex]
\frac{V_2}{V_1}&=& r^{-1},\\[0.5ex]
\frac{p_2}{p_1} &=& R,
\end{eqnarray}
where
\begin{equation}\label{e5.280}
R = 1+ \Gamma\,M_1^{\,2}\,(1-r^{-1}) + \beta_1^{-1}\,(1-r^2),
\end{equation}
and $r$ is a real positive root of the quadratic
\begin{equation}\label{e5.281}
F(r) = 2\,(2-\Gamma)\,r^2+ \Gamma\,[2\,(1+\beta_1)+ (\Gamma-1)\,\beta_1\,M_1^{\,2}] \,r- \Gamma\,(\Gamma+1)\,\beta_1\,M_1^{\,2}=0.
\end{equation}
Here, $\beta_1= 2\mu_0\,p_1/B_1^{\,2}$. 

Now, if $r_1$ and $r_2$ are the two roots of Eq.~(\ref{e5.281}) then
\begin{equation}
r_1\,r_2= -\frac{\Gamma\,(\Gamma+1)\,\beta_1\,M_1^{\,2}}{2\,(2-\Gamma)}.
\end{equation}
Assuming that $\Gamma < 2$, we conclude that one of the roots is negative,
and, hence, that Eq.~(\ref{e5.281}) only possesses {\em one}\/ physical
solution: {\em i.e.}, there is only one type of MHD shock which is
consistent with Eqs.~(\ref{e5.274}) and (\ref{e5.275}). Now, it is easily
demonstrated that $F(0)<0$ and $F(\Gamma+1/\Gamma-1)>0$. Hence, the
physical root lies between $r=0$ and $r=(\Gamma+1)/(\Gamma-1)$. 

Using similar analysis to that employed in the previous subsection, it
is easily  demonstrated that the second law of thermodynamics requires a
perpendicular shock to be compressive: {\em i.e.}, $r>1$. It follows that a physical solution
is only obtained when $F(1)<0$, which reduces to
\begin{equation}
M_1^{\,2} > 1 + \frac{2}{\Gamma\,\beta_1}.
\end{equation}
This condition can also be written
\begin{equation}
V_1^{\,2} > V_{S\,1}^{\,2} + V_{A\,1}^{\,2},
\end{equation}
where $V_{A\,1}=B_1/(\mu_0\,\rho_1)^{1/2}$ is the upstream
Alfv\'{e}n velocity. Now,  $V_{+\,1} = (V_{S\,1}^{\,2} + V_{A\,1}^{\,2})^{1/2}$ can be recognized as the velocity of a {\em fast wave}\/ propagating
perpendicular to the magnetic field---see Sect.~\ref{s5.4}. Thus, the
condition for the existence of a perpendicular shock is that the relative
upstream plasma velocity must be {\em greater}\/  than  the upstream fast wave velocity. Incidentally, it is easily
demonstrated that if this is the case then  the downstream plasma velocity is {\em less}\/ than the downstream
fast wave velocity.    We can also deduce  that, in a stationary plasma, a
perpendicular shock propagates across the magnetic field with
a velocity which exceeds the fast wave velocity.

In the strong shock limit, $M_1\gg 1$, Eqs.~(\ref{e5.280}) and (\ref{e5.281}) become identical to Eqs.~(\ref{e5.265}) and (\ref{e5.266}).
Hence, a strong  perpendicular shock is very similar to a strong hydrodynamic shock (except that the former shock
propagates perpendicular, whereas the latter
shock propagates parallel,  to the magnetic field). In particular, just like a hydrodynamic shock, a
perpendicular shock cannot
compress the density by more than a factor $(\Gamma+1)/(\Gamma-1)$. However, according to
Eq.~(\ref{e5.276}), a perpendicular shock compresses the magnetic field by the same
factor that it compresses the plasma density. It follows that there is
also an upper limit to the factor by which a perpendicular shock can compress the magnetic field.

\section{Oblique Shocks}
Let us now consider the general case in which the plasma velocities and
the magnetic fields on each side of the shock are neither parallel nor
perpendicular to the shock front. It is convenient to transform
into the so-called {\em de Hoffmann-Teller}\/ frame in which $|{\bf V}_1\times {\bf B}_1|=0$, or 
\begin{equation}\label{e2.285}
V_{x\,1}\,B_{y\,1} - V_{y\,1}\,B_{x\,1} = 0.
\end{equation}
In other words, it is convenient to transform to a frame which moves at the local ${\bf E}\times {\bf B}$
velocity of the plasma.
It immediately follows from the jump condition (\ref{e5.254}) that
\begin{equation}\label{e2.286}
V_{x\,2}\,B_{y\,2} - V_{y\,2}\,B_{x\,2} = 0,
\end{equation}
or $|{\bf V}_2\times {\bf B}_2|=0$. Thus,  in the de Hoffmann-Teller frame,  the upstream plasma
flow is {\em parallel}\/ to the upstream magnetic field, and the downstream plasma
flow is also parallel to the downstream magnetic field. Furthermore, the magnetic contribution to the jump
condition (\ref{e5.258}) becomes identically zero, which is a considerable simplification.

Equations~(\ref{e2.285}) and (\ref{e2.286}) can be combined with the
general jump conditions (\ref{e5.253})--(\ref{e5.258})
to give
\begin{eqnarray}
\frac{\rho_2}{\rho_1}&=& r,\\[0.5ex]
\frac{B_{x\,2}}{B_{x\,1}}&=& 1,\\[0.5ex]
\frac{B_{y\,2}}{B_{y\,1}}&=& 
r\left(\frac{v_{1}^{\,2} - \cos^2\theta_1\,V_{A\,1}^{\,2}}{v_{1}^{\,2}-r\,\cos^2\theta_1\,V_{A\,1}^{\,2}}\right),\label{e5.289}\\[0.5ex]
\frac{V_{x\,2}}{V_{x\,1}}&=& \frac{1}{r},\\[0.5ex]
\frac{V_{y\,2}}{V_{y\,1}}&=& 
\frac{v_{1}^{\,2} -\cos^2\theta_1\, V_{A\,1}^{\,2}}{v_{1}^{\,2}-r\,\cos^2\theta_1\,V_{A\,1}^{\,2}},\label{e5.291}\\[0.5ex]
\frac{p_2}{p_1}&=& 1  + \frac{\Gamma\,v_1^{\,2}\,(r-1)}{V_{S\,1}^{\,2}\,r}
\left[1 - \frac{r\,V_{A\,1}^{\,2}\,[(r+1)\,v_1^{\,2}-2\,r\,V_{A\,1}^{\,2}\,\cos^2\theta_1]}
{2\,(v_1^{\,2}-r\,V_{A\,1}^{\,2}\,\cos^2\theta_1)^2}\right].
\end{eqnarray}
where $v_1= V_{x\,1} = V_1\,\cos\theta_1$ is the component of the upstream velocity normal to the
shock front, 
and $\theta_1$ is the angle subtended between the upstream plasma flow and the
shock front normal. Finally, given the compression ratio, $r$,  the square of the normal
upstream velocity, $v_1^{\,2}$, is a real root of a cubic equation
known as the {\em shock adiabatic}:
\begin{eqnarray}
0&=&(v_{1}^{\,2}-r\,\cos^2\theta_1\,V_{A\,1}^{\,2})^2\left\{
\left[(\Gamma+1)-(\Gamma-1)\,r\right]
v_{1}^{\,2}- 2\,r\,V_{S\,1}^{\,2}\right\}\\[0.5ex]
&&-r\,\sin^2\theta_1\,v_{1}^{\,2}\,V_{A\,1}^{\,2}\left\{
\left[\Gamma+ (2-\Gamma)\,r\right]v_{1}^{\,2}
-\left[(\Gamma+1)-(\Gamma-1)\,r\right]r\,\cos^2\theta_1\,V_{A\,1}^{\,2}\right]\}.\nonumber
\end{eqnarray}
As before, the second law of thermodynamics mandates that $r>1$. 

Let us first consider the weak shock limit $r\rightarrow 1$. In this case, it is easily seen that the three roots of the
shock adiabatic reduce to
\begin{eqnarray}
v_1^{\,2} &=&V_{-\,1}^{\,2}\equiv \frac{V_{A\,1}^{\,2}+V_{S\,1}^{\,2}- [(V_{A\,1}+V_{S\,1})^2
-4\,\cos^2\theta_1\,V_{S\,1}^{\,2}\,V_{A\,1}^{\,2}]^{1/2}}{2},\\[0.5ex]
v_1^{\,2} &=& \cos^2\theta_1\,V_{A\,1}^{\,2},\\[0.5ex]
v_1^{\,2} &=&V_{+\,1}^{\,2}\equiv \frac{V_{A\,1}^{\,2}+V_{S\,1}^{\,2} + [(V_{A\,1}+V_{S\,1})^2
-4\,\cos^2\theta_1\,V_{S\,1}^{\,2}\,V_{A\,1}^{\,2}]^{1/2}}{2}.
\end{eqnarray}
However, from Sect.~\ref{s5.4}, we recognize these velocities as belonging to slow, intermediate
(or Shear-Alfv\'{e}n), and fast waves, respectively, propagating in the normal direction to the
shock front. We conclude that slow, intermediate, and fast MHD shocks degenerate into the
associated MHD waves in the limit of small shock amplitude. Conversely, we can think of
the various MHD shocks as {\em nonlinear}\/ versions of the associated MHD waves. Now it is easily demonstrated that
\begin{equation}
V_{+\,1}> \cos\theta_1\,V_{A\,1}> V_{-\,1}.
\end{equation}
In other words, a fast wave travels faster than an intermediate wave, which travels faster than a slow
wave. It is reasonable to suppose that the same is true of the associated MHD
shocks, at least at relatively low shock strength.
It follows from Eq.~(\ref{e5.289}) that $B_{y\,2}>B_{y\,1}$ for a fast shock,
whereas $B_{y\,2}<B_{y\,1}$ for a slow  shock. For the case of an intermediate shock, we
can show, after a little algebra, that $B_{y\,2}\rightarrow -B_{y\,1}$ in the limit
$r\rightarrow 1$.  We conclude that (in the de Hoffmann-Teller frame) {\em fast}\/ shocks refract the magnetic field and plasma
flow (recall that they are parallel in our adopted frame of the reference) {\em away}\/ from
the normal to the shock front, whereas {\em slow}\/ shocks refract these quantities {\em toward}\/
the normal. Moreover, the tangential magnetic field and plasma flow generally {\em reverse}\/
across an {\em intermediate}\/ shock front. This is illustrated in Fig.~\ref{f30a}.

\begin{figure}
\epsfysize=2.5in
\centerline{\epsffile{Chapter05/shock3.eps}}
\caption{\em Characteristic plasma flow patterns across the three different types of MHD shock
in the shock rest frame.}\label{f30a}
\end{figure}

When $r$ is slightly larger than unity it is easily  demonstrated that the conditions for the
existence of a  slow, intermediate, and fast shock are
$v_1> V_{-\,1}$, $v_1> \cos\theta_1\,V_{A\,1}$,  and $v_1> V_{+\,1}$, respectively.

Let us now consider the strong shock limit, $v_1^{\,2}\gg 1$. In this case, the shock
adiabatic yields $r\rightarrow r_m=(\Gamma+1)/(\Gamma-1)$, and
\begin{equation}
v_1^{\,2} \simeq \frac{r_m}{\Gamma-1}\,\frac{2\,V_{S\,1}^{\,2}+\sin^2\theta_1\,[\Gamma
+ (2-\Gamma)\,r_m]\,V_{A\,1}^{\,2}}{r_m-r}.
\end{equation}
There are no other real roots. The above root is clearly a type of
fast shock. The fact that there is only one real  root suggests that there exists a critical
shock strength above which the slow and intermediate shock solutions  cease to exist. (In fact,
they merge and annihilate one another.)
In other words, there is a limit to the strength of a slow or an intermediate shock.
 On the other hand, there is no limit to the strength of a fast shock. Note, however, that 
  the plasma density and tangential
 magnetic field cannot be compressed by more than a
 factor $(\Gamma+1)/(\Gamma-1)$  by any type of MHD shock.
 
Consider the special case $\theta_1=0$ in which both the plasma flow and the
 magnetic field are normal to the shock front. In this case, the three roots of the shock adiabatic are
 \begin{eqnarray}
 v_1^{\,2} &=& \frac{2\,r\,V_{S\,1}^{\,2}}{(\Gamma+1)-(\Gamma-1)\,r},\\[0.5ex]
 v_1^{\,2} &=& r\,V_{A\,1}^{\,2},\\[0.5ex]
 v_1^{\,2} &=& r\,V_{A\,1}^{\,2}.
 \end{eqnarray}
 We recognize the first of these roots as the hydrodynamic shock discussed in Sect.~\ref{shydro}---{\em cf.} Eq.~(\ref{e5.265}).
 This shock is classified as a slow shock when $V_{S\,1}<V_{A\,1}$, and as a fast shock
 when $V_{S\,1}> V_{A\,1}$. The other two roots are identical, and correspond to
 shocks which propagate at the velocity $v_1 =\sqrt{r}\, V_{A\,1}$ and  ``switch-on" the tangential
 components of the plasma flow and the magnetic field: {\em i.e.}, it can be seen from
 Eqs.~(\ref{e5.289}) and (\ref{e5.291}) that $V_{y\,1}=B_{y\,1} =0$ whilst
 $V_{y\,2}\neq 0$ and $B_{y\,2}\neq 0$ for these types of shock. Incidentally, it is also
 possible to have a ``switch-off'' shock which eliminates the tangential components
 of the plasma flow and the magnetic field. According to Eqs.~(\ref{e5.289}) and (\ref{e5.291}),
 such a shock propagates at the velocity $v_1=\cos\theta_1\,V_{A\,1}$. Switch-on and
 switch-off shocks are illustrated in Fig.~\ref{f31a}.
 
 \begin{figure}
\epsfysize=2.5in
\centerline{\epsffile{Chapter05/onoff.eps}}
\caption{\em Characteristic plasma flow patterns across switch-on and switch-off shocks
in the shock rest frame.}\label{f31a}
\end{figure}

Let us, finally, consider the special case $\theta=\pi/2$. As is easily demonstrated, the three roots of the
shock adiabatic are
 \begin{eqnarray}
 v_1^{\,2} &=& r \left(\frac{2\,V_{S\,1}^{\,2} + [\Gamma+(2-\Gamma)\,r]\,V_{A\,1}^{\,2}}
 {(\Gamma+1)-(\Gamma-1)\,r}\right)
 ,\\[0.5ex]
 v_1^{\,2} &=& 0,\\[0.5ex]
 v_1^{\,2} &=& 0.
 \end{eqnarray}
 The first of these roots is clearly  a fast shock, and is  identical to the perpendicular
 shock discussed in Sect.~\ref{sperp}, except that there is no plasma flow across the shock
 front in this case.  The fact that the two other  roots are zero indicates
 that, like the corresponding MHD waves, slow and intermediate MHD shocks do
 not propagate perpendicular to the magnetic field.
 
 MHD shocks have been observed in a large variety of situations. For instance, shocks are
 known to be formed by supernova explosions, by strong stellar winds, by solar flares, and
 by the solar wind upstream of planetary magnetospheres.\footnote{D.A.~Gurnett, and
 A.~Bhattacharjee, {\em Introduction to Plasma
 Physics}, Cambridge University Press, Cambridge UK, 2005.}
 
