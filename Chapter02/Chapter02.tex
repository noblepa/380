\chapter {Charged Particle Motion}\label{s2}
\section{Introduction}
All descriptions of plasma behaviour are based, ultimately, on the motions
of the constituent particles. For the case of an unmagnetized plasma, the
motions are fairly trivial, since the constituent particles move 
essentially in
straight lines between collisions. The motions are also trivial in
a magnetized plasma where the collision frequency $\nu$ greatly exceeds the
gyrofrequency ${\Omega}$: in this case, the  particles are scattered
after executing only a small fraction  of a gyro-orbit, and, therefore, still move
essentially in straight lines between collisions. The situation of primary
interest in this section is that of a {\em collisionless}\/
({\em i.e.}, $\nu\ll {\Omega}$), {\em magnetized}\/ plasma, where the gyroradius
$\rho$ is much smaller than the typical variation length-scale $L$ of the 
${\bf E}$ and ${\bf B}$ fields, and the gyroperiod ${\Omega}^{-1}$ is
much less than the typical time-scale $\tau$ on which these fields change.
In such a plasma, we expect the motion of the constituent particles to consist
of a rapid  gyration perpendicular to
 magnetic field-lines, combined with free-streaming parallel
to the field-lines. We are particularly interested in calculating how
this motion is affected by the spatial and temporal {\em gradients}\/ in the 
${\bf E}$ and ${\bf B}$ fields. In general, the motion of charged particles
in spatially and temporally {\em non-uniform}\/ electromagnetic
fields is extremely complicated: however, 
we hope to considerably simplify this motion by exploiting the
assumed smallness of the parameters $\rho/L$ and $({\Omega}\,\tau)^{-1}$. 
What we are really trying to understand, in this section, is how the
{\em magnetic confinement}\/ of an essentially collisionless plasma works
at an individual particle level. 
Note that the type of collisionless, magnetized plasma considered in this
section occurs primarily in magnetic fusion and space plasma physics
contexts. In fact, in the following
we shall be studying  methods of analysis first developed by fusion
physicists, and 
illustrating these methods primarily 
by investigating problems of interest in magnetospheric physics. 

\section{Motion in Uniform Fields}
Let us, first of all, consider the motion of charged particles
in spatially and temporally {\em uniform}\/ electromagnetic fields. 
The equation of motion of an individual particle takes the form
\begin{equation}\label{e2.1}
m\,\frac{d{\bf v}}{dt} = e\,({\bf E} + {\bf v}\times {\bf B}).
\end{equation}
The component of this equation parallel to the magnetic field,
\begin{equation}
\frac{d v_\parallel} {dt} = \frac{e}{m} \,E_\parallel,
\end{equation}
predicts uniform acceleration along magnetic field-lines. Consequently,
plasmas near equilibrium generally have either small or vanishing $E_\parallel$. 

As can easily be verified by substitution, the perpendicular component of
Eq. (\ref{e2.1}) yields
\begin{equation}
{\bf v}_\perp = \frac{{\bf E}\times{\bf B}}{B^2} + \rho\,{\Omega}
\left[ {\bf e}_1 \,\sin({\Omega}\, t +\gamma_0) + {\bf e}_2\,
 \cos({\Omega}\, t +\gamma_0)\right],
\end{equation}
where ${\Omega}=eB/m$ is the gyrofrequency, $\rho$ is the gyroradius, 
${\bf e}_1$ and ${\bf e}_2$
are unit vectors such that (${\bf e}_1$, ${\bf e}_2$, ${\bf B}$) form a
right-handed, mutually orthogonal set, and $\gamma_0$ is the initial
gyrophase of the particle. The motion consists of gyration around the
magnetic field at frequency ${\Omega}$, superimposed on a
steady drift at velocity
\begin{equation}\label{e2.4}
{\bf v}_E = \frac{{\bf E}\times{\bf B}}{B^2}.
\end{equation}
This drift, which is termed the {\em E-cross-B drift}\/ by plasma physicists,
is {\em identical}\/ for all plasma species, and can be eliminated entirely by
transforming to a new inertial frame in which ${\bf E}_\perp= {\bf 0}$. 
This frame, which moves with
velocity ${\bf v}_E$ with respect to the old frame, can properly be regarded as the {\em rest frame}\/ of the plasma. 

We complete the solution by integrating the velocity to find the particle
position:
\begin{equation}
{\bf r}(t) = {\bf R}(t) + \brho(t),
\end{equation}
where
\begin{equation}
\brho(t) = \rho \,[-{\bf e}_1\,\cos({\Omega}\, t+\gamma_0)
+{\bf e}_2\,\sin({\Omega}\,t + \gamma_0)],
\end{equation}
and
\begin{equation}
{\bf R}(t)=\left(v_{0\,\parallel } \,t + \frac{e}{m}\, E_\parallel \,\frac{t^2}{2}\right)
{\bf b} + {\bf v}_E \,\,t.
\end{equation}
Here, ${\bf b} \equiv {\bf B}/B$. Of course, the trajectory of the particle
describes a {\em spiral}. The gyrocentre ${\bf R}$ of this spiral, termed
the {\em guiding centre}\/ by plasma physicists, drifts across the magnetic
field with velocity ${\bf v}_E$, and also accelerates along the
field at a rate determined by the parallel electric field.

The concept of a guiding centre gives us a clue as to how to proceed. Perhaps,
when analyzing charged particle motion  in 
{\em non-uniform}\/  electromagnetic
fields, we can somehow neglect the rapid, and relatively uninteresting, gyromotion,
and focus, instead, on the far slower
motion of the guiding centre? Clearly, what we
need to do in order to achieve this goal is to somehow {\em average}\/ the equation of motion over gyrophase, so
as to obtain a reduced equation of motion for the guiding centre. 

\section{Method of Averaging}
In many dynamical problems, the motion consists of a rapid oscillation superimposed on
a slow secular drift. For such problems, the most efficient approach
is to describe the evolution in terms of the average values of the dynamical
variables. The method outlined below is adapted from a classic
paper by Morozov and Solov'ev.\footnote{A.I.~Morozov, and L.S.~Solev'ev, 
{\em Motion of Charged Particles in Electromagnetic Fields}, in
{\em Reviews of Plasma Physics}, Vol.~2 (Consultants Bureau, New York NY, 1966).}

Consider the equation of motion
\begin{equation}\label{e2.8}
\frac{d{\bf z}}{dt} = {\bf f}({\bf z}, t, \tau),
\end{equation}
where ${\bf f}$ is a periodic function of its last argument, with
period $2\pi$, and
\begin{equation}\label{e2.9}
\tau = t/\epsilon.
\end{equation}
Here, the small parameter $\epsilon$ characterizes the separation between the
short oscillation period $\tau$ and the time-scale $t$ for the slow secular evolution
of the ``position'' ${\bf z}$.

The basic idea of the averaging method is to treat $t$ and $\tau$ as {\em distinct}\/
independent variables, and to look for solutions of the form ${\bf z}(t,\tau)$
which are {\em periodic}\/ in $\tau$. Thus, we replace Eq.~(\ref{e2.8}) by
\begin{equation}\label{e2.10}
\frac{\partial{\bf z}}{\partial t} +\frac{1}{\epsilon}\frac{\partial {\bf z}}
{\partial \tau} = {\bf f}({\bf z}, t, \tau),
\end{equation}
and reserve Eq.~(\ref{e2.9}) for substitution in the final result. The
indeterminacy introduced by increasing the number of variables is lifted by
the requirement of periodicity in $\tau$. All of the secular drifts
are thereby attributed to the $t$-variable, whilst the oscillations are 
described entirely by the $\tau$-variable. 

Let us denote the $\tau$-average of ${\bf z}$ by ${\bf Z}$, and seek a
change of variables of the form 
\begin{equation}
{\bf z}(t,\tau)  = {\bf Z}(t) +\epsilon\,\bzeta({\bf Z}, t, \tau).
\end{equation}
Here, $\bzeta$ is a periodic function of $\tau$ with vanishing mean.
Thus,
\begin{equation}
\langle \bzeta({\bf Z}, t, \tau)\rangle\equiv
\frac{1}{2\pi}\oint \bzeta({\bf Z}, t, \tau)\,d\tau = 0,
\end{equation}
where $\oint$ denotes the integral over a full period in $\tau$. 

The evolution of ${\bf Z}$ is determined by substituting the
expansions
\begin{eqnarray}
\bzeta &=& \bzeta_0({\bf Z}, t, \tau) + \epsilon\,
\bzeta_1({\bf Z}, t, \tau) + \epsilon^2\,\bzeta_2({\bf Z}, t, \tau) +
\cdots,\\[0.5ex]
\frac{d{\bf Z}}{dt} &=& {\bf F}_0({\bf Z}, t) + \epsilon\,
{\bf F}_1({\bf Z}, t) + \epsilon^2\,{\bf F}_2({\bf Z}, t) +
\cdots,
\end{eqnarray}
into the equation of motion (\ref{e2.10}), and solving order by order in $\epsilon$. 

To lowest order, we obtain
\begin{equation}\label{e2.14}
{\bf F}_0({\bf Z}, t) +\frac{\partial\bzeta_0}{\partial\tau}
= {\bf f} ({\bf Z}, t, \tau).
\end{equation}
The solubility condition for this equation is
\begin{equation}\label{e2.15}
{\bf F}_0({\bf Z}, t) =\langle {\bf f}({\bf Z}, t,\tau)\rangle.
\end{equation}
Integrating the oscillating component of Eq.~(\ref{e2.14}) yields
\begin{equation}
\bzeta_0 ({\bf Z}, t,\tau) = \int_0^\tau\left(
{\bf f} - \langle{\bf f}\rangle\right)\,d\tau'.
\end{equation}

To first order, we obtain
\begin{equation}
{\bf F}_1 +\frac{\partial\bzeta_0}{\partial t} +
{\bf F}_0\cdot \nabla \bzeta_0 + \frac{\partial\zeta_1}{\partial\tau}
 = \bzeta_0\cdot
 \nabla {\bf f}.
\end{equation}
The solubility condition for this equation yields
\begin{equation}\label{e2.18}
{\bf F}_1 =\langle \bzeta_0\cdot\nabla{\bf f}\rangle.
\end{equation}

The final result is obtained by combining Eqs.~(\ref{e2.15}) and (\ref{e2.18}):
\begin{equation}
\frac{d{\bf Z}}{dt} = \langle {\bf f}\rangle + \epsilon\, \langle \bzeta_0\cdot
\nabla {\bf f}\rangle + O(\epsilon^2).
\end{equation}
Note that ${\bf f} = {\bf f}({\bf Z},t)$ in the above equation. 
Evidently, the secular motion of the ``guiding centre'' position ${\bf Z}$ 
is determined to lowest order by the average of the  ``force'' ${\bf f}$, and to
next order by the correlation between the oscillation in the ``position''
${\bf z}$ and the
oscillation in the spatial gradient of the ``force.'' 

\section{Guiding Centre Motion}
Consider the motion of a charged particle in the limit in which
the electromagnetic fields experienced
by the  particle do not vary much in a gyroperiod: {\em i.e.}, 
\begin{eqnarray}
\rho\,|\nabla{\bf B}| &\ll & B, \\[0.5ex]
\frac{1}{{\Omega}}\frac{\partial B}{\partial t} & \ll & B.
\end{eqnarray}
The electric force is assumed to be comparable to the magnetic force. 
To keep track of the order of the various quantities, we introduce the parameter
$\epsilon$ as a book-keeping device, and make the substitution
$\rho\rightarrow \epsilon\,\rho$, as well as $({\bf E}, {\bf B}, {\Omega})
\rightarrow \epsilon^{-1}({\bf E}, {\bf B}, {\Omega})$. The parameter
$\epsilon$ is set to unity in the final answer.


In order to make use of the technique described in the previous section, we write the
dynamical equations in first-order differential form,
\begin{eqnarray}\label{e2.21a}
\frac{d{\bf r}}{dt} &=& {\bf v},\\[0.5ex]
\frac{d{\bf v}}{dt} &=& \frac{e}{\epsilon\,m} \,({\bf E} + {\bf v}\times
{\bf B}),\label{e2.21b}
\end{eqnarray}
and seek a change of variables,
\begin{eqnarray}\label{e2.22a}
{\bf r} &=& {\bf R} + \epsilon\,\brho({\bf R}, {\bf U}, t, \gamma),
\\[0.5ex]
{\bf v} &=& {\bf U} + {\bf u} ({\bf R}, {\bf U}, t, \gamma),
\end{eqnarray}
such that the new guiding centre variables ${\bf R}$ and ${\bf U}$ are
free of oscillations along the particle trajectory. Here, $\gamma$ is a
new independent variable describing the phase of the gyrating particle. 
The functions $\brho$ and ${\bf u}$ represent the gyration radius
and velocity, respectively. We require periodicity of these functions
with respect to their last argument, with period $2\pi$, and with vanishing mean:
\begin{equation}\label{e2.23}
\langle \brho \rangle = \langle {\bf u} \rangle = {\bf 0}.
\end{equation}
Here, the angular brackets refer to the average over a period in $\gamma$. 

The equation of motion is used to determine the coefficients in the expansion
of $\brho$ and ${\bf u}$:
\begin{eqnarray}
\brho &=& \brho_0({\bf R}, {\bf U}, t, \gamma) +
\epsilon\,\brho_1({\bf R}, {\bf U}, t, \gamma) + \cdots, \\[0.5ex]
{\bf u} &=& {\bf u}_0({\bf R}, {\bf U}, t, \gamma) +
\epsilon\,{\bf u}_1({\bf R}, {\bf U}, t, \gamma) + \cdots.
\end{eqnarray}
The dynamical equation for the gyrophase is likewise expanded,
assuming that $d\gamma/dt \simeq {\Omega} = O(\epsilon^{-1})$, 
\begin{equation}
\frac{d\gamma}{dt} = \epsilon^{-1}\,\omega_{-1}({\bf R}, {\bf U}, t) 
+ \omega_0({\bf R}, {\bf U}, t) + \cdots.
\end{equation}
In the following, we suppress the subscripts on all quantities except
the guiding centre velocity ${\bf U}$, since this is the only quantity for
which the first-order corrections are calculated.

To each order in $\epsilon$, the evolution of the guiding centre position
${\bf R}$ and velocity ${\bf U}$
 are determined by
the solubility conditions for the equations of motion (\ref{e2.21a})--(\ref{e2.21b}) when
expanded to that order.
The oscillating components of the equations of motion determine the
evolution of the gyrophase. Note that the velocity equation
(\ref{e2.21a}) is {\em linear}. It follows that, to all orders in $\epsilon$, its solubility condition is simply
\begin{equation}
\frac{d{\bf R}}{dt} = {\bf U}.
\end{equation}

To lowest order [{\em i.e.}, $O(\epsilon^{-1})$], the momentum equation
(\ref{e2.21b}) yields
\begin{equation}\label{e2.27}
\omega\,\frac{\partial{\bf u}}{\partial\gamma} - {\Omega}\,{\bf u}\times{\bf 
b} = \frac{e}{m}\,\left({\bf E} + {\bf U}_0\times{\bf B}\right).
\end{equation}
The solubility condition ({\em i.e.}, the gyrophase average) is
\begin{equation}
{\bf E} + {\bf U}_0\times {\bf B} = {\bf 0}.
\end{equation}
This immediately implies that
\begin{equation}
E_\parallel \equiv {\bf E}\cdot {\bf b} \sim \epsilon\,E.
\end{equation}
Clearly, the rapid acceleration caused by a large parallel electric
field would invalidate the ordering assumptions used in this calculation. 
Solving for ${\bf U}_0$, we obtain
\begin{equation}\label{e2.30}
{\bf U}_0 = U_{0\,\parallel} \,{\bf b} + {\bf v}_E,
\end{equation}
where all quantities are evaluated at the guiding-centre position ${\bf R}$. The
perpendicular component of the velocity, ${\bf v}_E$, has the
same form (\ref{e2.4}) as for uniform fields. Note that the parallel velocity is
undetermined at this order.

The integral of the oscillating component of Eq.~(\ref{e2.27}) yields
\begin{equation}
{\bf u} = {\bf c} + u_\perp \left[ {\bf e}_1\,\sin \,({\Omega}\,\gamma/\omega)
+{\bf e}_2\,\cos\,({\Omega}\,\gamma/\omega)\right],
\end{equation}
where ${\bf c}$ is a constant vector, and ${\bf e}_1$ and ${\bf e}_2$ are again
mutually
orthogonal unit vectors perpendicular to ${\bf b}$. All quantities in the
above equation are functions of ${\bf R}$, ${\bf U}$, and $t$. 
 The periodicity
constraint, plus Eq.~(\ref{e2.23}), require that $\omega={\Omega}({\bf R}, t)$
and ${\bf c} = {\bf 0}$. The gyration velocity is thus
\begin{equation}
{\bf u} = u_\perp \,\left({\bf e}_1\,\sin\gamma + {\bf e}_2\,\cos\gamma\right),
\end{equation}
and the gyrophase is given by
\begin{equation}\label{e2.33}
\gamma = \gamma_0 + {\Omega}\,t,
\end{equation}
where $\gamma_0$ is the initial phase. Note that the amplitude $u_\perp$
of the gyration velocity is undetermined at this order.

The lowest order oscillating component of the velocity equation (\ref{e2.21a}) yields
\begin{equation}\label{e2.34}
{\Omega}\,\frac{\partial\brho}{\partial \gamma} = {\bf u}.
\end{equation}
This is easily integrated to give
\begin{equation}
\brho = \rho\,(-{\bf e}_1\,\cos\gamma+{\bf e}_2\,\sin\gamma),
\end{equation}
where $\rho = u_\perp/{\Omega}$. It follows that
\begin{equation}\label{e2.36}
{\bf u} = {\Omega}\, \brho\times {\bf b}.
\end{equation}

The gyrophase average of the first-order [{\em i.e.}, $O(\epsilon^0)$]
momentum equation (\ref{e2.21b}) reduces to
\begin{equation}
\frac{d{\bf U}_0}{dt} = \frac{e}{m}\,\left[ E_\parallel\,{\bf b}
+ {\bf U}_1\times {\bf B} +\langle{\bf u}\times(\brho\cdot\nabla)
\,{\bf B}\rangle\right].
\end{equation}
Note that all quantities in the above equation are functions of the
guiding centre position ${\bf R}$, rather than the instantaneous particle
position ${\bf r}$. 
In order to evaluate the last term, we make the substitution ${\bf u} = {\Omega}\, \brho\times {\bf b}$ and calculate
\begin{eqnarray}
\langle (\brho\times{\bf b})\times (\brho\cdot\nabla)\,{\bf B} \rangle
&=& {\bf b}\,\langle \brho \cdot (\brho\cdot\nabla)\,{\bf B} \rangle
- \langle \brho\,\,{\bf b}\cdot( \brho\cdot\nabla)\,{\bf B}\rangle\nonumber
\\[0.5ex]
&=& {\bf b}\,\langle \brho \cdot (\brho\cdot\nabla)\,{\bf B} \rangle
- \langle \brho\,( \brho\cdot\nabla B)\rangle.
\end{eqnarray}
The averages are specified  by
\begin{equation}
\langle \brho\,\brho\rangle = \frac{u_\perp^{~2}}{2\,{\Omega}^2}
\,({\bf I} - {\bf b}{\bf b}),
\end{equation}
where ${\bf I}$ is the identity tensor. Thus, making use of
${\bf I}\!:\!\nabla{\bf B} = \nabla\!\cdot\!{\bf B} = 0$, it follows
that
\begin{equation}
-e \,\langle {\bf u} \times (\brho\cdot\nabla )\,{\bf B}\rangle =
\frac{m\,u_\perp^{~2}}{2\,B} \,\nabla B.
\end{equation}
This quantity is the secular component of the gyration induced fluctuations in the magnetic
force acting on the particle. 

The coefficient of $\nabla B$ in the above equation,
\begin{equation}
\mu = \frac{m\,u_\perp^{~2}}{2\,B},
\end{equation}
plays a central role in the theory of magnetized particle motion. We can
interpret this coefficient as a {\em magnetic moment}\/ by drawing an analogy
between a gyrating particle and a current loop. The (vector) magnetic moment of
a current loop is simply
\begin{equation}
\bmu = I\,A\,{\bf n},
\end{equation}
where $I$ is the current, $A$ the area of the loop, and ${\bf n}$ the
unit normal to the surface of the loop. For a circular loop of
radius $\rho = u_\perp/{\Omega}$, lying in the
plane perpendicular to ${\bf b}$, and carrying the current $e\,{\Omega}/2\pi$,
we find
\begin{equation}
\bmu = I\,\pi\,\rho^2\,{\bf b} = \frac{m\,u_\perp^{~2}}{2\,B}\,{\bf b}.
\end{equation}
We shall  demonstrate later on that the (scalar) magnetic moment $\mu$ is a
{\em constant}\/ of the particle motion. Thus, the guiding centre behaves
exactly  like a particle with a conserved magnetic moment $\mu$ which
is always aligned with the magnetic field. 

The first-order guiding centre equation of motion reduces to
\begin{equation}\label{e2.44}
m \,\frac{d{\bf U}_0}{dt} = e\,E_\parallel\,{\bf b} + e\,{\bf U}_1\times {\bf B}
-\mu\,\nabla B.
\end{equation}
The component of this equation along the magnetic field determines the evolution
of the parallel guiding centre velocity:
\begin{equation}\label{e2.45}
m\,\frac{dU_{0\,\parallel}}{dt} = e\,E_\parallel - \bmu\cdot\nabla B
- m\,{\bf b}\cdot\frac{d{\bf v}_E}{dt}.
\end{equation}
Here, use has been made of Eq.~(\ref{e2.30}) and ${\bf b}\cdot d{\bf b}/dt=0$. 
The component of Eq.~(\ref{e2.44}) perpendicular to the magnetic field determines the
first-order perpendicular drift velocity:
\begin{equation}\label{e2.46}
{\bf U}_{1\,\perp} = \frac{{\bf b}}{{\Omega}} \times\left[
\frac{d{\bf U}_0}{dt} + \frac{\mu}{m}\,\nabla B\right].
\end{equation}
Note that the first-order correction to the parallel velocity, the parallel
drift velocity, is undetermined to this order. This is not generally a problem,
since the first-order parallel drift  is a small correction to a type of motion
which already exists at zeroth-order, whereas the first-order perpendicular drift is
a completely new type of motion. In particular, the first-order
perpendicular drift differs
fundamentally from the ${\bf E}\times{\bf B}$ drift, since it is
not the same for different species, and, therefore, cannot be eliminated by transforming
to a new inertial frame. 

We can now understand the motion of a charged particle as it moves through
slowly varying electric and magnetic fields. The particle always gyrates around
the magnetic field at the local gyrofrequency ${\Omega}=eB/m$. 
The local perpendicular gyration velocity $u_\perp$ is determined by the
requirement that the magnetic moment $\mu=m\,u_\perp^{~2}/ 2\,B$ be a
constant of the motion. This, in turn,  fixes the local gyroradius 
$\rho = u_\perp/{\Omega}$.
The parallel velocity of the particle is determined by Eq.~(\ref{e2.45}). 
Finally, the perpendicular drift velocity is the sum  of the ${\bf E}\times{\bf B}$
drift velocity ${\bf v}_E$ and the first-order drift velocity ${\bf U}_{1\,\perp}.$

\section{Magnetic Drifts}
Equations (\ref{e2.30}) and (\ref{e2.46}) can be combined to give
\begin{equation}\label{e2.47}
{\bf U}_{1\,\perp} = \frac{\mu}{m\,{\Omega}}\,
{\bf b}\times \nabla B +
\frac{U_{0\,\parallel}}{{\Omega}}\, {\bf b}\times\frac{d{\bf b}}{dt}
+ \frac{{\bf b}}{{\Omega}}\times \frac{d{\bf v}_E}{dt}.
\end{equation}
The three terms on the right-hand side of the above expression are conventionally
called the {\em magnetic}, or {\em grad-B}, {\em drift}, the {\em inertial
drift}, and the {\em polarization drift}, respectively. 

The magnetic drift,
\begin{equation}
{\bf U}_{\rm mag} = \frac{\mu}{m\,{\Omega}}\,
{\bf b}\times \nabla B,
\end{equation}
is caused by the slight variation of the gyroradius with gyrophase as a
charged  particle
rotates  in a non-uniform magnetic field. The gyroradius is reduced
on the high-field side of the Larmor orbit, whereas it is increased on the
low-field side. The net result is that the orbit does not quite
close. In fact, the motion consists of the conventional  gyration around the magnetic
field combined with a slow drift which is perpendicular to both the
local direction of the magnetic field and the local gradient of the 
field-strength.

Given that
\begin{equation}
\frac{d{\bf b}}{dt} = \frac{\partial{\bf b}}{\partial t} + ({\bf v}_E\cdot
\nabla) \,{\bf b} + U_{0\,\parallel} \,({\bf b}\cdot\nabla)\,{\bf b},
\end{equation}
the inertial drift can be written
\begin{equation}
{\bf U}_{\rm int} = \frac{U_{0\,\parallel}}{{\Omega}}\,
{\bf b}\times\left[\frac{\partial{\bf b}}{\partial t} +
({\bf v}_E\cdot\nabla)\,{\bf b}\right] + \frac{U_{0\,\parallel}^{~2}}{{\Omega}}\,
{\bf b}\times ({\bf b}\cdot\nabla)\,{\bf b}.
\end{equation}
In the important limit of stationary magnetic fields and weak electric fields, the 
above expression is dominated by the final term,
\begin{equation}
{\bf U}_{\rm curv} = \frac{U_{0\,\parallel}^{~2}}{{\Omega}}\,
{\bf b}\times ({\bf b}\cdot\nabla)\,{\bf b},
\end{equation}
which is called the {\em curvature drift}. 
As is easily demonstrated, the quantity $({\bf b}\cdot\nabla)\,{\bf b}$
is a vector whose direction is  towards the centre of the circle which
most closely approximates the magnetic field-line at a given point, and whose
magnitude is the inverse of the radius of this circle. Thus, the
centripetal acceleration imposed by the curvature of the magnetic field
on a charged particle following a field-line gives rise to a slow drift which is
perpendicular to both the local direction of the magnetic field and the
 direction to the local centre of curvature of the field. 

The polarization drift,
\begin{equation}
{\bf U}_{\rm polz} = \frac{{\bf b}}{{\Omega}}\times \frac{d{\bf v}_E}{dt},
\end{equation}
reduces to
\begin{equation}\label{e2.53}
{\bf U}_{\rm polz} = \frac{1}{{\Omega}} 
\frac{d}{dt}\!\left(\frac{{\bf E}_\perp}
{B}\right)
\end{equation}
in the limit in which the magnetic field is stationary but the electric
field varies in time. This expression can be understood as a polarization drift
by considering what happens when we suddenly impose an electric field on a
particle at rest. The particle initially accelerates in the direction of
the electric field, but is then deflected by the magnetic force. Thereafter,
the particle undergoes conventional gyromotion combined with
${\bf E}\times{\bf B}$ drift. The time
between the switch-on of the field and the magnetic deflection is approximately
${\Delta}t\sim {\Omega}^{-1}$. Note that there is no deflection if
the electric field is directed parallel to the magnetic field,
so this argument only applies to perpendicular electric fields. The initial
displacement of the particle in the direction of the field is of order 
\begin{equation}\label{e2.54}
{\bf \delta} \sim
\frac{e\,{\bf E}_{\perp}}{m}\,({\Delta} t)^2 \sim \frac{{\bf E}_\perp}{
{\Omega}\,B}.
\end{equation}
Note that, because ${\Omega}\propto m^{-1}$, the displacement of the
ions {\em greatly exceeds}\/ that of the electrons. 
Thus, when an electric field
is suddenly switched on in a plasma, there is an initial polarization of
the plasma medium caused, predominately, by a displacement of the ions in the direction of the
field. If the electric field, in fact, varies {\em continuously}\/
in time, then there is a slow drift due to the constantly changing polarization of the
plasma medium. This drift is essentially the time derivative of Eq.~(\ref{e2.54}) [{\em i.e.},
Eq.~(\ref{e2.53})].

\section{Invariance of  Magnetic Moment}
Let us now demonstrate that the magnetic moment $\mu=m\, u_\perp^2 / 2\,B$
is indeed a constant of the motion, at least to lowest order. The scalar
product of the equation of motion (\ref{e2.21b}) with the velocity ${\bf v}$ yields
\begin{equation}
\frac{m}{2} \frac{d v^2}{dt} = e\,{\bf v}\cdot{\bf E}.
\end{equation}
This equation governs the evolution of the  particle energy during its
motion. Let us make the substitution ${\bf v}  = {\bf U} + {\bf u}$,
as before, and then average the above equation over gyrophase. To lowest order, we obtain
\begin{equation}\label{e2.56}
\frac{m}{2} \frac{d}{dt} (u_\perp^{~2} + U_0^{~2}) = e\,U_{0\,\parallel}
\,E_\parallel + e\, {\bf U}_1\cdot {\bf E} + e\,\langle
{\bf u}\cdot (\brho\cdot\nabla)\,{\bf E}\rangle.
\end{equation}
Here, use has been made of the result
\begin{equation}
\frac{d}{dt} \langle f\rangle = \langle \frac{df}{dt}\rangle,
\end{equation}
which is valid for any $f$. The final term on the right-hand side of
Eq.~(\ref{e2.56}) can be written
\begin{equation}
e\,{\Omega} \,\langle (\brho\times{\bf b})\cdot (\brho\cdot\nabla)
\,{\bf E}\rangle = - \mu\,{\bf b}\cdot\nabla\times{\bf E} = \bmu\cdot
\frac{\partial {\bf B}}{\partial t}.
\end{equation}
Thus, Eq.~(\ref{e2.56}) reduces to
\begin{equation}\label{e2.59}
\frac{d K}{dt} = e\,{\bf U}\cdot{\bf E} + \bmu\cdot
\frac{\partial {\bf B}}{\partial t} = e\,{\bf U}\cdot{\bf E} + \mu\,
\frac{\partial B}{\partial t}.
\end{equation}
Here, ${\bf U}$ is the guiding centre velocity, evaluated to first order, and
\begin{equation}
K = \frac{m}{2} (U_{0\,\parallel}^{~2} + {\bf v}_E^{~2} + u_\perp^{~2})
\end{equation}
is the kinetic energy of the particle. Evidently, the kinetic energy can
change in one of two ways. Either by motion of the guiding centre along the
direction of the electric field, or by the acceleration of the gyration due
to the electromotive force generated around the Larmor orbit by a
changing magnetic field. 

Equations~(\ref{e2.30}), (\ref{e2.45}), and (\ref{e2.46}) can be used to eliminate $U_{0\,\parallel}$
and ${\bf U}_1$ from Eq.~(\ref{e2.59}). The final result is
\begin{equation}
\frac{d}{dt}\!\left(\frac{ m\,u_\perp^{~2}}{2\,B}\right) = \frac{d\mu}{dt} = 0.
\end{equation}
Thus, the magnetic moment $\mu$ is a constant of the motion to lowest order.
Kruskal\footnote{M.~Kruskal, J.\ Math.\ Phys.\ {\bf 3}, 806
(1962).} has shown that $m\,u_\perp^{~2}/2\,B$ is the lowest order 
approximation to a quantity which is a constant of the motion to
{\em all}\/ orders in the perturbation expansion. Such a quantity
is called an {\em adiabatic invariant}.

\section{Poincar\'{e} Invariants}
An adiabatic invariant is an approximation to a more fundamental type of
invariant known as a {\em Poincar\'{e} invariant}. A Poincar\'{e} invariant
takes the form
\begin{equation}
{\cal I} = \oint_{C(t)} {\bf p}\cdot d{\bf q},
\end{equation}
where all points on the closed curve $C(t)$ in phase-space move
according to the equations of motion. 

In order to demonstrate that ${\cal I}$ is a constant of the motion, we
introduce a periodic variable $s$ parameterizing the points on the curve $C$.
The coordinates of a general point on $C$ are thus written $q_i = q_i(s,t)$
and $p_i=p_i(s,t)$. The rate of change of ${\cal I}$ is then
\begin{equation}
\frac{d{\cal I}}{dt} =\oint\left(p_i\,\frac{\partial^2 q_i}{\partial t\,\partial s}
+\frac{\partial p_i}{\partial t} \frac{\partial q_i}{\partial s}\right)\,ds.
\end{equation}
We integrate the first term by parts, and then used Hamilton's
equations of motion to simplify the result. We obtain
\begin{equation}
\frac{d{\cal I}}{dt} =\oint\left( -
 \frac{\partial q_i}{\partial t}\frac{\partial p_i}{\partial s}
+\frac{\partial p_i}{\partial t} \frac{\partial q_i}{\partial s}\right)\,ds
=-\oint\left(
 \frac{\partial H}{\partial p_i}\frac{\partial p_i}{\partial s}+\frac{\partial H}{\partial q_i}
 \frac{\partial q_i}{\partial s}\right)\,ds,
\end{equation}
where $H({\bf p}, {\bf q}, t)$ is the Hamiltonian for the motion.
The integrand is now seen to be the total derivative of $H$ along $C$. 
Since the Hamiltonian is a single-valued function, it follows that
\begin{equation}
\frac{d{\cal I}}{dt} =-\oint\frac{d H}{ds}\,ds =0.
\end{equation}
Thus, ${\cal I}$ is indeed a constant of the motion. 

\section{Adiabatic Invariants}
Poincar\'{e} invariants are generally of little practical interest
unless the curve $C$ closely corresponds to the trajectories of 
actual particles. Now, for the motion of magnetized particles
it is clear from Eqs.~(\ref{e2.22a}) and (\ref{e2.33}) that points having the same
guiding centre at a certain time will continue to have approximately
the same guiding centre at a later time. An approximate Poincar\'{e}
invariant may thus be obtained by choosing the curve $C$ to be a circle of points
corresponding to a gyrophase period. In other words,
\begin{equation}
{\cal I} \simeq I = \oint {\bf p}\cdot \frac{\partial{\bf q}}{\partial\gamma}\,
d\gamma.
\end{equation}
Here, $I$ is an {\em adiabatic invariant}.

To evaluate $I$ for a magnetized plasma recall that the canonical momentum
for charged particles is
\begin{equation}
{\bf p} = m\,{\bf v} +e\,{\bf A},
\end{equation}
where ${\bf A}$ is the vector potential. We express ${\bf A}$ in terms
of its Taylor series about the guiding centre position:
\begin{equation}
{\bf A}({\bf r}) = {\bf A}({\bf R}) + (\brho\cdot\nabla)\,{\bf A}({\bf R})
+ O(\rho^2).
\end{equation}
The element of length along the curve $C(t)$ is [see Eq.~(\ref{e2.34})]
\begin{equation}
d{\bf r} = \frac{\partial\brho}{\partial \gamma}\,d\gamma = 
\frac{{\bf u}}{{\Omega}}\,\,d\gamma.
\end{equation}
The adiabatic invariant is thus
\begin{equation}
I = \oint \frac{{\bf u}}{{\Omega}} \cdot \left\{
m\,({\bf U} + {\bf u}) + e\,\left[{\bf A} + (\brho\cdot
\nabla)\,{\bf A}\right]\right\}\,d\gamma + O(\epsilon),
\end{equation}
which reduces to
\begin{equation}
I = 2\pi\,m\,\frac{u_\perp^{~2}}{{\Omega}} + 2\pi\,\frac{e}{{\Omega}}\,
\langle {\bf u}\cdot(\brho\cdot
\nabla)\,{\bf A}\rangle + O(\epsilon).
\end{equation}
The final term on the right-hand side is written [see Eq.~(\ref{e2.36})]
\begin{equation}
2\pi\,e\,\langle (\brho\times{\bf b}) \cdot(\brho\cdot
\nabla)\,{\bf A}\rangle = -2\pi\,e\,\frac{u_\perp^{~2}}{2\,{\Omega}^2}\,
{\bf b}\cdot \nabla\times{\bf A} = -\pi\,m\,\frac{u_\perp^{~2}}{{\Omega}}.
\end{equation}
It follows that
\begin{equation}
I = 2\pi\, \frac{m}{e}\,\mu + O(\epsilon).
\end{equation}
Thus, to lowest order the adiabatic invariant is proportional to the magnetic moment $\mu$. 

\section{Magnetic Mirrors}
Consider the important case in which
 the electromagnetic fields do not vary in time. It immediately follows from
Eq.~(\ref{e2.59}) that
\begin{equation}
\frac{d{\cal E}}{dt} =0,
\end{equation}
where
\begin{equation}\label{e2.75}
{\cal E} = K + e\,\phi = \frac{m}{2}\,(U_\parallel^{~2} + {\bf v}_E^{~2}) +\mu\,B
+ e\,\phi
\end{equation}
is the total particle energy, and $\phi$ is the electrostatic potential.
Not surprisingly, a charged particle neither gains nor loses energy as
it moves around in non-time-varying electromagnetic fields. Since
both ${\cal E}$ and $\mu$ are constants of the motion, we can rearrange
Eq.~(\ref{e2.75}) to give
\begin{equation}
U_\parallel = \pm \sqrt{(2/m)[{\cal E} -\mu\,B-e\,\phi]-{\bf v}_E^{~2}}.
\end{equation}
Thus, in regions where ${\cal E} > \mu\,B +e\,\phi + m\,{\bf v}_E^{~2}/2$
charged particles can drift in either direction along magnetic field-lines.
However, particles are excluded from regions where
${\cal E} < \mu\,B +e\,\phi + m\,{\bf v}_E^{~2}/2$ (since particles cannot have
imaginary parallel velocities!). Evidently, charged particles must reverse direction
at those points on magnetic field-lines where ${\cal E} = \mu\,B +e\,\phi + m\,{\bf v}_E^{~2}/2$:
such points are termed ``bounce points'' or ``mirror points.''

Let us now consider how we might construct a device to confine a 
collisionless ({\em i.e.}, very hot) plasma. Obviously, we cannot use conventional
solid walls, because they would melt. However, it is possible to confine a
hot plasma using a magnetic field (fortunately, magnetic fields do not melt!): this 
technique is
called {\em magnetic confinement}.
The electric field in confined plasmas is
usually weak ({\em i.e.}, $E\ll B\,v$), so that the ${\bf E}\times {\bf B}$ drift
is similar in magnitude to the magnetic and curvature drifts. In this
case, the bounce point condition, $U_\parallel = 0$, reduces to
\begin{equation}\label{e2.77}
{\cal E} = \mu\,B.
\end{equation}
Consider the magnetic field configuration shown in Fig.~1. This is most easily
 produced using two Helmholtz coils. Incidentally, this type of magnetic
confinement device is called a {\em magnetic mirror machine}. 
The magnetic field configuration obviously possesses
axial symmetry. Let $z$ be a coordinate which measures distance along the
axis of symmetry. Suppose that $z=0$ corresponds to the mid-plane of the device
({\em i.e.}, halfway between the two field-coils). 

\begin{figure}
\epsfysize=2.5in
\centerline{\epsffile{Chapter02/mirror.eps}}
\caption{\em Motion of a trapped particle in a mirror machine.}\label{f1}
\end{figure}

It is clear from Fig.~\ref{f1} that the magnetic field-strength $B(z)$ 
on a magnetic field-line situated close to the axis of the device attains a
local minimum $B_{\rm min}$ at $z=0$,  increases symmetrically
 as  $|z|$ increases until
reaching a maximum value $B_{\rm max}$ at about the location of the two
field-coils, and then decreases as $|z|$ is further increased. According to
Eq.~(\ref{e2.77}), any particle which satisfies the inequality
\begin{equation}\label{e2.78}
\mu> \mu_{\rm trap}  = \frac{{\cal E}}{B_{\rm max}}
\end{equation}
is {\em trapped}\/ on such a  field-line. In fact, the particle undergoes
periodic motion along the field-line between two symmetrically placed (in $z$)
mirror points. The magnetic field-strength at the mirror points is
\begin{equation}
B_{\rm mirror} = \frac{\mu_{\rm trap}}{\mu}\,B_{\rm max} < B_{\rm max}.
\end{equation}

Now, on the mid-plane $\mu = m\, v_\perp^{~2}/2\, B_{\rm min}$ and
${\cal E} = m\,(v_\parallel^{~2} + v_\perp^{~2})/2$. 
({\em n.b.}\/ From now on, we shall write ${\bf v} = v_\parallel\,{\bf b} +
{\bf v}_\perp$, for ease of notation.)  Thus, the trapping
condition (\ref{e2.78}) reduces to
\begin{equation}\label{e2.80}
\frac{|v_\parallel|}{|v_\perp|} < (B_{\rm max}/B_{\rm min} - 1)^{1/2}.
\end{equation}
Particles on the mid-plane which satisfy this inequality are trapped: particles
which do not satisfy this inequality escape along magnetic field-lines.
Clearly, a magnetic mirror machine is incapable of trapping charged particles which
are moving parallel, or nearly parallel, to the direction of the magnetic field.
In fact, the above inequality defines a {\em loss cone}\/ in velocity space---see
Fig.~\ref{f2}. 

\begin{figure}
\epsfysize=2.5in
\centerline{\epsffile{Chapter02/loss_cone1.eps}}
\caption{\em Loss cone in velocity space. The particles lying inside the cone are
not reflected by the magnetic field.}\label{f2}
\end{figure}

It is clear that if  plasma is placed inside  a magnetic mirror machine then all
of the particles whose velocities lie in the loss cone  promptly escape, but the
remaining particles are confined. Unfortunately, that is not
the end of the story. There is no such thing as an absolutely collisionless
plasma. Collisions take place at a low rate even in very hot plasmas. 
One important 
effect of collisions is to cause {\em diffusion}\/ of particles in velocity space.
Thus, in a mirror machine collisions continuously scatter trapped particles into
the loss cone, giving rise to a slow leakage of plasma out of the device. 
Even worse, plasmas whose distribution functions deviate strongly from an isotropic
Maxwellian ({\em e.g.}, a plasma confined
in a mirror machine) are prone to {\em velocity space instabilities}, which tend to
relax the distribution function back to a Maxwellian. Clearly, such instabilities
are likely to have a disastrous effect on plasma confinement in a mirror machine. 
For these reasons, magnetic mirror machines are not particularly successful
plasma confinement devices, and attempts to achieve nuclear fusion using
this type of  device have mostly been abandoned.\footnote{This is not quite
true. In fact, fusion scientists have developed advanced mirror concepts
which do not suffer from the severe end-losses characteristic of
standard mirror machines. Mirror research is still being carried out, 
albeit at a
comparatively low level, in Russia and Japan.}

\section{Van Allen Radiation Belts}
Plasma confinement via magnetic mirroring occurs in nature as
well as in unsuccessful fusion devices. For instance, the Van Allen radiation belts,
which surround the Earth, consist of energetic particles trapped 
in the Earth's dipole-like magnetic field. These belts were discovered by 
James A.~Van Allen and co-workers using data taken from Geiger counters
which flew on the early U.S.\ satellites, Explorer~1 (which was, in fact, the
first U.S.\ satellite), Explorer~4, and Pioneer~3. Van Allen was actually
trying to measure the flux of cosmic rays (high energy particles whose origin is
outside the Solar System) in outer space, to see if it
was similar to that measured on Earth. However, the flux of energetic particles
detected by his instruments so greatly exceeded the expected value that it
prompted one of his co-workers to exclaim, ``My God, space is radioactive!''
It was quickly realized that this flux was due to energetic
particles trapped in the Earth's magnetic field, rather than to cosmic rays. 

There are, in fact, two radiation belts surrounding the Earth. The {\em inner
belt}, which extends from about 1--3 Earth radii in the equatorial plane
is mostly populated by protons with energies exceeding $10$~MeV. The origin of
these protons is thought to be the decay of neutrons which are emitted from
the Earth's atmosphere as it is bombarded by cosmic rays. The inner belt is fairly
quiescent. Particles eventually escape due to collisions with neutral atoms
in the upper atmosphere above the Earth's poles. However, such collisions
are sufficiently uncommon that the lifetime of particles in the belt  range
from a few hours to 10 years. Clearly, with such long trapping
times only a small input rate of energetic particles is required to produce a
region of intense radiation. 

The {\em outer belt}, which extends from about 3--9 Earth radii in the equatorial
plane, consists mostly of electrons with energies below $10$~MeV. The origin
of these electrons is via injection from the outer magnetosphere. Unlike the
inner belt, the outer belt is very dynamic, changing on time-scales of a few
hours in response to perturbations emanating from the outer magnetosphere. 

In regions not too far distant 
({\em i.e.}, less than 10 Earth radii) from the Earth, the geomagnetic field can be approximated
as a dipole field,
\begin{equation}\label{e2.81}
{\bf B} = \frac{\mu_0}{4\pi} \frac{M_E}{r^3} (-2\cos\theta, -\sin\theta, \,0),
\end{equation}
where we have adopted conventional spherical polar coordinates $(r,\theta,\varphi)$ aligned
with the Earth's dipole moment, whose magnitude is $M_E = 8.05\times 10^{22}~{\rm A\,m}^2$. It is usually convenient to work in terms of the {\em latitude}, $\vartheta
= \pi/2 - \theta$, rather than the polar angle, $\theta$. 
An individual magnetic  field-line satisfies the equation
\begin{equation}\label{e2.82}
r = r_{\rm eq}\,\cos^2\vartheta,
\end{equation}
where $r_{\rm eq}$ is the radial distance to the field-line in the equatorial
plane ($\vartheta=0^\circ$). It is conventional to label field-lines
using the {\em L-shell parameter}, $L=r_{\rm eq}/R_E$. Here, $R_E = 6.37\times 10^6\,{\rm m}$ is the Earth's radius. Thus, the variation of the
 magnetic field-strength along a field-line characterized by a given $L$-value
is
\begin{equation}\label{e2.83}
B = \frac{B_E}{L^3} \frac{(1+3\,\sin^2\vartheta)^{1/2}}{\cos^6\vartheta},
\end{equation}
where $B_E= \mu_0 M_E/(4\pi\, R_E^{~3}) = 3.11\times 10^{-5}\,{\rm T}$ is the
equatorial magnetic field-strength on the Earth's surface.

Consider, for the sake of simplicity, charged particles located
 on the equatorial plane ($\vartheta=0^\circ$)
whose velocities are predominately directed perpendicular to the magnetic
field. The proton and electron gyrofrequencies are written\footnote{It
is conventional to take account of the negative charge of electrons by
making the electron gyrofrequency ${\Omega}_e$ negative. This approach is
implicit in formulae such as Eq.~(\ref{e2.47}).}
\begin{equation}
{\Omega}_p = \frac{e\,B}{m_p} = 2.98\,L^{-3}\,\,{\rm kHz},
\end{equation}
and
\begin{equation}
|{\Omega}_e| = \frac{e\,B}{m_e} = 5.46\,L^{-3}\,\,{\rm MHz},
\end{equation}
respectively. The proton and electron gyroradii, expressed as fractions
of the Earth's radius, take the form
\begin{equation}
\frac{\rho_p}{R_E} = \frac{\sqrt{2\,{\cal E}\,m_p}}{e\,B\,R_E} =
\sqrt{{\cal E}({\rm MeV)}}\left( \frac{L}{11.1}\right)^3,
\end{equation}
and
\begin{equation}
\frac{\rho_e}{R_E} =  \frac{\sqrt{2\,{\cal E}\,m_e}}{e\,B\,R_E} =\sqrt{{\cal E}({\rm MeV)}}\left( 
\frac{L}{38.9}\right)^3,
\end{equation}
respectively. It is clear that MeV energy charged particles in the inner
magnetosphere ({\em i.e}, $L \ll 10$) gyrate at frequencies which are much
greater than the typical  rate of change of the magnetic field (which changes on
time-scales which are, at most, a few minutes). Likewise, the gyroradii of
such  particles are much smaller than the typical variation length-scale
of the magnetospheric magnetic field. 
Under these circumstances, we expect the magnetic moment 
to be a conserved quantity:
{\em i.e.}, we expect the magnetic moment to be a good adiabatic invariant. 
It immediately follows that any MeV energy protons and electrons in the inner magnetosphere
which have a
sufficiently large magnetic moment are {\em trapped}\/ on the dipolar field-lines of the Earth's magnetic
field, bouncing back and forth between mirror points located just above
the Earth's poles.

It is helpful to define the {\em pitch-angle},
\begin{equation}
\alpha = \tan^{-1}(v_\perp/v_\parallel),
\end{equation}
of a charged particle in the magnetosphere. If the magnetic moment is
a conserved quantity then a particle of fixed energy drifting along a  field-line
satisfies
\begin{equation}\label{e2.89}
\frac{\sin^2\alpha}{\sin^2\alpha_{\rm eq}}=\frac{B}{B_{\rm eq}},
\end{equation}
where $\alpha_{\rm eq}$ is the {\em equatorial pitch-angle}\/ ({\em i.e.}, the
pitch-angle on the equatorial plane) and $B_{\rm eq}=B_E/L^3$ is the magnetic
field-strength on the equatorial plane. It is clear from Eq.~(\ref{e2.83}) that the
pitch-angle increases ({\em i.e.}, the parallel component of the particle
velocity decreases)
as the particle drifts off the equatorial plane towards the Earth's
poles.


The mirror points correspond to $\alpha=90^\circ$ ({\em i.e.}, $v_\parallel=0$).
It follows from Eqs.~(\ref{e2.83}) and (\ref{e2.89}) that 
\begin{equation}\label{e2.90}
\sin^2\alpha_{\rm eq} = \frac{B_{\rm eq}}{B_m} = \frac{\cos^6\vartheta_m}
{(1+3\,\sin^2\vartheta_m)^{1/2}},
\end{equation}
where $B_m$ is the magnetic field-strength at the mirror points, and $\vartheta_m$
is the latitude of the mirror points. Clearly, the latitude of a particle's
mirror point depends only on its equatorial pitch-angle, and is independent
of the $L$-value of the field-line on which it is trapped.

Charged particles with large equatorial pitch-angles have small parallel
velocities, and mirror points located at relatively low latitudes. Conversely,
charged particles with small equatorial pitch-angles have large parallel velocities,
and mirror points located at high latitudes. Of course, if the pitch-angle
becomes too small then the mirror points enter the Earth's atmosphere, and
the particles are lost via collisions with neutral particles.
Neglecting the thickness of the atmosphere with respect to
the radius of the Earth, we can say that all particles whose mirror points
lie inside the Earth are lost via collisions. It follows from
Eq.~(\ref{e2.90}) that the {\em equatorial loss cone}\/ is of approximate width
\begin{equation}
\sin^2\alpha_l = \frac{\cos^6\vartheta_E}{(1+3\sin^2\vartheta_E)^{1/2}},
\end{equation}
where $\vartheta_E$ is the latitude of the point where the magnetic field-line
under investigation intersects the Earth. 
Note that all particles with $|\alpha_{\rm eq}| < \alpha_l$ 
and $|\pi-\alpha_{\rm eq}| < \alpha_l$ lie in the
loss cone. 
It is easily demonstrated from Eq.~(\ref{e2.82}) that
\begin{equation}
\cos^2\vartheta_E = L^{-1}.
\end{equation}
It follows that
\begin{equation}
\sin^2\alpha_l = (4\,L^6 - 3\,L^5)^{-1/2}.
\end{equation}
Thus, the width of the loss cone is independent of the charge, the mass, or
the energy of the particles drifting along a given field-line, 
and is a function only of the field-line radius
on the equatorial plane. The loss cone is surprisingly small. For instance,
at the radius of a geostationary orbit ($6.6\,R_E$), the loss cone is
less than $3^\circ$ degrees wide. The smallness of the 
 loss cone is a consequence of  the very strong variation of the magnetic field-strength along field-lines in a dipole field---see Eqs.~(\ref{e2.80}) and (\ref{e2.83}). 

A dipole field is clearly a far more effective configuration for 
confining a collisionless plasma via magnetic mirroring
 than the more traditional linear configuration 
 shown in Fig.~\ref{f1}. In fact, M.I.T.\ has recently constructed a dipole
 mirror machine. The dipole field is generated by a
superconducting current loop levitating in a vacuum chamber. 

The {\em bounce period}, $\tau_b$, is the time it takes a particle to move
from the equatorial plane to one mirror point, then to the other, and then return
to the equatorial plane. It follows that
\begin{equation}
\tau_b = 4\int_0^{\vartheta_m} \frac{d\vartheta}{v_\parallel}\frac{ds}{d\vartheta},
\end{equation}
where $ds$ is an element of arc length along the field-line under investigation,
and $v_\parallel = v\,(1-B/B_m)^{1/2}$. The above integral cannot
be performed analytically. However, it can be solved numerically, and is
conveniently  approximated as
\begin{equation}
\tau_b \simeq \frac{L\,R_E}{({\cal E}/m)^{1/2}} \,(3.7-1.6\sin\alpha_{\rm eq}).
\end{equation}
Thus, for protons 
\begin{equation}
(\tau_b)_p \simeq 2.41\,\frac{L}
{\sqrt{{\cal E}({\rm MeV})}} \,(1- 0.43 \sin\alpha_{\rm eq})\,\,\,{\rm secs},
\end{equation}
whilst for electrons
\begin{equation}
(\tau_b)_e \simeq 5.62\times 10^{-2}\frac{L}{\sqrt{{\cal E}({\rm MeV})}} \,(1- 0.43 \sin\alpha_{\rm eq})\,\,\,{\rm secs}.
\end{equation}
It follows that  MeV electrons typically have bounce periods which are less than
a second, whereas the bounce periods for MeV protons usually lie in the range 1 to
10 seconds. The bounce period only depends weakly on equatorial
pitch-angle, since particles with small pitch angles have relatively
large parallel velocities but a comparatively long way to travel to their
mirror points, and {\em vice versa}. Naturally, the bounce period
is longer for longer field-lines ({\em i.e.}, for larger $L$). 

\section{Ring Current}
Up to now, we have only considered the lowest order motion ({\em i.e.}, 
gyration combined with parallel drift) of charged particles in the
magnetosphere. Let us now examine the higher order corrections to this
motion. For the case of non-time-varying fields, and  a weak electric
field, these corrections consist of a combination of ${\bf E}\times{\bf B}$ drift,
grad-$B$ drift, and curvature drift:
\begin{equation}
{\bf v}_{1\perp} = \frac{{\bf E}\times{\bf B}}{B^2} + \frac{\mu}{m\,{\Omega}}
\,\,{\bf b}\times\nabla B +\frac{ v_\parallel^{~2} }{{\Omega}}\,
{\bf b}\times({\bf b}\cdot\nabla)\,{\bf b}.
\end{equation}
Let us neglect ${\bf E}\times{\bf B}$ drift, since this motion merely gives
rise to the convection of plasma within the magnetosphere, without generating a
current. By contrast, there is a net current associated with  grad-$B$ drift
and curvature drift. In the limit in which this current does not strongly
modify the ambient magnetic field ({\em i.e.},
 $\nabla\times{\bf B} \simeq {\bf 0}$),
which is certainly the situation in the Earth's magnetosphere, we can write
\begin{equation}
({\bf b}\cdot\nabla)\,{\bf b} =-{\bf b}\times(\nabla\times
{\bf b})\simeq \frac{\nabla_\perp B}{B}.
\end{equation}
It follows that the higher order drifts can be combined to give
\begin{equation}
{\bf v}_{1\perp} = \frac{(v_\perp^{~2}/2+v_\parallel^{~2})}{{\Omega}\,B}\,
{\bf b}\times\nabla B.
\end{equation}
For the dipole field (\ref{e2.81}), the above expression yields
\begin{equation}
{\bf v}_{1\perp} \simeq -{\rm sgn}({\Omega})\,\frac{6\,{\cal E}\,L^2}{e\,B_E\,R_E} \,\,
(1-B/2B_m)\,\frac{\cos^5\vartheta\,(1+\sin^2\vartheta)}{(1+3\sin^2\vartheta)^2}\,\,
\hat{\bvarphi}.\label{e2.101}
\end{equation}
Note that the drift is in the azimuthal direction. A positive drift velocity
corresponds
to eastward motion, whereas a negative velocity corresponds to westward motion.
It is  clear that, in addition to their gyromotion and  periodic
bouncing motion along field-lines, charged particles trapped
in the magnetosphere also
slowly {\em precess}\/ around the Earth. The ions drift westwards and the electrons
drift eastwards, giving rise to a net westward current circulating around the
Earth. This current is known as the {\em ring current}.

Although the perturbations to the Earth's magnetic field induced by the ring
current are small, they are still detectable. In fact, the ring current
causes a slight {\em reduction}\/ in the Earth's magnetic field in equatorial
regions. The size of this reduction is a good measure of the
number of charged particles contained in the Van Allen belts. During
the development of so-called {\em geomagnetic storms}, charged particles are
injected into the Van Allen belts from the outer magnetosphere, giving rise
to a sharp increase in the ring current, and a corresponding decrease in the
Earth's equatorial magnetic field. These particles eventually precipitate
out of the magnetosphere into the upper atmosphere
at high latitudes, giving rise to intense
auroral activity, serious interference in electromagnetic communications, and, in extreme
cases,  disruption of electric
power grids. The ring current induced reduction in the
Earth's magnetic field is measured by the so-called
{\em Dst index}, which is based on
hourly averages of the northward horizontal component of the terrestrial
magnetic 
field recorded at four low-latitude observatories; Honolulu (Hawaii), 
San Juan (Puerto Rico), Hermanus (South Africa),
and Kakioka (Japan). Figure~\ref{f5} shows the Dst index for the month of March
1989.\footnote{Dst data
is freely availabel from the following web site in Kyoto (Japan): {\tt
 http://swdcdb.kugi.kyoto-u.ac.jp/dstdir}}
The very marked reduction in the index, centred about March 13th, corresponds
to one of the most severe geomagnetic storms experienced in recent decades. 
In fact, this particular storm was so severe that it tripped out the
whole Hydro~Quebec electric distribution system, plunging more than 6 million
customers into darkness. Most of Hydro~Quebec's neighbouring systems in the United
States came uncomfortably close to experiencing the same cascading power
outage scenario. Note that a reduction in the  Dst index by 600\,nT
corresponds to a $2\%$ reduction in the terrestrial magnetic field at the equator.

\begin{figure}
\epsfysize=1.2in
\centerline{\epsffile{Chapter02/dstfinal198903-1.ps}}
\caption{\em Dst data for March 1989 showing an exceptionally
severe geomagnetic storm on the 13th.}\label{f5}
\end{figure}

According to Eq.~(\ref{e2.101}), the precessional drift velocity of charged particles
in the magnetosphere is a rapidly decreasing function of increasing latitude
({\em i.e.}, most of the ring current is concentrated in the equatorial plane).
Since particles typically complete many bounce orbits during a full rotation around
the Earth, it is convenient to average Eq.~(\ref{e2.101}) over a bounce
period to obtain the {\em average drift velocity}. This averaging can
only be performed numerically. The final answer is well approximated by
\begin{equation}\label{e2.102}
\langle v_d\rangle \simeq \frac{6\,{\cal E}\,L^2}{e\,B_E\,R_E}
\,(0.35 + 0.15\,\sin\alpha_{\rm eq}).
\end{equation}
The {\em average drift period}\/ ({\em i.e.}, the time required to
perform a complete rotation around the Earth) is simply
\begin{equation}
\langle \tau_d\rangle = \frac{2\pi\,L\,R_E}{\langle v_d\rangle}
\simeq \frac{\pi\,e\,B_E\,R_E^{~2}}{3\,{\cal E}\,L} \,
(0.35 + 0.15\,\sin\alpha_{\rm eq})^{-1}.
\end{equation}
Thus, the drift period for protons and electrons is
\begin{equation}
\langle \tau_d\rangle_p=\langle \tau_d\rangle_e \simeq \frac{1.05}{{\cal E}({\rm MeV})\,L} \,(1+0.43\,\sin\alpha_{\rm eq})^{-1}\,\,\,{\rm hours}.
\end{equation}
Note that MeV energy electrons and ions  precess around the Earth with about the same
velocity, only in opposite directions, because there is no explicit
mass dependence in Eq.~(\ref{e2.102}). It typically takes an hour to perform a full
rotation. 
The drift period only depends weakly on
the equatorial pitch angle, as is the case for the bounce period. 
Somewhat paradoxically, the drift period is
shorter on more distant $L$-shells. 
Note, of course, that particles only get a chance to complete a 
full rotation around the Earth if the inner magnetosphere remains quiescent
on time-scales of order an hour, which is, by no means, always the case. 

Note, finally, that, since the rest mass of an electron is $0.51$\,MeV, most
of the above formulae require relativistic correction when applied to
MeV energy electrons. 
Fortunately, however, there is no such problem for protons, whose rest mass
energy is $0.94$\,GeV.

\section{Second Adiabatic Invariant}
We have seen that there is an adiabatic invariant associated with the
periodic gyration of a charged particle around magnetic field-lines.
Thus, it is reasonable to suppose that there is a second adiabatic invariant
associated with the periodic bouncing motion of a particle trapped 
between two mirror points on
a magnetic field-line. This is indeed the case. 

Recall that an adiabatic invariant is the lowest order approximation
to a Poincar\'{e} invariant:
\begin{equation}
{\cal J} = \oint_C {\bf p}\cdot d{\bf q}.
\end{equation}
In this case, let the curve $C$ correspond to the trajectory of
a guiding centre  as a charged
particle trapped in the
Earth's magnetic field 
executes a bounce orbit. Of course, this trajectory does not quite close,
because of the slow azimuthal drift of particles around the Earth. However,
it is easily demonstrated that 
the azimuthal displacement of the end point of the trajectory, with respect to
the beginning point, is of order the gyroradius. Thus, in the limit
in which the ratio of the gyroradius, $\rho$, to the variation length-scale of the
magnetic field,  $L$, tends to zero, the trajectory of the guiding centre
can be regarded as being  approximately closed,
 and the actual particle trajectory conforms very closely
to that of the guiding centre. Thus, the adiabatic invariant associated with
the bounce motion can be written
\begin{equation}
{\cal J} \simeq J = \oint p_\parallel\,ds,
\end{equation}
where the path of integration is along a field-line: from the equator to
the upper mirror point, back along the field-line to the lower mirror point, and
then back 
to the equator. Furthermore, $ds$ is an element of arc-length along the
field-line, and $p_\parallel\equiv {\bf p}\cdot{\bf b}$. 
Using ${\bf p} = m\,{\bf v} + e\,{\bf A}$, the above
expression yields
\begin{equation}
J = m\oint v_\parallel\,ds + e\,\oint {\bf A}_\parallel \,ds 
= m\oint v_\parallel\,ds + e\,{\Phi}.
\end{equation}
Here, ${\Phi}$ is the total magnetic flux enclosed by the curve---which,
in this case, is obviously zero. Thus, the so-called {\em second adiabatic
invariant}\/ or {\em longitudinal adiabatic invariant}\/ takes the form
\begin{equation}
J = m\oint v_\parallel\,ds.
\end{equation}
In other words, the second invariant is proportional to the loop integral
of the parallel (to the magnetic field) velocity taken over a bounce orbit.
Actually,  the above ``proof'' is not particularly  rigorous: the rigorous proof
that $J$ is an adiabatic invariant was first given by Northrop and Teller.\footnote{
T.G.~Northrop, and E.~Teller, Phys.\ Rev.\ {\bf 117}, 215 (1960).} It should
be noted, of course, that $J$ is only a constant of the motion
for particles trapped in the inner magnetosphere  provided that the
magnetospheric
magnetic field  varies on time-scales much longer than the bounce time,
$\tau_b$. Since the bounce time for MeV energy protons and electrons is,
at most, a few seconds, this is not a particularly onerous constraint.

\begin{figure}
\epsfysize=2in
\setbox0=\hbox{\epsffile{Chapter02/solarwind.ps}}
\centerline{\rotu0}
\caption{\em The distortion of the Earth's magnetic field by the solar wind.}\label{f6}
\end{figure}

The invariance of $J$ is of great importance for charged particle
dynamics in the Earth's inner magnetosphere. It turns out that the
Earth's magnetic field is distorted from pure axisymmetry by the action of
the solar wind, as illustrated in Fig.~\ref{f6}. Because of this asymmetry, there 
is no particular reason to believe that a particle will return to its
earlier trajectory as it makes a full rotation around the Earth.  In other words,
the particle may well end up on a different field-line when it returns to
the same azimuthal angle. However, at a given azimuthal angle, each 
field-line has a different length between mirror points, and a different
variation of the field-strength $B$ between the mirror points, for a particle
with given  energy ${\cal E}$ and magnetic moment $\mu$. Thus, each field-line
represents a different value of $J$ for that particle. So, if $J$ is
conserved, as well as ${\cal E}$ and $\mu$, then the particle
{\em must}\/ return to the same field-line after precessing around the Earth. 
In other words, the conservation of $J$ prevents charged particles from
spiraling radially in or out of the Van Allen belts as they rotate around the Earth.
This helps to explain the persistence of these belts. 

\section{Third Adiabatic Invariant}
It is clear, by now, that there is an adiabatic invariant associated
with every periodic motion of a charged particle in 
an electromagnetic field.
Now, we have just demonstrated that, as a consequence of $J$-conservation,
the drift orbit of a charged particle precessing around the Earth is approximately
closed, despite the fact that the Earth's magnetic field is non-axisymmetric. 
Thus, there must be a third adiabatic invariant associated with the
precession of particles around the Earth. Just as we can define a
guiding centre associated with a particle's gyromotion around field-lines,
we can also define a {\em bounce centre}\/ associated with a particle's bouncing
motion between mirror points. The bounce centre lies on the equatorial plane, and
orbits the Earth once every drift period, $\tau_d$. 
 We can write the third adiabatic invariant as
\begin{equation}
K \simeq \oint p_\phi\,ds,
\end{equation}
where the path of integration is the trajectory of the bounce centre around
the Earth. Note that the drift trajectory
effectively collapses onto the trajectory of the bounce centre 
in the limit in which $\rho/L\rightarrow 0$---all
of the particle's gyromotion and bounce motion  averages to zero.
Now $p_\phi = m\,v_\phi + e\,A_\phi$ is dominated by its second term,
since the drift velocity $v_\phi$ is very small. Thus,
\begin{equation}
K \simeq e\oint A_\phi\,ds = e\,{\Phi},
\end{equation}
where ${ \Phi}$ is the total {\em magnetic flux}\/ enclosed by the drift
trajectory ({\em i.e.}, the flux enclosed by the orbit of the bounce centre
around the Earth). The above ``proof'' is, again, not particularly rigorous---the invariance of ${\Phi}$ is demonstrated rigorously by Northrup.\footnote{
T.G.~Northrup, {\em The Adiabatic Motion of Charged Particles} (Interscience,
New York NY, 1963).} Note, of course, that ${\Phi}$ is only a constant
of the motion for particles trapped in the inner magnetosphere provided that the
magnetospheric magnetic field varies on time-scales much longer than
the drift period, $\tau_d$. Since the drift period for MeV energy 
protons and electrons is of order an hour, this is only likely
to be the case when the magnetosphere is relatively quiescent ({\em i.e.},
when there are no geomagnetic storms in progress). 

The invariance of ${\Phi}$ has interesting consequences for charged
particle dynamics in the Earth's inner magnetosphere. Suppose, for instance,  
that the
strength of the solar wind were to increase slowly ({\em i.e.}, on time-scales
significantly longer than the drift period), thereby, compressing the Earth's
magnetic field. The invariance of ${\Phi}$ would cause the charged particles
which constitute the Van Allen belts 
 to move radially inwards, towards the Earth, in order
to conserve the magnetic flux enclosed by their drift orbits. Likewise, a
slow decrease in the strength of the solar wind would cause an outward radial motion
of the Van Allen belts.

\section{Motion in Oscillating Fields}
We have seen that charged particles can be confined by a static magnetic field.
A somewhat more surprising fact is that charged
particles can also be confined by a rapidly
oscillating, inhomogeneous electromagnetic wave-field. In order to demonstrate this,
we again make use of our averaging technique. To lowest order, a particle
executes simple harmonic motion in response to an oscillating wave-field.
However, to higher order, any weak inhomogeneity in the field causes the restoring force
at one turning point to exceed that at the other. On average, this yields a net force
which acts on the {\em centre of oscillation}\/ of the particle.

Consider a spatially inhomogeneous electromagnetic wave-field oscillating at
frequency $\omega$:
\begin{equation}
{\bf E}({\bf r}, t) = {\bf E}_0({\bf r})\,\cos\omega t.
\end{equation}
The equation of motion of a charged particle placed in this field is written
\begin{equation}\label{e2.112}
m\,\frac{d{\bf v}}{dt}= e\,\left[{\bf E}_0({\bf r})\,\cos\omega t
+{\bf v}
\times {\bf B}_0({\bf r})\,\sin\omega t\right],
\end{equation}
where 
\begin{equation}\label{e2.113}
{\bf B}_0 = -\omega^{-1}\,\nabla\times{\bf E}_0,
\end{equation}
according to Faraday's law. 

In order for our averaging technique to be applicable, the electric field
${\bf E}_0$ experienced by the particle must remain approximately constant
during an oscillation. Thus,
\begin{equation}\label{e2.114}
({\bf v}\cdot\nabla)\,{\bf E} \ll \omega\,{\bf E}.
\end{equation}
When this inequality is satisfied, Eq.~(\ref{e2.113}) implies that the magnetic
force experienced by the particle
is smaller than the electric force by one order in the
expansion parameter. In fact, Eq.~(\ref{e2.114}) is equivalent to the requirement,
${\Omega}\ll \omega$, that the particle be unmagnetized. 

We now apply the averaging technique. We make the substitution $t\rightarrow\tau$
in the oscillatory terms, and seek a change of variables,
\begin{eqnarray}
{\bf r} &=& {\bf R} + \bxi({\bf R}, {\bf U}\, t,\tau),\\[0.5ex]
{\bf v} &=& {\bf U} + {\bf u}({\bf R}, {\bf U}\, t,\tau),
\end{eqnarray}
such that $\bxi$ and ${\bf u}$ are periodic functions of $\tau$
with vanishing mean. Averaging $d{\bf r}/dt = {\bf v}$ again yields
$d{\bf R}/dt = {\bf U}$ to all orders. To lowest order, the momentum evolution
equation reduces to
\begin{equation}
\frac{\partial{\bf u}}{\partial \tau} = \frac{e}{m}\,{\bf E}_0({\bf R})\,\cos\omega\tau.
\end{equation}
The solution, taking into account the constraints $\langle {\bf u}\rangle
=\langle\bxi\rangle = {\bf 0}$, is
\begin{eqnarray}
{\bf u} &=& \frac{e}{m\,\omega}\,{\bf E}_0\,\sin\omega\tau,\\[0.5ex]
\bxi &=& -\frac{e}{m\,\omega^2}\,{\bf E}_0\,\cos\omega\tau.
\end{eqnarray}
Here, $\langle\cdots\rangle\equiv(2\pi)^{-2}\oint(\cdots)\,d(\omega\tau)$ represents
an oscillation average. 

Clearly, there is no motion of the centre of oscillation to lowest order.
To first order, the oscillation average of Eq.~(\ref{e2.112}) yields
\begin{equation}
\frac{d{\bf U}}{dt} = \frac{e}{m} \left\langle (\bxi\cdot\nabla)
{\bf E} + {\bf u}\times{\bf B}\right\rangle,
\end{equation}
which reduces to
\begin{equation}\label{e2.120}
\frac{d{\bf U}}{dt} = -\frac{e^2}{m^2\,\omega^2}
\left[ ({\bf E}_0\cdot\nabla){\bf E}_0\,\langle\cos^2\omega\tau\rangle
+ {\bf E}_0\times(\nabla\times{\bf E}_0)\,\langle\sin^2\omega\tau\rangle
\right].
\end{equation}
The oscillation averages of the trigonometric functions are both equal to $1/2$. 
Furthermore, we have
$\nabla(|{\bf E}_0|^2/2)\equiv  ({\bf E}_0\cdot\nabla){\bf E}_0
+{\bf E}_0\times(\nabla\times{\bf E}_0)$. Thus, the equation of motion for
the centre of oscillation reduces to
\begin{equation}
m\,\frac {d{\bf U}}{dt} = -e\,\nabla{\Phi}_{\rm pond},
\end{equation}
where
\begin{equation}
{\Phi}_{\rm pond} = \frac{1}{4} \frac{e}{m\,\omega^2}\,|{\bf E}_0|^2.
\end{equation}
It is clear that the oscillation centre experiences a force, called the
{\em ponderomotive force}, which is proportional to the gradient
in the amplitude of the wave-field. The ponderomotive force is
independent of the sign of the charge, so both electrons and
ions can be confined in the same potential well. 

The total energy of the oscillation centre,
\begin{equation}
{\cal E}_{\rm oc} = \frac{m}{2}\,U^2 +e\,{\Phi}_{\rm pond},
\end{equation}
is conserved by the equation of motion (\ref{e2.120}). Note that the ponderomotive
potential energy is equal to the average kinetic energy of the
oscillatory motion:
\begin{equation}
e\,{\Phi}_{\rm pond} = \frac{m}{2}\,\langle u^2\rangle.
\end{equation}
Thus, the force on the centre of oscillation originates in a transfer of
energy from the oscillatory motion to the average motion.

Most of the important applications of the ponderomotive force occur in laser
plasma physics. 
For instance, a laser beam can propagate in a plasma provided that
its frequency exceeds the plasma frequency. If the beam is sufficiently
intense then plasma particles are repulsed from the centre of the
beam by the ponderomotive force. The resulting variation in the plasma
density gives rise to a cylindrical well in the index of refraction
which acts as a wave-guide for the laser beam.