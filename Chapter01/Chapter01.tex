\chapter{Introduction}\label{s1}

\section{Sources}
The major sources for this course are:
\begin{description}
\item [{\sf The Theory of Plasma Waves}:] T.H.~Stix, 1st Ed.\ (McGraw-Hill,
New York NY, 1962). 
\item [{\sf Plasma Physics}:] R.A.~Cairns (Blackie, Glasgow UK, 1985).
\item [{\sf The Framework of Plasma Physics}:] R.D.~Hazeltine, and F.L.~Waelbroeck (Westview,
Boulder CO, 2004).
\end{description}
Other sources include:
\begin{description}
\item [{\sf The Mathematical Theory of Non-Uniform Gases}:] S.~Chapman, and
T.G.~Cowling (Cambridge University Press, Cambridge UK, 1953).
\item [{\sf Physics of Fully Ionized Gases}:] L.~Spitzer, Jr., 1st Ed.\
(Interscience, New York NY, 1956).
\item [{\sf Radio Waves in the Ionosphere}:] K.G.~Budden (Cambridge University Press,
Cambridge UK, 1961). 
\item [{\sf The Adiabatic Motion of Charged Particles}:] T.G.~Northrop 
(Interscience, New York NY, 1963).
\item [{\sf Coronal Expansion and the Solar Wind}:] A.J.~Hundhausen (Springer-Verlag,
Berlin, 1972).
\item [{\sf Solar System Magnetic Fields}:] E.R.~Priest, Ed.\ (D.~Reidel
Publishing Co., Dordrecht, Netherlands, 1985).
\item [{\sf Lectures on Solar and Planetary Dynamos}:] M.R.E.~Proctor,
and A.D.~Gilbert, Eds.\ (Cambridge University Press,
Cambridge UK, 1994). 
\item [{\sf Introduction to Plasma Physics}:] R.J.~Goldston, and P.H.~Rutherford
(Institute of Physics Publishing, Bristol UK, 1995).
\item [{\sf Basic Space Plasma Physics}:] W.~Baumjohann, and R.~A.~Treumann
(Imperial College Press, London UK, 1996). 
\end{description}

\section{What is Plasma?}
The electromagnetic force is generally observed to create {\em structure}: 
{\em e.g.},
stable atoms and molecules, crystalline solids. In fact, the most widely studied
consequences of  the electromagnetic force form the subject matter of 
Chemistry and Solid-State
Physics, which are both disciplines developed to understand essentially static
structures. 

Structured systems have binding energies larger than the ambient thermal
energy. Placed in a sufficiently hot environment, they decompose: {\em e.g.},
crystals melt, molecules disassociate. At temperatures near or exceeding
atomic ionization energies,  atoms similarly decompose into
negatively charged electrons and positively charged ions. These charged
particles are by no means free: in fact, they are strongly
affected by each others' electromagnetic fields. Nevertheless, because
the charges are no longer bound, their assemblage becomes
capable of collective motions of great vigor and complexity. Such
an assemblage is termed a {\em plasma}. 

Of course, bound systems can display extreme complexity of
structure: {\em e.g.}, a protein molecule. Complexity in a plasma
is somewhat different, being expressed {\em temporally}\/ as much as
{\em spatially}. It is predominately characterized by the
excitation of an enormous variety of {\em collective}\/ dynamical modes.

Since thermal decomposition breaks interatomic bonds before ionizing,
most terrestrial  plasmas begin as gases. In fact, a plasma is sometimes
defined as a gas that is sufficiently ionized to exhibit plasma-like
behaviour. Note that plasma-like behaviour  ensues after a remarkably
small fraction of the gas has undergone ionization. Thus, fractionally
ionized gases exhibit most of the exotic phenomena characteristic of
fully ionized gases. 

Plasmas resulting from ionization of neutral gases generally contain equal numbers
of positive and negative charge carriers. In this situation, the
oppositely charged fluids are strongly coupled, and tend to electrically   neutralize
one another  on macroscopic length-scales. Such plasmas
are termed {\em quasi-neutral}\/ (``quasi'' because the small deviations
from exact neutrality  have important dynamical consequences for certain types
of plasma mode). Strongly {\em non-neutral}\/ plasmas, which may even
contain charges of only one sign, occur primarily in laboratory experiments:
their equilibrium depends on the existence of intense magnetic fields,
about which the charged fluid rotates. 

It is sometimes remarked that 95\% (or 99\%, depending on whom you are
trying to impress) of the baryonic content of the Universe consists of plasma. This statement has the
double merit of being extremely  flattering to Plasma Physics, and
quite  impossible to disprove (or verify). Nevertheless, it is worth
 pointing out the prevalence of the plasma state. In earlier epochs
of the Universe, everything was plasma. In the present epoch, stars, nebulae,
and even interstellar space, are filled with plasma. The Solar System
is also permeated  with plasma, in the form of the solar wind, and the
Earth is completely surrounded by plasma trapped within its magnetic field. 

Terrestrial plasmas are also  not hard to find. They occur in lightning,
fluorescent lamps, a variety of laboratory experiments, and a growing
array of industrial processes. In fact, the glow discharge has recently
become the mainstay of the micro-circuit fabrication industry. 
Liquid and even solid-state systems can occasionally display the
collective electromagnetic effects that characterize plasma:
{\em e.g.}, liquid mercury exhibits many dynamical modes, such
as Alfv\'{e}n waves,  which occur in conventional  plasmas. 

\section{Brief History of Plasma Physics}
When blood is cleared of its various corpuscles there remains a transparent liquid,
which was named {\em plasma}\/ (after the Greek word $\pi\lambda\alpha\sigma\mu\alpha$,
which means ``moldable substance'' or ``jelly'') by the great Czech medical
scientist, Johannes Purkinje (1787-1869). The Nobel prize winning
American chemist Irving Langmuir first used this term to describe an
ionized gas in 1927---Langmuir was reminded of the way blood plasma carries
red and white corpuscles by the way an electrified fluid carries
electrons and ions. Langmuir, along with his colleague Lewi Tonks, was 
investigating the physics and chemistry of tungsten-filament light-bulbs, with a view to finding a way to greatly
extend the lifetime of the filament (a goal which he
eventually achieved). In the process, he developed the theory of
 {\em plasma sheaths}---the
boundary layers which form between  ionized plasmas and  solid surfaces. 
He also  discovered that certain regions of
a plasma discharge tube exhibit periodic variations of the electron
density, which we nowadays term {\em Langmuir waves}. This was the
genesis of Plasma Physics. Interestingly enough,  
Langmuir's research 
 nowadays forms the theoretical basis of most {\em plasma processing}\/ techniques for
fabricating integrated circuits. 
After Langmuir, plasma research gradually spread in other
directions, of which {\em five}\/ are particularly significant.

Firstly, the development of radio broadcasting led to the discovery of
the Earth's {\em ionosphere}, a layer of partially ionized gas in the
upper atmosphere which reflects radio waves, and is responsible for the fact that
radio signals
 can be received when the transmitter is over the horizon. Unfortunately,
the ionosphere also occasionally absorbs and distorts radio waves. 
For instance, the Earth's
magnetic field causes waves  with different
polarizations (relative to the orientation of the magnetic field) to propagate at
different velocities, an effect which can give rise to ``ghost signals'' ({\em
i.e.}, signals which arrive a little before, or a little after, the main signal). 
In order to understand, and possibly correct, some of the deficiencies
in  radio communication,
various scientists, such as E.V.~Appleton and K.G.~Budden, systematically
developed the theory of 
electromagnetic wave propagation  through  non-uniform magnetized plasmas.

Secondly, astrophysicists quickly 
recognized that much of the Universe consists of plasma,
and, thus, that a better understanding of astrophysical phenomena requires
a better grasp of plasma physics. The pioneer in this
field was Hannes Alfv\'{e}n, who around 1940 developed the theory of
{\em magnetohydrodyamics}, or MHD, in which plasma is treated essentially
 as a conducting
fluid. This theory has been both widely and successfully employed to
investigate sunspots, solar flares, the solar wind, star formation, and a
host of other topics in astrophysics. Two topics of particular
interest in MHD theory are {\em magnetic reconnection}\/ and
{\em dynamo theory}. Magnetic reconnection is a process by which magnetic field-lines
suddenly change their topology: it can give rise to the sudden
conversion of a great deal of magnetic energy into thermal energy, 
as well as the acceleration of some charged particles to extremely high energies,
and is
generally thought to be the basic  mechanism behind  solar flares. Dynamo theory
studies how the motion of an MHD fluid can give rise to the generation of
a macroscopic magnetic field. This process is important because
both the terrestrial and solar magnetic fields would decay away
comparatively rapidly (in astrophysical terms)
were they not maintained by dynamo action. The Earth's
magnetic field is maintained by the motion of its molten core, which can
be treated as an MHD fluid to a reasonable approximation. 

Thirdly, the creation of the hydrogen bomb in 1952  generated a great deal of
interest in {\em controlled thermonuclear fusion}\/ as a possible power
source for the future. At first, this research was carried out secretly,
and independently, by the United States, the Soviet Union, and Great Britain. 
However, in 1958 thermonuclear fusion research was declassified, leading to
the publication of a number of immensely important and influential papers in the late
1950's and the early 1960's. Broadly speaking, theoretical plasma physics
first emerged as a mathematically rigorous discipline in these years. 
Not surprisingly, Fusion physicists are mostly concerned with 
understanding how a thermonuclear plasma can be trapped---in most
cases by a magnetic field---and
investigating the many plasma instabilities which may allow it to escape. 

Fourthly,  James A.~Van Allen's discovery in 1958  of the Van Allen radiation belts 
surrounding the
Earth, using data transmitted by the  U.S.\ Explorer satellite,  marked the
start of the systematic exploration of the Earth's magnetosphere via
satellite, and opened up the field of {\em space plasma physics}. 
Space scientists borrowed the theory of plasma trapping by a magnetic
field from fusion research, the theory of plasma waves from ionospheric
physics, and the notion of magnetic reconnection as a mechanism for
energy release and particle acceleration from astrophysics. 

Finally, the development of high powered lasers in the 1960's 
opened up the field of {\em laser plasma physics}. When a high powered laser
beam strikes a solid target, material is immediately ablated, and a plasma
forms at the boundary between the beam and the target. Laser plasmas tend to  have
fairly extreme properties ({\em e.g.}, densities
characteristic of solids) not found in more conventional plasmas. 
A major application of laser plasma physics is the approach to fusion
energy known as {\em inertial confinement fusion}. In this approach, 
tightly focused laser
beams are used to implode a small solid target until the densities and
temperatures characteristic of nuclear fusion ({\em i.e.},  the centre of
a hydrogen bomb) are achieved. Another interesting application
of  laser plasma physics is the
use of the extremely strong electric fields generated when a high intensity laser pulse passes
through a plasma to accelerate particles. High-energy physicists 
hope to use plasma acceleration
techniques to dramatically reduce the size and cost of particle accelerators.

\section{Basic Parameters}
Consider an idealized plasma consisting of an equal number of electrons, with
mass $m_e$ and charge $-e$ (here, $e$ denotes the {\em magnitude}\/ of the electron
charge), and ions, with  mass $m_i$ and charge $+e$. We do not necessarily
demand that the system has attained thermal equilibrium, but nevertheless use
the symbol
\begin{equation}
T_s \equiv \frac{1}{3} \,m_s\, \langle \,v_s^{\,2}\rangle
\end{equation}
to denote a {\em kinetic temperature}\/ measured in energy units ({\em i.e.}, joules).
Here, $v$ is a particle speed, and the angular brackets denote an
ensemble average. The kinetic temperature of species $s$ is essentially
the average kinetic energy of particles of
this species. In plasma physics, kinetic temperature
is invariably measured in {\em electron-volts}\/ (1 joule is equivalent to
$6.24\times 10^{18}$\,eV). 

Quasi-neutrality demands that
\begin{equation}
n_i \simeq n_e \equiv n,
\end{equation}
where $n_s$ is the number density ({\em i.e.}, the number of particles 
per cubic meter) of species $s$. 

Assuming that both ions and electrons are characterized by the
same $T$ (which is, by no means, always the case in plasmas), we can
estimate typical particle speeds via the so-called {\em thermal speed},
\begin{equation}
v_{ts} \equiv \sqrt{2\,T/m_s}.
\end{equation}
Note that the ion thermal speed is usually far smaller than the
electron thermal speed:
\begin{equation}
v_{ti} \sim \sqrt{m_e/m_i}\,\,v_{te}.
\end{equation}

Of course, $n$ and $T$ are generally functions of position in a plasma.

\section{Plasma Frequency}\label{s1.5}
The {\em plasma frequency},
\begin{equation}\label{e1.5}
\omega_p^{~2} = \frac{n\,e^2}{\epsilon_0\,m},
\end{equation}
is the most fundamental time-scale in plasma physics. Clearly, there is a
different plasma frequency for each species. However, the relatively
fast electron frequency is, by far,  the most important, and references
to ``the plasma frequency'' in text-books invariably mean the {\em electron}\/ plasma
frequency.

It is easily seen that $\omega_p$ corresponds to the typical electrostatic
oscillation frequency of a given species in response to a small charge separation. 
For instance, consider a one-dimensional situation in which a slab 
consisting entirely of one charge species is displaced from its
quasi-neutral position by an infinitesimal distance $\delta x$. The
resulting charge density which develops on the  leading face of the slab is
$\sigma=e\,n\,\delta x$. An equal and opposite charge density develops on the
opposite face. 
The $x$-directed electric field generated inside
the slab is of magnitude $E_x= -\sigma/\epsilon_0 =- e\,n\,\delta x/\epsilon_0$.
Thus, Newton's law applied to an individual particle inside the slab yields
\begin{equation}
m\,\frac{d^2 \delta x}{dt^2} = e\,E_x = -m\,\omega_p^{~2}\,\delta x,
\end{equation}
giving $\delta x = (\delta x)_0\,\cos\,(\omega_p\,t)$.

Note that plasma oscillations will only be observed if the plasma system
is studied  over time periods $\tau$ longer than the plasma period
$\tau_p\equiv 1/\omega_p$, and if external actions change the system at
a rate no faster than $\omega_p$. In the opposite case, one is clearly studying
something other than plasma physics ({\em e.g.}, nuclear reactions), and the
system cannot  not usefully be considered to be a plasma. Likewise, observations
over length-scales $L$ shorter than the distance $v_t\,\tau_p$ traveled
by a typical plasma particle during a plasma period will also not
detect plasma behaviour. In this case, particles will exit the
system before completing a plasma oscillation. This distance, which is the
spatial equivalent to $\tau_p$, is called the {\em Debye length}, and takes
the form
\begin{equation}
\lambda_D \equiv \sqrt{T/m}\,\,\omega_p^{-1}.
\end{equation}
Note that
\begin{equation}\label{e1.8}
\lambda_D = \sqrt{\frac{\epsilon_0\,T}{n\,e^2}}
\end{equation}
is independent of mass, and therefore generally comparable for different
species. 

Clearly, our idealized system can  only usefully be considered to be a plasma provided 
that
\begin{equation}\label{e1.9}
\frac{\lambda_D}{L} \ll 1,
\end{equation}
and
\begin{equation}
\frac{\tau_p}{\tau}\ll 1.
\end{equation}
Here, $\tau$ and $L$ represent the typical time-scale and length-scale of the
process under investigation.

It should be noted that, despite the conventional requirement (\ref{e1.9}),
plasma physics {\em is}\/  capable of considering structures on the Debye scale.
The most important example of this is the {\em Debye sheath}: {\em i.e.}, the
boundary layer which surrounds
a plasma confined  by a material surface. 

\section{Debye Shielding}
Plasmas generally do not contain strong electric fields in their
rest frames. The shielding of an external electric field from the
interior of a plasma can be viewed as a result of  high plasma conductivity:
{\em i.e.}, plasma current generally flows freely enough to short out interior electric fields.
However, it is more useful to consider the shielding as a {\em dielectric}\/ 
phenomena: {\em i.e.}, it is the {\em polarization}\/ of the plasma medium, and the
associated redistribution of space charge, which prevents penetration by an
external electric field. Not surprisingly, the length-scale associated with such
shielding is the Debye length.

Let us consider the simplest possible example. Suppose that a quasi-neutral
plasma is sufficiently close to thermal equilibrium that its particle
densities are distributed according to the Maxwell-Boltzmann law,
\begin{equation}
n_s = n_0\,{\rm e}^{-e_s\,{\Phi}/T},
\end{equation}
where ${\Phi}({\bf r})$ is the electrostatic potential, and $n_0$ and
$T$ are constant. From $e_i = - e_e = e$, it is clear that quasi-neutrality
requires the equilibrium potential to be a constant. Suppose that this
equilibrium potential is  perturbed, by an amount $\delta{\Phi}$,
by a small, localized charge density $\delta\rho_{\rm ext}$. The total
perturbed charge density is written
\begin{equation}
\delta\rho = \delta\rho_{\rm ext} + e\,(\delta n_i - \delta n_e)
= \delta\rho_{\rm ext} -2 \,e^2\, n_0 \,\delta{\Phi}/T.
\end{equation}
Thus, Poisson's equation yields
\begin{equation}
\nabla^2 \delta {\Phi} = - \frac{\delta\rho}{\epsilon_0} =
-\left(\frac{\delta\rho_{\rm ext} -2 \,e^2\, n_0\, \delta{\Phi}/T}{\epsilon_0}
\right),
\end{equation}
which reduces to
\begin{equation}
\left(\nabla^2 - \frac{2}{\lambda_D^{~2}}\right) \delta {\Phi}
= - \frac{\delta\rho_{\rm ext}}{\epsilon_0}.
\end{equation}
If the perturbing  charge density actually consists of a point charge $q$, located
at the origin, so that $\delta\rho_{\rm ext} = q\,\delta({\bf r})$, then
the solution to the above equation is written
\begin{equation}
\delta{\Phi}(r) = \frac{q}{4\pi\epsilon_0\,r}\,\, {\rm e}^{-\sqrt{2}\,r/\lambda_D}.
\end{equation}
Clearly, the Coulomb potential of the perturbing point charge $q$ is
shielded on distance scales longer than the Debye length by a {\em shielding
cloud}\/ of approximate radius $\lambda_D$ consisting of charge of the opposite sign. 

Note that the above argument, by treating $n$ as a continuous function, implicitly
assumes that there are many particles in the shielding cloud. Actually,
Debye shielding remains statistically significant, and physical, in the
opposite limit in which the cloud is barely populated. In the latter case,
it is the
{\em probability}\/ of observing charged particles within a Debye length of the
perturbing charge which is modified. 

\section{Plasma Parameter}\label{s1.7}
Let us define the average distance between particles,
\begin{equation}\label{e1.16}
r_d \equiv n^{-1/3},
\end{equation}
and the distance of closest approach,
\begin{equation}\label{e1.17}
r_c\equiv \frac{e^2}{4\pi\epsilon_0\, T} .
\end{equation}
Recall that $r_c$ is the distance at which the Coulomb energy
\begin{equation}
U(r,v) = \frac{1}{2}\, mv^2 - \frac{e^2}{4\pi\epsilon_0 \,r}
\end{equation}
of one charged particle in the electrostatic field of another vanishes.
Thus, $U(r_c,v_t) = 0$. 

The significance of the ratio $r_d/r_c$ is readily understood. When this
ratio is small, charged particles are dominated by one another's electrostatic
influence more or less continuously, and their kinetic energies are small
compared to the interaction potential energies. Such plasmas are termed
{\em strongly coupled}. On the other hand, when the ratio is large,
strong electrostatic interactions between individual particles are
occasional and relatively rare events. A typical particle
is electrostatically influenced by all of the other particles within its
Debye sphere, but this interaction very rarely causes any sudden change in its motion.
Such plasmas are termed {\em weakly coupled}. It is possible to
describe a weakly coupled plasma using a standard Fokker-Planck equation
 ({\em i.e.}, the same type of equation as is conventionally used to describe a
neutral gas).
Understanding the strongly coupled limit is far more difficult, and will not
be attempted in this course. Actually, a strongly coupled plasma has more in
common with a liquid than a conventional weakly coupled plasma. 

Let us define the {\em plasma parameter}\/
\begin{equation}\label{e1.19}
{\Lambda} = 4\pi\, n\, \lambda_D^{~3}.
\end{equation}
This dimensionless parameter is obviously equal to the typical number of particles
contained in a Debye sphere. However, Eqs.~(\ref{e1.8}), (\ref{e1.16}), (\ref{e1.17}), and (\ref{e1.19})
 can be combined to
give
\begin{equation}\label{e1.20}
{\Lambda} = \frac{\lambda_D}{r_c}=\frac{1}{\sqrt{4\pi}} \left(\frac{r_d}{r_c}\right)^{3/2} =
\frac{4\pi \,\epsilon_0^{~3/2}}{e^3} \frac{T^{3/2}}{n^{1/2}}.
\end{equation}
It can be seen that the case ${\Lambda}\ll 1$, in which the Debye
sphere is sparsely populated, corresponds to a strongly coupled plasma.
Likewise, the case ${\Lambda}\gg 1$, in which the Debye sphere is
densely populated, corresponds to a weakly coupled plasma. It can also
be appreciated, from Eq.~(\ref{e1.20}), that strongly coupled plasmas
tend to be cold and dense, whereas weakly coupled plasmas are diffuse and
hot. Examples of strongly coupled plasmas include solid-density laser ablation
plasmas, the very ``cold''  ({\em i.e.}, with kinetic temperatures
similar to the ionization energy) plasmas found in ``high pressure'' arc discharges, and the plasmas which constitute the 
atmospheres of collapsed objects such as white dwarfs and neutron stars.
On the other hand, the hot diffuse plasmas typically encountered in ionospheric
physics, astrophysics, nuclear fusion, and space plasma physics are invariably
weakly coupled. Table~\ref{t1} lists the key parameters for some typical
weakly coupled plasmas.

\begin{table}\centering
\begin{tabular}{c|ccccc}
& $n({\rm m}^{-3})$ & $T({\rm eV})$ & $\omega_p({\rm sec}^{-1})$ & $\lambda_D({\rm m})$ & ${\Lambda}$ \\[0.5ex] \hline
&&&&&\\[-1ex]
Interstellar              & $10^6$              & $10^{-2}$             & $6\times 10^4$                   & $0.7$                       & $4\times 10^6$ \\
Solar Chromosphere &  $10^{18}$ & $2$ & $6\times 10^{10}$ & $5\times 10^{-6}$ & $2\times 10^3$\\
Solar Wind (1AU) & $10^7$ & $10$ & $2\times 10^5$ & $7$ & $5\times 10^{10}$\\
Ionosphere              & $10^{12}$       & $0.1$                   & $6\times 10^7$      &  $2\times 10^{-3}$ & $1\times 10^5$\\
Arc discharge          & $10^{20}$       & $1$                     & $6\times 10^{11}$ & $7\times 10^{-7}$  & $5\times 10^2$ \\
Tokamak      & $10^{20}$        & $10^4 $             & $6\times 10^{11}$ & $7\times 10^{-5}$ & $4\times 10^8$ \\
Inertial Confinement & $10^{28}$ & $10^4$ & $6\times 10^{15}$ & $7\times 10^{-9}$ & $5\times 10^4$\\
\end{tabular}
\caption{{\em Key parameters for some typical weakly coupled plasmas.}}
\label{t1}
\end{table}

In conclusion, characteristic {\em collective}\/
plasma behaviour is only observed on time-scales
longer than the plasma period, and on length-scales larger than the Debye length.
The statistical character of this behaviour is controlled by the plasma parameter.
Although $\omega_p$, $\lambda_D$, and ${\Lambda}$ are the three most fundamental
plasma parameters, there are a number of other parameters which are worth
mentioning. 

\section{Collisions}
Collisions between charged particles in a plasma differ fundamentally from
those between molecules in a neutral gas because of the long range of the
Coulomb force. In fact, it is clear from the discussion in Sect.~\ref{s1.7} that
{\em binary}\/ collision processes can only be defined for weakly coupled
plasmas. Note, however, that binary collisions in weakly coupled plasmas
are still modified by collective effects---the many-particle
process of Debye shielding enters in a crucial manner. Nevertheless, for
large ${\Lambda}$ we can speak of binary collisions, and therefore of
a {\em collision frequency}, denoted by $\nu_{ss'}$. Here,
$\nu_{ss'}$ measures the rate at which particles of species $s$ are scattered 
by those of species $s'$. When specifying only a single subscript, one is
generally referring to the {\em total}\/ collision rate for that species,
including impacts with all other species. Very roughly,
\begin{equation}
\nu_s \simeq \sum_{s'} \nu_{ss'}.
\end{equation}
The species designations are generally important. For instance,
the relatively small electron mass implies that, for unit ionic charge and
comparable species temperatures,
\begin{equation}
\nu_e \sim \left(\frac{m_i}{m_e}\right)^{1/2}\!\nu_i.
\end{equation}
Note that the collision frequency $\nu$ measures the frequency with which a
particle trajectory undergoes a {\em major}\/ angular change due to
Coulomb interactions with other particles. Coulomb collisions are, in fact,
predominately small angle scattering events, so the collision frequency
is {\em not}\/ the inverse of the typical time between collisions. Instead, it is
the inverse of the typical time needed for enough collisions to occur that the particle trajectory is deviated through $90^\circ$.
For this reason, the collision frequency is sometimes termed the ``$90^\circ$ scattering
rate.'' 

It is conventional to define the {\em mean-free-path},
\begin{equation}
\lambda_{\rm mfp} \equiv v_t/\nu.
\end{equation}
Clearly, the mean-free-path measures the typical distance a particle
travels between ``collisions'' ({\em i.e.},
$90^\circ$ scattering events). A collision-dominated, or {\em collisional},
plasma is simply one in which
\begin{equation}
\lambda_{\rm mfp} \ll L,
\end{equation}
where $L$ is the observation length-scale. The opposite limit of
large mean-free-path is said to correspond to a {\em collisionless}\/ plasma. 
Collisions greatly simplify plasma behaviour by driving the system towards
statistical equilibrium, characterized by Maxwell-Boltzmann distribution
functions. Furthermore, short mean-free-paths generally ensure that plasma
transport is {\em local}\/ ({\em i.e.}, diffusive) in nature, which is a considerable
simplification. 

The typical magnitude of the collision frequency is
\begin{equation}\label{e1.25}
\nu \sim \frac{\ln{\Lambda}}{{\Lambda}}\,\omega_p.
\end{equation}
Note that $\nu\ll \omega_p$ in a weakly coupled plasma. It follows that
collisions do not seriously interfere with plasma oscillations in such systems.
On the other hand, Eq.~(\ref{e1.25}) implies that $\nu\gg \omega_p$ in a strongly
coupled plasma, suggesting that collisions effectively prevent plasma
oscillations in such systems. This accords well with our basic picture of a strongly
coupled plasma as a system 
dominated by Coulomb interactions which does not exhibit conventional
plasma dynamics.
It follows from Eqs.~(\ref{e1.5}) and (\ref{e1.20}) that
\begin{equation}
\nu \sim \frac{e^4\,\ln{\Lambda}}{4\pi\epsilon_0^{~2}\, m^{1/2}}
\frac{n}{T^{3/2}}.
\end{equation}
Thus, diffuse, high temperature plasmas tend to be collisionless, whereas dense,
low temperature plasmas are more likely to be collisional. 

Note that whilst collisions are crucial to the confinement and dynamics ({\em e.g.},
sound waves) of neutral gases, they play a far less important role in plasmas. 
In fact, in many plasmas the magnetic field effectively plays the role that 
collisions play in a neutral gas. In such plasmas, charged particles are
constrained from moving perpendicular to the field by their small Larmor
orbits, rather than by collisions. Confinement along the field-lines is
more difficult to achieve, unless the field-lines form closed loops
(or closed surfaces). Thus, it makes sense to talk about a
``collisionless plasma,'' whereas it makes little sense to talk about
a ``collisionless neutral gas.'' Note that many plasmas are collisionless to
a very good approximation, especially those encountered in astrophysics and space plasma
physics contexts. 

\section{Magnetized Plasmas}
A {\em magnetized}\/ plasma is one in which the ambient magnetic field
${\bf B}$ is strong enough to significantly alter particle
trajectories. In particular,
magnetized plasmas are {\em anisotropic}, responding differently to
forces which are parallel and perpendicular to the direction
of ${\bf B}$. Note that a magnetized plasma moving with
mean velocity ${\bf V}$ contains an electric field
${\bf E} = - {\bf V}\times {\bf B}$ which is {\em not affected}\/
by Debye shielding. Of course, in the rest frame of the plasma the electric
field is essentially zero. 

As is well-known, charged particles respond to the Lorentz force,
\begin{equation}
{\bf F} = q\,{\bf v}\times {\bf B},
\end{equation}
by freely streaming in the direction of ${\bf B}$, whilst executing
circular Larmor orbits, or {\em gyro-orbits}, in the plane perpendicular to ${\bf B}$. 
As the field-strength increases, the resulting helical orbits become more
tightly wound, effectively tying particles to magnetic field-lines. 

The typical Larmor radius, or {\em gyroradius}, of a charged particle 
gyrating in a
magnetic field is given by
\begin{equation}
\rho\equiv \frac{v_t}{\Omega},
\end{equation}
where
\begin{equation}
{\Omega} = e B/m
\end{equation}
is the cyclotron frequency, or {\em gyrofrequency}, associated with the
gyration. As usual, there is a distinct gyroradius for each species.
When species temperatures are comparable, the electron gyroradius is
distinctly smaller than the ion gyroradius:
\begin{equation}
\rho_e \sim \left(\frac{m_e}{m_i}\right)^{1/2}\!\rho_i.
\end{equation}

A plasma system, or process, is said to be {\em magnetized}\/ if its
characteristic length-scale $L$ is large compared to the gyroradius.
In the opposite limit, $\rho\gg L$, charged particles  have essentially
straight-line trajectories. Thus, the ability of the magnetic field to
significantly affect particle trajectories is measured by the
{\em magnetization parameter}
\begin{equation}
\delta \equiv \frac{\rho}{L}.
\end{equation}

There are some cases of interest in which the electrons are magnetized, but the
ions are not. However, a ``magnetized'' plasma conventionally refers to
one in which both species are magnetized. This state is generally achieved
when
\begin{equation}
\delta_i \equiv \frac{\rho_i}{L} \ll 1.
\end{equation}

\section{Plasma Beta}\label{s1.10}
The fundamental measure of a magnetic field's effect on a plasma is the
magnetization parameter $\delta$. The fundamental measure of the
inverse effect is called $\beta$, and is defined as the ratio
of the thermal energy density $n\,T$ to the magnetic energy
density $B^2/2\,\mu_0$. It is conventional to identify the plasma
energy density with the pressure,
\begin{equation}
p \equiv n\,T,
\end{equation}
as in an ideal gas, and to define a separate $\beta_s$ for each plasma
species. Thus,
\begin{equation}
\beta_s = \frac{2\,\mu_0\,p_s}{B^2}.
\end{equation}
The total $\beta$ is written
\begin{equation}
\beta = \sum_s \beta_s.
\end{equation}
